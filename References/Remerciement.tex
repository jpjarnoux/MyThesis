\chapter*{Remerciements}
\addcontentsline{toc}{chapter}{Remerciements}

Je commencerai par remercier toutes les personnes qui ont contribué de près ou de loin à la conclusion de ce chapitre de ma vie. La thèse de doctorat est une expérience unique, que j'ai pu vivre en y prenant beaucoup de plaisir. Je retiendrai de ces années que le monde de la recherche est un monde vivant et dynamique, où chaque personne contribue à la réussite et au progrès collectif. À toutes les personnes qui ne verraient pas leurs noms cités, veuillez m'excuser par avance et sachez que je garde en mémoire toutes les personnes qui ont pu me soutenir ces dernières années.

Je me dois d'abord (mais je le fais bien volontiers) de remercier David Vallenet, directeur du LABGeM et directeur de ma thèse. Merci de m'avoir fait confiance pour mener à bien ces projets de recherches. J'ai beaucoup appris à tes côtés, sur le plan technique, mais aussi sur les rouages du monde académique. Je me souviendrai particulièrement de ma première conférence internationale à FEMS où tu m'as accompagné.

Je remercie aussi Alexandra Calteau, chercheuse au LABGeM et co-directrice de ma thèse. Au-delà de toutes les connaissances biologiques et bioinformatiques que tu m'as partagées, je garderai l'image d'une personne extrêmement humaine qui a su me présenter des opportunités pour me former à devenir un chercheur dans tous ces aspects. Tu as su être plus que compréhensive sur des événements personnels qui ont certainement impacté ma thèse et pour ça je te remercie énormément.

Merci à Jean Mainguy, tu as été d'une aide plus que bienvenue pendant tous mes travaux de thèse. Je pense sincèrement que tu as abattu le travail de plusieurs personnes à toi tout seul et sans ça je ne sais pas si les outils de pangénomique du LABGeM seraient dans leur état actuel. 

À Adelme Bazin et Guillaume Gautreau, je vous remercie de ne pas avoir quitté le bateau après votre thèse et d'être restés disponibles pour mes questions et pour tout le travail que nous avons pu accomplir sur PPanGGOLiN. Je vous tiens en haute estime, vous m'impressionnez toujours par vos connaissances, vos compétences et aussi par votre sympathie.

Pour Laura Bry et Quentin Fernandez De Grado, j'espère avoir su être un bon encadrant et que vous gardiez un bon souvenir de votre passage au LABGeM. Sachez que j'ai aussi beaucoup appris à vos côtés. Je vous souhaite toute la réussite possible pour votre avenir et j'espère vous croiser à l'occasion.

À Eddy Élisée, merci pour ta joie de vivre et pour avoir été toujours été le premier arrivé au laboratoire pour prendre le premier café. Tu es une personne rayonnante avec qui on aimerait travailler plus souvent ou même juste faire tourner quelques molécules.

Alexandre Protat, tu restes un membre honoraire du LABGeM. Merci d'avoir organisé tous ces GenoPub, et merci pour toutes ces discussions politiques où même si nos opinions étaient parfois opposées, nous avons pu parler sans amertume.

Merci à tous les autres membres du LABGeM. David Roche pour ta sympathie, ton calme et ta bienveillance. Stéphanie Fouteau, pour ton sourire et ta gentillesse. Zoé Rouy, pour tes conseils et ton savoir. Aurélie Génin-Lajus, pour ton énergie à revendre. Marc Stam, pour nos discussions autour du café. À tous les membres passés et présents, encore merci.

Je remercie également les membres de mon comité de suivi : Hélène Chiapello, Sophie Abby et Vincent Lacroix. Vous avez été de précieux conseils et vous avez toujours jugé mon travail avec honnêteté et bienveillance. J'espère que le résultat final sera à la hauteur des promesses et que nous pourrons nous revoir à l'occasion.

Je remercie tous mes amis qui m'ont supporté pendant ces 3 dernières années. Merci à Maud Repellin pour avoir été présente lorsque j'en avais besoin. Merci à Chloé Beaumont, Florian Jeanneret et Alba Caparros-Roissard pour nos moments de partage d'expérience de doctorant.

Merci à tous les membres de ma famille qui ont su me soutenir, chacun à leur manière. Merci à mon frère et ma s\oe ur qui, même si mes travaux de recherches ne les passionnaient pas, ont bien essayé de me supporter. Merci à mes parents, d'avoir cru en moi et de m'avoir permis de saisir les opportunités qui se présentaient à moi. Merci à mes grands-parents pour leur soutien moral dans toutes les difficultés.

Pour terminer, je ne peux que remercier ma femme Helen qui a toujours été présente pour me soutenir dans toutes les difficultés que j'ai pu rencontrer pendant ma thèse. Même si ce n'était pas toujours évident de comprendre pourquoi j'ai choisi ce métier et cette voie, tu as toujours pensé d'abord à moi et à mon bonheur. Au moment où j'écris ces lignes, tu me prépares le plus beau des cadeaux et je n'ai pas les mots pour t'exprimer ma reconnaissance.

Merci bien sûr à vous, lecteur de ce manuscrit de thèse, j'espère que celui-ci sera à la hauteur de vos attentes.