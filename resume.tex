

% page des résumés à garder en 2ème page. Si les résumés sont trop longs pour tenir sur une seule et même page, on peut mettre un résumé par page
\thispagestyle{empty}
\newgeometry{top=1.5cm, bottom=1.25cm, left=2cm, right=2cm}

\footnotesize
\noindent 
%*****************************************************
%***** LOGO DE L'ED À CHANGER IMPÉRATIVEMENT *********
%*****************************************************
\includegraphics[height=2.45cm]{images/logos/logo_usp_SDSV.png}
%*****************************************************

\begin{mdframed}[linecolor=Prune,linewidth=1]

\textbf{Titre :} Méthodes d’analyse comparée des pangénomes procaryotes : explorer la diversité génomique inter-espèces pour une meilleure  compréhension du métabolisme

\noindent \textbf{Mots clés :} Bioinformatique, Microbiologie environnementale, Pangénomique, Dynamique des génomes, Îlot génomique, Systèmes de défense aux phages
\vspace{-12pt}
\begin{multicols}{2}
\noindent \textbf{Résumé :} L’essor des projets de séquençage a généré plus d’un million de génomes procaryotes dans les bases publiques, nécessitant de nouvelles approches pour analyser cette masse de données. La suite logicielle PPanGGOLiN a été développée pour structurer ces informations sous forme de graphes de pangénome, permettant de compresser les données tout en conservant l’information de colocalisation des gènes. Elle intègre également des méthodes d’analyse de pangénome, panRGP, qui identifie les régions de plasticité génomique, et panModule, qui caractérise ces régions variables en sous-modules fonctionnels. Malgré ces avancées, aucune méthode ne permettait de comparer des pangénomes. Les travaux de cette thèse ont consisté à développer de nouvelles approches pour combler cette lacune. Tout d’abord, PPanGGOLiN a été enrichie par l’intégration de nouvelles méthodes, comme la recherche de contextes génomiques, et par une amélioration de son environnement logiciel. Ensuite, la méthode PANORAMA, qui se base sur les graphes de PPanGGOLiN,  a été conçue pour annoter des systèmes macromoléculaires, en combinant des critères de présence/absence de fonctions et de colocalisation génomique, et pour comparer des pangénomes. Appliqué aux systèmes de défense bactériens contre les phages, PANORAMA a permis d’identifier des systèmes et des sites d’insertions conservés entre différentes espèces. Finalement, un premier prototype de base  de données orientée graphe a été développé pour intégrer les données de plusieurs pangénomes afin d’exploiter au mieux leur information. Cette approche a permis d’analyser et de comparer des milliers de génomes bactériens et d’identifier des modules d’antibiorésistance communs à plusieurs espèces, mettant en lumière des mécanismes évolutifs partagés. Ces travaux ouvrent la voie à la pangénomique comparée, offrant un cadre inédit pour explorer le potentiel adaptatif des procaryotes et mieux comprendre leur dynamique évolutive. En facilitant la comparaison des pangénomes et l’identification de contextes génomiques conservés, ces développements contribuent à l’étude des interactions entre bactéries et à la caractérisation de systèmes biologiques d’intérêt.
\end{multicols}

\end{mdframed}
% page des résumés à garder en 2ème page. Si les résumés sont trop longs pour tenir sur une seule et même page, on peut mettre un résumé par page
% \thispagestyle{empty}
% \newgeometry{top=1.5cm, bottom=1.25cm, left=2cm, right=2cm}

\vspace{-12pt}
% \noindent 
% %*****************************************************
% %***** LOGO DE L'ED À CHANGER IMPÉRATIVEMENT *********
% %*****************************************************
% \includegraphics[height=2.45cm]{images/logos/logo_usp_SDSV.png}
% \vspace{1cm}
% %*****************************************************

\begin{mdframed}[linecolor=Prune,linewidth=1]

\textbf{Title:} Methods for comparative analysis of prokaryotic pangenomes: exploring interspecies genomic diversity for a better understanding of metabolism

\noindent \textbf{Keywords:} Bioinformatics, Environmental microbiology, Pangenomics, Genome dynamics, Genomic island, Phage defense systems
\vspace{-12pt}
\begin{multicols}{2}
\noindent \textbf{Abstract:} The boom in sequencing projects has generated over a million prokaryotic genomes in public databases, requiring new approaches to analyze this mass of data. The PPanGGOLiN software suite has been developed to structure this information in the form of pangenome graphs, enabling data compression while preserving gene colocalization information. It also integrates pangenome analysis methods: panRGP, which identifies regions of genomic plasticity, and panModule, which characterizes these variable regions into functional submodules. Despite these advances, there was no method for comparing pangenomes. The aim of this thesis was to develop new approaches to fill this gap. Firstly, PPanGGOLiN was enriched by integrating new methods, such as genomic context search, and by improving its software environment. Secondly, the PANORAMA method, based on PPanGGOLiN graphs, has been designed to annotate macromolecular systems, combining criteria of presence/absence of functions and genomic colocalization, and to compare pangenomes. Applied to bacterial phage defense systems, PANORAMA identified systems and insertion sites conserved between different species. Finally, a first prototype of a graph-oriented database was developed to integrate data from several pangenomes in order to make the most of their information. This approach has made it possible to analyze and compare thousands of bacterial genomes, and to identify antibiotic resistance modules common to several species, highlighting shared evolutionary mechanisms. This work paves the way for comparative pangenomics, offering a novel framework for exploring the adaptive potential of prokaryotes and better understanding their evolutionary dynamics. By facilitating the comparison of pangenomes and the identification of conserved genomic contexts, these developments contribute to the study of interactions between bacteria and the characterization of biological systems of interest.
\end{multicols}
\end{mdframed}