

% page des résumés à garder en 2ème page. Si les résumés sont trop longs pour tenir sur une seule et même page, on peut mettre un résumé par page
\thispagestyle{empty}
\newgeometry{top=1.5cm, bottom=1.25cm, left=2cm, right=2cm}


\noindent 
%*****************************************************
%***** LOGO DE L'ED À CHANGER IMPÉRATIVEMENT *********
%*****************************************************
\includegraphics[height=2.45cm]{images/logos/logo_usp_SDSV.png}
\vspace{1cm}
%*****************************************************

\small

\begin{mdframed}[linecolor=Prune,linewidth=1]

\textbf{Titre:} Méthodes d’analyse comparée des pangénomes procaryotes : explorer la diversité génomique inter-espèces pour une meilleure  compréhension du métabolisme

\noindent \textbf{Mots clés:} Bioinformatique, Microbiologie environnementale, Pangénomique, Dynamique des génomes, Îlot génomique, Systèmes de défense aux phages

\vspace{-.5cm}
\begin{multicols}{2}
\noindent \textbf{Résumé:} Ces dernières années ont vu l'explosion des projets de séquençage conduisant à un déluge de plusieurs centaines de milliers de génomes disponibles dans les banques publiques. Les approches de génomique comparée en microbiologie utilisent maintenant des milliers de génomes pour analyser la diversité d’une espèce. En effet, de nombreuses études se concentrent sur le contenu global en gènes d'une espèce (le pangénome) pour comprendre son évolution en termes de gènes communs et accessoires au regard de données épidémiologiques ou environnementales [1]. Néanmoins, le traitement de cette masse de données impose un changement de paradigme dans la représentation des connaissances et dans les algorithmes utilisés [2]. 

Dans cette optique, notre laboratoire travaille depuis plusieurs années sur une structuration des données génomiques sous la forme d’un graphe de pangénome, celle-ci permettant de compresser l’information de milliers de génomes tout en conservant l’organisation chromosomique des gènes. Nous avons ainsi développé des méthodes pour la reconstruction et l’analyse de pangénomes (méthode PPanGGOLiN) [3] et l’identification des régions de plasticité génomique (RPG; méthode panRGP) [4]. 

Le présent sujet de thèse a pour objectif de réaliser de nouveaux développements méthodologiques  pour l’étude comparée des pangénomes. Il s’agira de développer de nouvelles méthodes bioinformatiques pour des comparaisons inter-pangénomes qui s’appuieront notamment sur les développements réalisés pour l’identification et la caractérisation des RPG en sous-modules fonctionnels (méthode panModule). Les RPG regroupent à la fois des régions qui sont échangées entre les souches par transfert horizontal de gènes (comme par exemple les îlots génomiques) et des régions perdues différentiellement dans différentes lignées. Elles sont d'une importance primordiale pour comprendre le potentiel adaptatif des bactéries. L’exploration de ces modules fonctionnels au sein de différentes espèces permettra de mieux comprendre la dynamique évolutive à l’origine de la diversité métabolique des microorganismes.

Les algorithmes et outils développés au cours de ce projet seront mis en application afin d’étudier différents groupes bactériens d'intérêt médical, agronomique ou biotechnologique tels que les actinobactéries, les firmicutes ou les entérobactéries pour lesquelles de grandes quantités de données sont disponibles. Ces méthodes pourront être également appliquées à l’échelle d’un écosystème afin de comprendre la dynamique des génomes et les interactions entre différentes espèces vivant dans un même environnement. Une attention particulière sera donnée à l’analyse fonctionnelle des îlots génomiques au regard du métabolisme des organismes en termes de production de métabolites secondaires ou de voies cataboliques.

Ce travail bénéficiera des développements et des outils intégrés au sein de la plateforme MicroScope (mage.genoscope.cns.fr/microscope) [5] ainsi que de l’expertise dans notre unité de recherche sur le métabolisme microbien. Les outils développés dans le cadre de la thèse seront valorisés au sein la plateforme MicroScope et permettront également de répondre aux besoins d’analyses des partenaires académiques et industriels. Une des originalités de ce travail de thèse réside dans l’approche pangénomique pour la comparaison de génomes qui permet de répondre à un des challenges de la bioanalyse à l’ère du big data en biologie. 
\end{multicols}

\end{mdframed}

\vspace{8mm}

\begin{mdframed}[linecolor=Prune,linewidth=1]

\textbf{Title:} Methods for comparative analysis of prokaryotic pangenomes: exploring interspecies genomic diversity for a better understanding of metabolism

\noindent \textbf{Keywords:} Bioinformatics, Environmental microbiology, Pangenomics, Genome dynamics, Genomic island, Phage defense systems

\begin{multicols}{2}
\noindent \textbf{Abstract:} The last few years have seen the explosion of sequencing projects leading to a deluge of several hundred thousand genomes available in public databases. Comparative genomics approaches in microbiology now use thousands of genomes to analyze the diversity of a species. Indeed, many studies focus on the overall gene content of a species (the pangenome) to understand its evolution in terms of common and accessory genes with regard to epidemiological or environmental data [1]. Nevertheless, the processing of such mass of data imposes a paradigm shift in knowledge representation and in the algorithms used [2].

In this context, our laboratory has been working several years on a new model to represent genomic data in the form of a pangenome graph, which makes it possible to compress the information of thousands of genomes while preserving the chromosomal organization of genes. We have thus developed methods for the reconstruction and analysis of pangenomes (PPanGGOLiN method) [3] and the identification of regions of genomic plasticity (RGPs; panRGP method) [4].

The aim of this PhD thesis is to achieve new methodological developments for the comparative study of pangenomes. This will involve the development of  new bioinformatic methods for inter-pangenome comparisons, which will be particularly based on the developments carried out for the identification and characterization of RGPs in functional sub-modules (panModule method). RGPs include both regions which are exchanged between strains by horizontal gene transfer (such as genomic islands) and regions lost differentially among lineages. They are of paramount importance for understanding the adaptive potential of bacteria. The exploration of these functional modules in different species will provide a better understanding of the evolutionary dynamics behind the metabolic diversity of microorganisms.

The algorithms and tools developed during this project will be applied to study different bacterial groups of medical, agronomic or biotechnological interest such as actinobacteria, firmicutes or enterobacteria for which large amounts of data are available. These methods might also be applied at the scale of an ecosystem in order to understand the dynamics of genomes and the interactions between different species living in the same environment. Particular attention will be given to the functional analysis of genomic islands with regard to the metabolism of organisms in terms of production of secondary metabolites or catabolic pathways.

This work will benefit from the developments and tools integrated within the MicroScope platform (mage.genoscope.cns.fr/microscope) [5] as well as the expertise in our research unit on microbial metabolism. The tools developed in the context of the thesis will be promoted within the MicroScope platform to meet the analysis needs of academic and industrial partners. One of the originalities of this thesis work lies in the pangenomic approach for comparative genomics which addresses one of the challenges of bioanalysis in the era of big data in biology.
\end{multicols}
\end{mdframed}