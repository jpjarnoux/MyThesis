\part{Conclusion et perspectives}

\chapter{Conclusions sur le travail de thèse}

La pangénomique procaryote est un domaine en plein essor, bénéficiant de l’augmentation du nombre de génomes disponibles dans les bases de données. Bien que relativement récente dans l’histoire de la génomique, de la bioinformatique et, plus largement, de la microbiologie, l’analyse de pangénomes a déjà été largement adoptée comme outil de routine pour offrir une vision globale de la diversité génomique qu’elle permet d’explorer. Pour contribuer à ce domaine, l’objectif de ma thèse était de développer des méthodes permettant la comparaison des pangénomes en s’appuyant sur les graphes et les analyses de PPanGGOLiN, afin d’identifier des structures conservées entre différentes espèces.

Mon premier travail a ainsi consisté à concevoir une méthode de recherche de contextes génomiques dans le graphe de pangénome. Intégrée à PPanGGOLiN, cette méthode devait servir à la détection de contextes conservés à l’échelle des pangénomes. Afin de faciliter la comparaison des pangénomes, j’ai travaillé sur une nouvelle version de PPanGGOLiN, en collaboration avec Jean Mainguy, ingénieur de l’équipe, Alexandra Calteau, chercheuse au LABGeM, et David Vallenet, chef du laboratoire, ainsi qu’avec la contribution des anciens développeurs de PPanGGOLiN, Guillaume Gautreau (chercheur MaIAGE, INRAE) et Adelme Bazin (Ingénieur de recherche), également doctorants à l’époque. Cette version 2 de PPanGGOLiN apporte de nombreuses améliorations, notamment au niveau du stockage et de la lecture des données, mais aussi de nouveaux outils d'analyse, comme le \textit{clustering} des régions de plasticité génomique. Cette version contient aussi une refonte du code selon les standards Python, une documentation renforcée ainsi qu'une restructuration du cycle de vie du logiciel. Ce travail de fond a constitué une étape clé de ma thèse, garantissant une base robuste pour la suite des développements. L’ensemble de ces contributions a été récompensé par le prix "science ouverte du logiciel libre de la recherche", "espoir" de la catégorie ’Scientifique et technique, décerné par le ministère de l’Enseignement supérieur et de la Recherche en 2023.

En parallèle du maintien et de l’amélioration de PPanGGOLiN, j’ai développé PANORAMA, le premier outil dédié à la prédiction de systèmes biologiques dans les pangénomes et à la comparaison de pangénomes. La première étape a été de concevoir la méthode de prédiction et d’élaborer les modèles associés, un travail réalisé en 2022 au cours du stage de 2\ieme{} année de DUT de Laura Bry, que j’ai eu l’opportunité d’encadrer. Ces modèles offrent une représentation à la fois détaillée et flexible des systèmes biologiques, tout en restant facilement modifiables manuellement. Durant ce stage, nous avons également mis en place des méthodes de traduction des modèles de prédiction de systèmes de défense contre les phages, notamment PADLOC et DefenseFinder.

Parallèlement, j’ai poursuivi le développement de l’algorithme de prédiction de systèmes, qui a été affiné lors du stage de Quentin Fernandez De Grado en 2023, étudiant en 5\ieme{} année à l’INSA de Toulouse, que j’ai co-encadré. Au cours de ce stage, nous avons testé et évalué la méthode sur des systèmes de défense, en la confrontant aux références de la littérature scientifique. Nous avons aussi associé les systèmes identifiés aux RGPs et aux spots pour rechercher d’éventuels îlots de défense. Ce travail a donné lieu à plusieurs présentations dans des conférences, sous forme de posters et de présentations orales (voir annexes \ref{Ann:Communication}).

Dans le même temps, j’ai développé une approche dédiée à la comparaison des pangénomes, fondée sur l’analyse du contenu en familles de gènes au sein de structures déjà identifiées, telles que les RGPs, les spots, les modules et les systèmes. En calculant un score de similarité, cette méthode permet d’identifier des structures potentiellement conservées entre différentes espèces. En s’appuyant sur le pangénome, cette approche offre une comparaison globale du contenu génomique des espèces tout en intégrant l’ensemble de leur diversité.

Ces développements ont été intégrés à PANORAMA, un outil que j’ai conçu, tout en tirant parti de certains acquis de PPanGGOLiN. Il était donc nécessaire, en parallèle du développement des méthodes d’analyse, de concevoir un outil accessible à la communauté scientifique, facile à diffuser, bien documenté et conçu pour être maintenable sur le long terme.

Ce travail a donné lieu à un article qui sera prochainement soumis pour publication. L’article propose un benchmark comparatif entre PANORAMA et deux outils de référence pour la prédiction des systèmes de défense, DefenseFinder et PADLOC. Il inclut également une application sur le pangénome de \textit{Pseudomonas aeruginosa}, où les systèmes prédits sont associés aux spots, afin d’identifier des îlots et hotspots de défense. Concernant l’aspect comparatif, j’analyse la conservation des spots entre quatre espèces de la famille des Enterobacteriaceae et les associe aux systèmes de défense afin d’identifier d’éventuels îlots de défense conservés au sein de cette famille.

Au cours de ma thèse, j’ai également contribué au développement d’une base de données orientée graphe dédiée aux pangénomes. Ce projet a débuté lors du hackathon D4GEN 2022, où, aux côtés de Guillaume Gautreau, Lucas Gruda et Sullian Le Bozec (deux étudiants de Telecom SudParis) ainsi que Stefania Dumbrava (SAMOVAR, Institut Polytechnique de Paris, Télécom SudParis, ENSIIE), nous avons remporté la troisième place. Cette initiative a donné lieu à une collaboration avec Stefania Dumbrava et Angela Bonifati (LIRIS, Université Lyon 1), toutes deux spécialistes des bases de données orientées graphe. Dans ce cadre, j’ai développé un script permettant la construction et l’intégration des pangénomes dans une base de données de ce type. Nous avons ensuite défini plusieurs requêtes visant à analyser et comparer les pangénomes stockés. À titre d’application, nous avons intégré dix pangénomes d’ESKAPEE (un groupe d’espèces pathogènes résistantes aux antibiotiques) et effectué des recherches directes dans la base pour identifier des modules, calculés par PPanGGOLiN, similaires à ceux associés à des résistances. Ce projet a constitué une part importante de ma thèse, nécessitant l’acquisition de nouvelles compétences en bases de données orientées graphe et en langage de requête Cypher. Il a fait l’objet d’une publication et a été présenté lors du workshop SEAGRAPH de la conférence ICDE 2024, où il a suscité un vif intérêt au sein de la communauté informatique, ce type d’application et de données restant encore peu exploré.

Au cours de ma thèse, j’ai également eu l’opportunité de contribuer, de manière plus ponctuelle, à d’autres projets en lien avec la pangénomique. Ces collaborations m’ont permis d’échanger avec des chercheurs issus de disciplines variées, enrichissant ainsi ma compréhension des approches interdisciplinaires. Ces interactions ont également été l’occasion de développer ma capacité à adapter mon discours en fonction de mon public et de son niveau de connaissance.

Dans cette même perspective, j’ai eu la chance d’enseigner aux étudiants de première année du master GENIOMHE (Université Evry Val d'Essonne - Paris Saclay), dans le cadre d’un cours sur la génomique comparée et la présentation de la plateforme d’annotation des génomes procaryotes MicroScope. Cette expérience m’a permis d’affiner ma capacité à transmettre des concepts complexes de manière claire et accessible.

\chapter{Perspectives sur les méthodes développées}

\section{Critique et amélioration possible des méthodes}

\subsection{PPanGGOLiN et recherche de contexte génomique}

La méthode développée pour la recherche du contexte génomique permet d’identifier les familles conservées entourant un ensemble de familles d’intérêt. Nos expérimentations montrent qu’elle est globalement efficace, mais qu’elle présente une sensibilité à la taille du contexte recherché ainsi qu’au paramètre de transitivité. Cette sensibilité semble être inhérente à l’algorithme de construction, qui repose sur un parcours exhaustif des génomes pour reconstituer le contexte.
L’algorithme, après l’identification des familles cibles, procède à la construction du graphe de contexte et nécessite pour cela un retour aux génomes, une opération particulièrement coûteuse en termes de temps de calcul et de ressources. Ce choix méthodologique est lié à l’architecture du graphe de pangénome, qui encode uniquement le voisinage direct des gènes dans les génomes. Par conséquent, l’établissement des relations de transitivité exige un retour au niveau génomique.
Un autre facteur limitant réside dans la nature multigénique de certaines familles du pangénome : plusieurs gènes appartenant à une même famille peuvent être présents au sein d’un même génome. Or, cette information n’est pas directement conservée dans le graphe, ce qui empêche de déterminer les relations de voisinage sans une analyse approfondie des génomes.

Ainsi, bien que la méthode actuelle soit bien adaptée à la recherche de contextes clairement définis et à des paramètres de transitivité modérés, son extension à la détection de systèmes complexes, notamment dans le cadre de PANORAMA, soulève déjà des défis méthodologiques. Des améliorations techniques sont envisageables, notamment la parallélisation du parcours du pangénome afin d’accélérer l’exécution de l’algorithme. Une autre approche consisterait à revoir l’étiquetage du pangénome pour y intégrer directement les informations nécessaires à la reconstruction du contexte, évitant ainsi un retour systématique aux données génomiques.

\subsection{PANORAMA et prédiction des systèmes}

La prédiction des systèmes biologiques dans PANORAMA peut être divisée en 3 étapes : annotation fonctionnelle des familles de gènes, recherche de contextes génomiques et identification des systèmes dans le contexte.

L'étape d'annotation consiste à aligner une base de données HMM contre les séquences des familles de gènes. Cette étape est efficace et repose sur un outil externe : pyHMMER, laissant peu de place à des améliorations dans PANORAMA. Toutefois, on peut noter que les autres outils utilisent des seuils, comme la e-value, pour filtrer les résultats d'alignement. En utilisant le pangénome, le nombre de séquences par rapport à un génome peut fortement varier, ce qui modifie la e-value. Nous avons pu expérimenter dans notre benchmark des différences dans l'annotation entre les gènes et les familles de gènes à cause de tels seuils. Une amélioration possible serait de remplacer ces seuils par un critère d'alignement qui ne dépendent pas de la taille de la base de données.

Concernant la recherche de contextes génomiques et l'identification de systèmes, nous en avons déjà discuté précédemment dans le cadre de PPanGGOLiN. Toutefois, il convient d’ajouter que l’algorithme a été conçu pour être exhaustif, retournant ainsi tous les systèmes possibles. Pour améliorer son efficacité, nous pourrions envisager une réécriture de l’algorithme en intégrant des méthodes heuristiques. J’avais d’ailleurs commencé à développer des fonctions basées sur des algorithmes gloutons et hongrois, afin d’optimiser l’exécution de certaines parties du code jugées relativement lentes. En particulier, pour rechercher les systèmes rares dans les génomes, nous nous appuyons sur une approche combinatoire permettant d’identifier les systèmes potentiellement existants. Toutefois, lorsqu’elles ont été appliquées aux bases de données de systèmes de défense aux phages, ces optimisations n’ont pas significativement réduit les temps de calcul. De plus, une partie des systèmes n’était plus correctement prédite, suggérant un compromis entre rapidité et exhaustivité à explorer davantage.

Un autre point qui mériterait d’être exploré concerne la multiplication des résultats pour un même ensemble de familles. Il peut arriver que certains modèles soient proches en termes de composition et de paramètres, comme c’est le cas des systèmes de restriction-modification ou des systèmes CBASS. Avec PANORAMA, il est possible de prédire, pour un même ensemble de familles de gènes, plusieurs systèmes appartenant à la même catégorie ou non. Dans ce cas, les fichiers de sortie contiennent l’ensemble des systèmes détectés.
Dans la version 2 de MacSyFinder, un calcul de score a été introduit pour pallier ce problème. Ce score repose sur la composition du système dans le génome \cite{neron_macsyfinder_2023}. Ainsi, dans MacSyFinder (sauf indication contraire), un gène ne peut appartenir qu’à un seul système, et celui ayant le meilleur score est sélectionné.
À l’échelle du pangénome, un tel score pourrait toutefois s’avérer trop restrictif dans PANORAMA. Néanmoins, l’intégration d’un score fournirait une indication supplémentaire et permettrait, entre deux systèmes de la même catégorie, de filtrer les résultats et de limiter le nombre de prédictions.
Si un tel score devait être mis en place, il devrait reposer sur la composition en familles du système (comme dans MacSyFinder), mais aussi prendre en compte la partition des familles. De cette manière, en ajustant des bonus et pénalités liés aux partitions, il serait possible de favoriser la prédiction des systèmes présents dans les parties variables ou conservées du pangénome.

\section{Perspectives et projets autour de la pangénomique}

Lors de ma thèse, j'ai pu participer et discuter de plusieurs projets autour de la pangénomique.

Le premier projet auquel j’ai participé est PanGBanK. Son objectif est de constituer une base de données regroupant, pour chaque espèce, un pangénome construit avec PPanGGOLiN. De plus, cette base de données sera accessible en ligne, et des métriques ainsi que des analyses pangénomiques seront disponibles pour chaque pangénome.
Une telle base constitue aujourd’hui un atout majeur pour l’ensemble des développements réalisés au cours de cette thèse, en particulier pour PANORAMA et la comparaison des pangénomes.

Parmi les autres projets auxquels j’ai pu contribuer, BlueRemediomics s’intéresse, entre autres, à la caractérisation des enzymes et des voies de biodégradation des filtres UV, ainsi qu’à la biosynthèse des exopolysaccharides (EPS) dans les bactéries marines. Mon implication a porté sur les EPS, en participant aux discussions sur l’utilisation des modèles et la recherche de systèmes dans les pangénomes.
Au LABGeM, Jean Mainguy travaille sur la définition de ces modèles. Une fois finalisés, ils permettront notamment de détecter les voies de biosynthèse des EPS dans les pangénomes, mais aussi de réaliser des analyses comparatives sur les systèmes identifiés.

\chapter{Perspectives sur la pangénomique et la génomique comparée}

Au cours de ma thèse, j’ai développé et éprouvé des méthodes d’analyse appliquées à des jeux de données de grande envergure, en m’appuyant sur les relations phylogénétiques entre les génomes afin de construire des ensembles de données au niveau de l’espèce. L’objectif principal de cette approche est la constitution de familles de gènes présentant une similarité suffisante pour être considérées comme homologues, garantissant ainsi une certaine homogénéité fonctionnelle au sein de ces familles. Ce postulat est largement utilisé dans la construction des pangénomes, qu’ils soient basés sur des séquences génomiques ou sur des familles de gènes homologues. L'essor des études pangénomiques a permis d’étendre considérablement notre compréhension de la diversité intra-espèce, et une majorité des travaux actuels se concentrent sur l’analyse du pangénome à l’échelle d’une espèce.


Bien que le concept de pangénome ait été initialement appliqué à l’étude d’une espèce donnée, son intérêt réside dans la prise en compte et l’analyse de la diversité génomique à une échelle plus large. Cependant, peu d’études se sont jusqu’ici orientées vers une analyse pangénomique au niveau du genre, une approche pourtant prometteuse pour mieux comprendre l’évolution des génomes.
Des travaux récents, tels que ceux de Jonkheer et al. \cite{jonkheer_pectobacterium_2021}, ont exploré cette voie en construisant et en analysant le pangénome du genre \textit{Pectobacterium} à partir de 197 génomes répartis en 19 espèces, en utilisant l’outil PanTools \cite{sheikhizadeh_pantools_2016}. PanTools présente l’avantage d’estimer de manière optimisée les paramètres de construction des familles de gènes homologues, permettant ainsi une analyse pangénomique plus robuste à l’échelle du genre. L’extension de ces approches constituerait une avancée majeure dans notre compréhension des dynamiques évolutives et fonctionnelles des génomes bactériens.

Un autre domaine prometteur concerne l’étude du pangénome de communautés microbiennes et son application en métagénomique. L’analyse pangénomique dans un contexte métagénomique permettrait d’obtenir une vision plus intégrative des interactions entre les espèces d’un même environnement, d’identifier des fonctions partagées et d’améliorer notre compréhension des dynamiques écologiques des génomes présents. En particulier, cette approche pourrait révéler des adaptations fonctionnelles clés ainsi que des éléments de co-évolution entre les espèces d’une même niche écologique. Par ailleurs, la métapangénomique ouvre des perspectives pour une meilleure caractérisation des microbiotes complexes, notamment en santé humaine, en agronomie et en écologie environnementale \cite{delmont_linking_2018}).

Les progrès et la démocratisation des méthodes d’apprentissage automatique pourraient largement contribuer à l’étude des pangénomes. Les algorithmes de \textit{machine learning} permettent d’extraire des motifs complexes, de prédire des fonctions géniques et d’identifier des relations inédites entre génomes au sein d’un pangénome \cite{kavvas_machine_2018}. Par exemple, les modèles d’apprentissage profond peuvent être exploités pour classifier les gènes en fonction de leurs rôles biologiques, tandis que les approches de clustering non supervisé permettent de révéler des familles de gènes selon des critères encore inexplorés. Ces avancées méthodologiques ouvrent ainsi la voie à une compréhension plus fine de l’organisation et de la fonction des génomes microbiens.

\newpage

La pangénomique demeure un domaine de recherche relativement récent, offrant encore de nombreuses perspectives d’amélioration pour affiner notre compréhension de la diversité et de l’évolution des génomes. Nous assistons aujourd’hui à une nouvelle étape dans le développement des méthodes pangénomiques.
La première phase a été marquée par la définition et la construction des pangénomes, avec des outils comme PPanGGOLiN. La seconde s’est concentrée sur leur analyse et leur exploration, illustrée par des approches comme panRGP et panModule. Désormais, nous entrons dans une troisième phase, caractérisée par l’émergence de nouvelles méthodologies : la construction de pangénomes à des rangs taxonomiques plus élevés, l’essor de la métapangénomique, l’intégration de l’intelligence artificielle et, enfin, le développement de nouvelles approches comparatives, comme PANORAMA.
Ces avancées méthodologiques permettront une meilleure appréhension de la dynamique évolutive des génomes et de leur rôle dans l’adaptation métabolique des organismes.
