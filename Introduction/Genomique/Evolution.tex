\newpage
\section{Dynamique évolutive des génomes}
\label{sec:dyn_evo}

La dynamique évolutive des procaryotes est caractérisée par des processus continus de gain, perte et modification de gènes (\autoref{fig:dyna_evo}). La taille des génomes étant restreinte, la perte de gènes peut optimiser le génome en éliminant les séquences redondantes ou non essentielles, favorisant ainsi une efficacité accrue dans des environnements spécifiques. Les modifications génétiques, quant à elles, jouent un rôle crucial dans l'adaptation fine des procaryotes face aux pressions sélectives variées. L’acquisition de nouveaux gènes introduit une diversité génétique, conférant des traits avantageux, tels que la résistance aux antibiotiques ou la capacité à métaboliser de nouvelles sources de nutriments.

\begin{figure}[htbp]
    \centering
    \includegraphics[width=\linewidth]{images/DynamiqueEvo.png}
    \caption[Schéma de la dynamique évolutive des génomes procaryotes]{\textbf{Schéma résumant la dynamique évolutive des génomes procaryotes.} Source LABGeM}
    \label{fig:dyna_evo}
\end{figure}

 Les gènes doivent ensuite être transmis dans la population. Les \textbf{transferts verticaux} permettent de transférer les gènes de génération en génération, assurent la continuité et la stabilité des traits essentiels. Les \textbf{transferts horizontaux} permettent l'échange de gènes entre les organismes, favorisant une diversification rapide des génomes, qui peut radicalement transformer les capacités adaptatives des lignées procaryotes. Cette dynamique complexe façonne la biodiversité procaryote et témoigne de la capacité évolutive exceptionnelle de ces organismes à coloniser une multitude d'écosystèmes.

\subsection{Mécanismes d'évolution par transfert vertical}
\label{sec:evo_ver}
Les mécanismes d'évolution par héritage, regroupent les processus menant à une modification du génome entre la cellule mère et la cellule fille. Théoriquement, lors de la division cellulaire, la cellule mère se divise en 2 cellules filles possédant exactement la même information génétique qu'elle. Pourtant, malgré un ensemble de mécanismes de protection et de correction de l'ADN, le génome peut différer entre les cellules mère et fille. Ce sont ces "erreurs" qui vont nous intéresser, car ce sont elles qui sont à l'origine de l'innovation et de la diversité génétique.

\newpage

\subsubsection{Impact des mutations génétiques : SNPs, Indels et pseudogènes}
\paragraph{\textit{Single Nucleotid Polymorphism}}

Un \textit{Single Nucleotide Polymorphism} (SNP) correspond à une modification de la séquence induite par la mutation d'un nucléotide en un autre.
Étant donné que le code génétique est dégénéré\footnote{Un acide aminé peut être codé par plusieurs codons différents.}, la mutation peut ne pas avoir d'impact sur la séquence de la protéine, on dit alors que la mutation est silencieuse ou même sens. Si la modification entraîne un changement d’acide aminé dans la séquence protéique, on parle de mutation faux-sens. Enfin, une mutation est qualifiée de non-sens lorsqu'elle introduit prématurément un codon STOP, interrompant ainsi la traduction et conduisant à une perte de fonction de la protéine. Une telle mutation peut également affecter un site fonctionnel clé (comme un site actif), compromettant l’activité de la protéine. Lorsque l’introduction d’un codon STOP précoce rend un gène non fonctionnel, ce dernier devient un \textbf{pseudogène}, un vestige génomique dépourvu de rôle biologique actif, un phénomène appelé pseudogénisation.

Sur la \autoref{fig:mec_evo}, la première mutation implique un changement de glutamine en histidine, des acides aminés aux propriétés de polarité et de charge différentes. Il s’agit donc d’une mutation faux-sens, qui aura probablement un impact significatif sur la structure de la protéine. En revanche, les deux autres SNPs ne modifient pas l’acide aminé codé, ils sont donc silencieux.

\begin{figure}[htbp]
    \centering
    \includegraphics[width=.65\textwidth]{images/Mec_evo.jpg}
    \caption[Identification des SNP et indels entre 2 génomes]{\textbf{SNP et InDels entre deux génomes.} On suppose que le premier codon commence par le premier nucléotide. Figure extraite et adaptée de \cite{qi_detection_2014}}
    \label{fig:mec_evo}
\end{figure}

\paragraph{Indels: insertion, délétion et pseudogènes}

Un indel correspond à l'insertion (In) ou la délétion (del)\footnote{On regroupe l'insertion et la délétion, car sans une analyse phylogénétique, il est impossible de les différencier par comparaison de séquence.} d'un ou plusieurs nucléotides dans la séquence d'un gène. Lorsque la taille de l'indel est un multiple de 3 (insertion ou délétion d'un codon), la séquence protéique peut soit être allongée, soit raccourcie d'un acide aminé, soit coupée de façon précoce si le codon est un codon STOP.

Si la taille de l'indel n'est pas un multiple de 3, il y aura un décalage du cadre de lecture ou \textit{frameshift}. Ce décalage va induire un changement de tous les acides aminés de l'indel à la fin du gène, provoquant avec lui un changement dans la fonction de la protéine ou une inactivation de la fonction. La partie du gène qui n'est pas décalée est alors considérée comme un fragment du gène initial, il est alors qualifié de pseudogène. À nouveau, cette mutation peut être délétère pour la cellule. Sur la \autoref{fig:mec_evo}, les indels sont de taille 1 et 2, elles ne provoquent pas l'apparition d'un codon STOP précoce, mais l'ensemble des acides aminés est modifié.

Les indels vont donc transformer la séquence protéique traduite, pouvant nuire à la fonction de cette dernière et être délétère pour l'organisme. Pour éviter les problèmes liés aux \textit{frameshifts}, il a été montré qu'il existe un fort taux de codon STOP hors du cadre de lecture \cite{tse_natural_2010}. Cette adaptation permettrait de limiter la traduction des protéines mutantes et d'ainsi limiter le coût énergétique pour la cellule. Il a aussi été montré que les \textit{frameshifts} pourraient être à l'origine d'un réservoir d'adaptation à l'environnement \cite{koch_catastrophe_2004}. Lors d'un changement dans l'environnement créant une nouvelle pression de sélection, un \textit{frameshift} pourrait produire une protéine qui permet à l'organisme de s'adapter à son environnement et donc d'améliorer sa \textit{fitness}\footnote{Le \textit{fitness} correspond à la capacité d'un individu de survivre dans son environnement et à se reproduire}. Une fois que l'élément perturbateur de l'environnement disparaît, un nouveau \textit{frameshift} pourrait ramener le cadre de lecture à sa place d'origine. Ce mécanisme, en accord avec la petite taille des génomes, aurait l'intérêt de ne pas perdre des gènes d'adaptation à l'environnement, même s'ils ne sont nécessaires que ponctuellement.

\subsubsection{Réarrangement génomique : un moteur de l'évolution}
\label{sec:rearragement}

Les génomes évoluent également suite à des événements de réarrangement. Ils impliquent des segments d'ADN plus importants. La forme du génome obtenue, appelée variant structural (SV pour \textit{Structural variant} en anglais), est plus difficile à détecter que les SNP et les indels \cite{periwal_insights_2015}.

Le mécanisme de recombinaison est à l'origine des réarrangements. Une recombinaison implique l'échange de 2 portions d'ADN entre 2 molécules ou 2 régions d'ADN. La recombinaison peut être homologue, se produisant entre des séquences similaires, ou non-homologue, impliquant des séquences différentes. Elle est souvent médiée par des enzymes spécialisées comme RecA ou des intégrases, qui permettent l'intégration, la réparation ou le réarrangement précis des séquences. La recombinaison homologue est cruciale pour la réparation des cassures de l'ADN, les réarrangements et également dans l'acquisition de nouveaux gènes par transfert horizontal (cf. \autoref{sec:evo_hz}) \cite{eisenstark_genetic_1977}.

Les réarrangements de l'ADN correspondent donc à un échange entre 2 segments du génome, induisant une insertion, une délétion ou une modification de l'ordre des nucléotides (\autoref{fig:rearrangement}). Les réarrangements sont fréquents dans les génomes procaryotes \cite{sun_genome-wide_2012} et peuvent être spontanés ou facilités par la présence d'éléments mobiles, tels que les transposons, qui sont des séquences d'ADN capables de se déplacer au sein du génome. Ils sont composés de gènes codant pour une transposase, l'enzyme responsable de son déplacement, ainsi que de séquences répétées aux extrémités, nécessaires à la reconnaissance et à l'excision du transposon. 

L'ordre des gènes étant important dans l'expression des gènes et la fonction des protéines, le SV résultant peut conduire à une modification de l'expression génique ou à un changement dans la fonction de la protéine. Il existe 3 formes de réarrangement : symétrique, asymétrique et au sein d'un réplicon. Ces formes ne sont pas toutes équiprobables, car elles affectent plus ou moins la structure du génome. Aussi, les réarrangements proches de l'Ori sont plus fréquents que ceux proches du site de terminaison \cite{darling_dynamics_2008}. 

\begin{figure}[htbp]
    \centering
    \subfloat{\includegraphics[width=0.5\textwidth,keepaspectratio]{images/rearrangement1.png}}
    \subfloat{\includegraphics[width=0.4\textwidth,keepaspectratio]{images/rearrangement2.png}}
    \caption[Réarrangement et implication]{\textbf{Réarrangement et conséquences des variants structuraux.} (A) Région génomique sans SV. Les rectangles représentent les gènes et les petits connecteurs à côté représentent le promoteur du gène concerné. (B) Réarrangement intragénique illustrant la délétion et la fusion de gènes à la suite d'une duplication partielle du gène. Les régions codantes modifiées produisent des transcrits aberrants. La délétion ou la duplication peut entraîner une modification du nombre des gènes dans des régions par ailleurs fonctionnellement intactes. (C) Délétion du promoteur, la régulation est modifiée et une duplication/délétion qui modifie le nombre de copies des gènes. (D) Inversions affectant la structure du gène, le gène est inversé, retourné et réarrangé, ce qui éloigne l'un des promoteurs du premier gène (orange). (E) Translocations affectant le contexte génique. Figure extraite et adaptée de \cite{periwal_insights_2015}}
    \label{fig:rearrangement}
\end{figure}

Les recombinaisons peuvent également conduire à la duplication de gènes ou de régions génomiques, un mécanisme clé dans l’évolution des procaryotes en générant une redondance génétique. Cette redondance offre une opportunité évolutive : tandis qu’une copie du gène conserve sa fonction initiale, l’autre peut accumuler des mutations, potentiellement aboutissant à une nouvelle fonction, sans compromettre la survie de l’organisme. En outre, la duplication peut jouer un rôle dans la régulation de l’expression génique. Par exemple, les gènes codant pour les pompes à efflux, impliquées dans l’évacuation des antibiotiques hors de la cellule, sont fréquemment dupliqués, favorisant ainsi une meilleure résistance aux traitements \cite{maddamsetti_duplicated_2024}. Toutefois, les événements de duplication restent moins fréquents que les transferts horizontaux de gènes dans les génomes procaryotes \cite{tria_gene_2021}. Cette rareté s’explique en partie par les mécanismes d’élimination de la redondance, qui optimisent la compacité et l’efficacité des génomes bactériens.

Les mécanismes qui viennent d'être décrits apportent de l'innovation dans les génomes procaryotes, qui doit ensuite être transmise dans la population. Avec le transfert vertical, cette transmission se fait uniquement d'une génération à l'autre, un processus limité par le temps de génération, qui varie selon l'espèce (\textit{E. coli} : 20 min, \textit{Lactobacillus acidophilus} : 80 min, \textit{Mycobacterium tuberculosis} : 800 min). Un temps de génération plus long semble aussi réduire le taux de mutation spontanée de l'ADN \cite{weller_generation-time_2015}. Pour contourner ces contraintes, les procaryotes échangent de l’ADN avec leur environnement (autres bactéries, virus, eucaryotes, ADN libre\dots), par un ensemble de processus regroupé sous le terme de \textbf{transfert horizontal}, qui leur permet d’acquérir de nouvelles fonctions génétiques.

\subsection{Mécanismes d'évolution par transfert horizontal}
\label{sec:evo_hz}

Les transferts horizontaux de gènes (\textit{Horizontal Gene Transfert} en anglais, HGT) constituent un phénomène central dans l'évolution des procaryotes, permettant l'échange de matériel génétique entre organismes sans nécessiter une relation de lignage directe. La proportion de gènes acquis par transfert horizontal varie considérablement selon les espèces et les environnements, mais elle peut représenter une part significative du génome procaryote. On estime que 20 \% des gènes en moyenne ont été acquis par HGT, certaines études montent même jusqu'à 25 \% pour certaines bactéries \cite{ochman_lateral_2000,popa_directed_2011}. Cette proportion élevée témoigne de l'importance des HGT dans l'évolution et l'adaptation des procaryotes.

Les gènes sont transférés via des éléments génétiques mobiles (MGE), incluant les plasmides, les transposons et les phages (virus de bactérie, cf. \autoref{sec:phage}), chacun possédant des capacités uniques pour mobiliser les gènes. Ces vecteurs facilitent le transfert et l'intégration de l'ADN étranger dans le génome hôte. Les séquences répétées, telles que les insertions et les répétitions en tandem, jouent également un rôle, en servant de sites d'intégration pour les MGE. 

Il existe 3 grands mécanismes de HGT : la \textbf{transformation}, la \textbf{conjugaison} et la \textbf{transduction}, chacun facilitant le mouvement de gènes entre cellules de manière distincte.

\subsubsection{Conjugaison : la sexualité des procaryotes}

La conjugaison a été découverte en 1946 par Joshua Lederberg et Edward L. Tatum \cite{lederberg_sex_1953}, qui décrivent ce mécanisme comme la manière sexuée des bactéries d'échanger de l'ADN. En effet, par analogie, la conjugaison demande un contact direct entre une cellule donneuse et une cellule receveuse pour l'échange de matériel génétique\footnote{N.B : Le transfert est unidirectionnel, la cellule donneuse ne peut recevoir de l'ADN et la receveuse ne peut en donner.}. Il existe 2 catégories d'éléments génétiques mobiles conjugatifs : les plasmides et les éléments intégratifs et conjugatifs (ICEs, \textit{Intergrative and Conjugative elements} en anglais). Sur la \autoref{fig:conjugaison} est représenté l'échange d'un plasmide par conjugaison. Les ICEs \cite{johnson_integrative_2015}, contrairement aux plasmides, sont directement intégrés au chromosome, ce qui rend leur réplication dépendante  de celui-ci. Toutefois cette intégration favorise un transfert vertical plus stable au cours des générations. Les ICEs pour être échangés doivent suivre un schéma circulaire : excision du chromosome, circularisation, réplication, transfert et réintégration dans le chromosome. Lors de l'étape d'excision, il peut arriver que des gènes flanquant l'ICEs soient excisés aussi, apportant une nouvelle forme à l'ICE \cite{gibbons_genomic_2011}.

\begin{figure}[htbp]
    \centering
    \includegraphics[width=0.7\linewidth]{images/Conjugation.png}
    \caption[Schéma du fonctionnement de la conjugaison]{\textbf{Schéma du fonctionnement de la conjugaison, dans le cas d'un plasmide conjugatif.} (1) Formation d'un pili sexuel par la bactérie donneuse. (2) Contact direct entre les 2 bactéries via le pili. (3) Réplication de l'ADN plasmidique et transfert à la bactérie donneuse. (4) Terminaison de la conjugaison et nouvelle formation d'un pili pour la receveuse devenue donneuse. Image sous licence Creative Commons 3.0 \url{https://commons.wikimedia.org/wiki/File:Conjugation.svg}}
    \label{fig:conjugaison}
\end{figure}

Plasmides et ICEs sont généralement de petite taille, mais ils contiennent des gènes clés d'adaptation à l'environnement. La présence de ces gènes dans les éléments mobiles permet à des colonies de répondre efficacement et rapidement aux nouvelles conditions environnementales, comme la présence de métaux lourds ou d'antibiotiques \cite{botelho_role_2021}. Toutefois, tous les MGEs ne sont pas forcément conjugatifs \cite{valentine_mobilization_1988}, ils vont profiter de la conjugaison codée par un autre élément pour se transférer. Dans ces conditions, la bactérie receveuse ne devient pas conjugative à son tour, même si elle reçoit l'élément mobile. Ces éléments mobilisables sont appelés des IMEs (élément intégratif mobilisable). Il est d'ailleurs à noter que tous les plasmides ne sont pas mobilisables, il y aurait d'ailleurs autant de plasmides conjugatifs que de plasmides non mobilisables \cite{smillie_mobility_2010}.

La conjugaison est un mécanisme majeur de transfert horizontal de matériel génétique, qui a la caractéristique de rapidement répandre les éléments mobiles. Il a toutefois le défaut de limiter le transfert de gènes entre cellules procaryotes et donc de limiter le transfert aux innovations génétiques déjà intégrées par un autre organisme procaryote. De plus, tous les organismes ne sont pas capables de réaliser la conjugaison, ce qui réduit d'autant plus la capacité de transfert au niveau des communautés. 
%Ce n'est pas le cas des prochains mécanismes que nous décrirons.   
\newpage
\subsubsection{Transformation : recycler l'ADN environnant}

La transformation correspond à l'intégration d'un fragment d'ADN étranger dans le génome de l'organisme. Les bactéries pouvant réaliser la transformation sont dites compétentes. Ce qui  différencie la transformation de la conjugaison, c'est que l'ADN intégré est libre dans l'environnement
\footnote{La découverte de la transformation en 1928 par Fred Griffith \cite{griffith_significance_1928}, précède de nombreuses années celle qui a mis en évidence que l'ADN est le porteur de l'information génétique \cite{avery_studies_1944}. La transformation est donc une preuve anticipée et un socle pour démontrer le rôle de l'ADN.}. De plus, la transformation est la seule forme de HGT, totalement contrôlée par la cellule receveuse \cite{huang_activation_2021}. Assez peu d'espèces sont connues pour être capables de réaliser la transformation de manière naturelle, toutefois un nombre plus important contient  la machinerie nécessaire à sa réalisation \cite{johnston_bacterial_2014}. De plus, au sein d'une espèce, le taux d'individu compétent peu varier, par exemple chez \textit{S. pneumoniae}, 66 \% des individus sont capables de la réaliser \cite{evans_significant_2013}.
Pour terminer, les mécanismes de la transformation, notamment l’incorporation de l’ADN dans la cellule (\autoref{fig:transformation}), sont bien décrits dans la littérature \cite{johnston_bacterial_2014,dubnau_mechanisms_2019}. Toutefois, ils varient d’une espèce procaryote à l’autre, tout comme la proportion d’individus capables de réaliser cette transformation \cite{stewart_biology_1986}. Nous ne reviendrons donc pas sur les mécanismes, mais seulement sur des exemples d'application.

\begin{figure}[htbp]
    \centering
    \includegraphics[width=0.625\linewidth]{images/transformation.png}
    \caption[Schéma du mécanisme de transformation]{\textbf{Schéma du mécanisme de transformation.} Extrait de \cite{johnston_bacterial_2014}}
    \label{fig:transformation}
\end{figure}

Les bactéries du genre \textit{Nesseria} et particulièrement \textit{N. gonorrhoeae}\footnote{Ce genre bactérien, vivant dans les muqueuses des mammifères, est non pathogène à l'exception de \textit{N. meningitidis}, impliqué dans la méningite et \textit{N. gonorrhoeae}, responsable de la gonorrhée, une infection sexuellement transmissible.} reconnaissent préférentiellement une séquence d'ADN non palindromique de leur propre ADN \cite{goodman_identification_1988,duffin_dna_2010}. Ce système permet d'intégrer uniquement l'ADN de souches proches, ainsi que des gènes d'adaptation, comme des gènes de résistance aux antibiotiques \cite{centers_for_disease_control_and_prevention_cdc_update_2007}. Ainsi, les gènes d'adaptation d'intérêt sont préférentiellement distribués dans l'espèce.

\textit{Streptococcus pneumoniae}\footnote{Bactérie connue pour son rôle d'agent pathogène dans les pneumonies et responsable de co-infection pendant la grippe espagnole} utilise la transformation comme mécanisme de réparation de l'ADN, car cette espèce ne possède pas de système de réparation SOS \cite{gasc_lack_1980}. Les souches de \textit{S. pneumoniae} s'engagent alors dans une "guerre fratricide" pour récupérer l'ADN des autres souches de leur espèce \cite{claverys_cannibalism_2007}.

Pour terminer, chez \textit{Bacillus subtilis}\footnote{Bactérie du sol, mais qu'on retrouve dans de nombreux habitat dû à ses capacités d'adaptation. Elle est utilisée comme modèle d'étude des bactéries Gram+.}, la transformation entre individus de la même espèce, mais de souche éloignée, est privilégiée \cite{lyons_combinatorial_2016}. Les bactéries vont sécréter dans l'environnement des antibiotiques, auxquels elles sont résistantes, pour tuer les autres individus de l'espèce. L'ADN récupéré est donc différent de celui de la bactérie et donc potentiellement source de nouvelles fonctions.

Ces exemples montrent aussi une opposition dans la philosophie des mécanismes de conjugaison et de transformation. La transformation demande que l'ADN soit libre dans l'environnement et donc que les bactéries environnantes soient détruites, alors que la conjugaison laisse les 2 cellules en vie. 

\subsubsection{Transduction : les virus mis à profit}

La transduction est un mécanisme reposant sur l'intervention d'un virus pour transporter et transférer le matériel génétique d'une cellule procaryote à l'autre (\autoref{fig:transduction}). Les virus de bactéries, surnommés (bacterio)phages, vont infecter la cellule donneuse pour répliquer leur ADN. Lors de la réplication, de l'ADN de la cellule donneuse peut se trouver intégré à celui du phage. Lorsqu'il infectera une cellule receveuse, la portion d'ADN de la donneuse pourra reprendre une forme plasmidique (si c'est un plasmide qui a été transféré) ou être intégrée au génome de la cellule par recombinaison homologue. La transduction est aujourd'hui largement utilisée en génétique et microbiologie pour transférer de l'ADN et modifier les génomes \cite{wang_phage-based_2024}. 

\begin{figure}[htbp]
    \centering
    \includegraphics[width=0.65\linewidth]{images/transduction.png}
    \caption[Schéma synthétique de la transduction]{\textbf{Schéma représentant les étapes de transduction.} Extrait de \cite{chiang_genetic_2019}}
    \label{fig:transduction}
\end{figure}

La première forme de transduction identifiée décrivait le transfert de n'importe quel gène de la donneuse à la receveuse par le phage. Cette forme a donc été nommée transduction généralisée \cite{zinder_genetic_1952}. Une seconde forme dite spécifique a été découverte en étudiant le phage $\lambda$ infectant les \textit{E. coli} \cite{morse_transduction_1956}. Le transfert se limite à un ensemble de gènes définis. Enfin, une dernière forme, la transduction latérale, a récemment été découverte \cite{chen_genome_2018}. Là où les formes générale et spécifique peuvent être vues comme une erreur et un événement lié au hasard, la transduction latérale fait partie du cycle de vie du phage, menant à un taux de transfert beaucoup plus important. 