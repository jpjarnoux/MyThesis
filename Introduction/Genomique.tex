\chapter{Génomique des procaryotes : organisation, évolution et fonctions}

%Les génomes procaryotes sont souvent décrits comme plus simple et plus facile à étudier que les génomes eucaryotes. Pourtant, sous cette simplicité apparente, il reste encore de nombreuses parts d'ombre sur l'organisation et la régulation des génomes procaryotes. Quant à la dynamique évolutive de ces génomes, nous avons vu qu'elle pose encore de nombreux problèmes aux spécialistes de la phylogénie. Enfin, les procaryotes sont toujours autant étudiés, car ce sont des réservoirs d'enzymes et processus chimiques qui peuvent être utilisés dans de nombreux domaines. Des molécules et des réactions qui nous sont parfois encore inconnus et que nous sommes incapables de reproduire. Dans cette partie, je décrirai les mécanismes les plus connus et les plus répandus qui seront également des principes fondamentaux de nos hypothèses de développement méthodologique et d'analyse pangénomique. Je laisserai donc à chacun se faire une idée de la simplicité des génomes procaryotes.

Les génomes procaryotes sont souvent décrits comme plus simples et plus faciles à étudier que les génomes eucaryotes. La simplicité apparente des génomes cache en réalité des mécanismes complexes. Dans cette partie, je décrirai les mécanismes les plus connus et les plus répandus. Je laisserai donc chacun se faire une idée de la simplicité des génomes procaryotes.

\section{Structure et organisation des génomes procaryotes}
\label{sec:structure_org}

%Le génome correspond à l'ensemble du matériel génétique, c.-à-d., des éléments qui seront hérités par les cellules de la génération suivante. Le génome, c'est aussi la structure de base qui va contenir l'ensemble des informations nécessaires au fonctionnement et à la survie de la cellule. Ces informations sont contenues dans la molécule d'ADN, ce qui nous amène à la structure primaire du génome, la séquence nucléotidique. Cette séquence est souvent circulaire chez les procaryotes et est de petite taille, quelques centaines de milliers de bases, mais certains génomes peuvent atteindre plusieurs millions de bases\footnote{En bioinformatique, on utilise l'unité base (b) ou paire de base (pb), pour mesurer la taille d'un génome. Un génome procaryote sera donc compris entre 100 kb et 10 Mb. Pour comparaison, le génome humain mesure environs 3 Gb.} (\autoref{fig:genome_size}).

Le génome procaryote correspond à la séquence de nucléotides qui composent la molécule d'ADN qui est bicaténaire, \textit{i.e.}, composée de deux brins antiparallèles reliés par complémentarité des bases (A$\leftrightarrow$T, C$\leftrightarrow$G). Le génome est souvent circulaire et de petite taille, quelques centaines de milliers de bases, mais certains génomes peuvent atteindre plusieurs millions de bases\footnote{En bioinformatique, on utilise l'unité base (b) ou paire de base (pb), pour mesurer la taille d'un génome. Un génome procaryote sera donc compris entre 100 kb et 15 Mb. Pour comparaison, le génome humain mesure environs 3 Gb.} (\autoref{fig:genome_size}). Enfin, le génome se divise en deux grandes catégories : l'ADN codant et l'ADN non codant.

\begin{figure}[htbp]
    \centering
    \includegraphics[width=\textwidth]{images/genome_sizes_boxplot.png}
    \caption[Distribution de la taille des génomes chez les procaryotes]{Distribution de la taille des génomes (en base) par classe chez les procaryotes. Les données utilisées proviennent de RefSeq version 28 janvier 2025.}
    \label{fig:genome_size}
\end{figure}

\newpage
\subsection{Constituant du génome : le codant et le non codant}
\label{sec:gene}
%\subsection{Organisation génique des procaryotes}
Chez les procaryotes, le génome est majoritairement constitué de séquences codantes (entre 85 et 90 \%), ce qui compense leur petite taille. 
L'ADN codant correspond à des blocs de nucléotides d’environ 1 kb. Ces blocs codants sont appelés gènes et ils jouent un rôle essentiel puisqu’ils contiennent l’information nécessaire à la production des protéines impliquées dans toutes les réactions cellulaires (cf. \autoref{sec:fn_reg}). De plus, l’ADN procaryote étant bicaténaire, chaque gène peut alors être lu dans les deux directions (sur le brin direct ou complémentaire), doublant ainsi la quantité d'information sur une position précise du génome (locus). 

Dans le génome, les gènes ne sont pas répartis aléatoirement. Ceux qui codent une fonction biologique similaire sont souvent regroupés dans un  contexte génomique. La conservation de l'ordre des gènes, appelé aussi synténie, peut varier entre les génomes, mais les gènes restent dans le même contexte \cite{lathe_gene_2000}, on parle alors de contexte conservé ou de synténie conservée. De plus, la position des gènes par rapport à l'origine de réplication (Ori: région où commence la réplication de l'ADN) à aussi son importance. Il a été montré que chez les bactéries avec un fort taux de division, les gènes ayant un rôle essentiel sont plus proches de l'Ori afin d'être plus fortement exprimés \cite{sharp_chromosomal_1989,vieira-silva_systemic_2010}.
%Les gènes sont soumis à des mécanismes de régulation communs (cf. \autoref{sec:fn_reg}).

Pour finir, les gènes peuvent être classés selon l’importance de leur fonction pour la survie de la cellule. Les gènes indispensables au cycle de vie d'une cellule, par exemple la réplication de l’ADN, la transcription, ou la traduction, sont dits "essentiel" et se distinguent des gènes "accessoires", qui codent pour des fonctions d'adaptation à des conditions particulières, comme la résistance aux antibiotiques, la défense contre les virus ou des transformations métaboliques spécifiques.



%Le génome est divisé en sous-unité que l'on appelle gène. Le gène contient l'information nécessaire pour produire une protéine qui réalisera une fonction dans la cellule (\autoref{fig:gene2prod}), on dit que le gène code pour une protéine. Pour ça, l'ADN est d'abord transcrit en une molécule d'ARNm, qui sera traduite en protéine (gène A sur la \autoref{fig:genome_size}) par des complexes protéine/ARN, les ribosomes. Ces protéines correspondent à une chaîne d'acides aminés, que l'on peut représenter sous forme de séquence. Pour passer d'un gène à une protéine, on utilise une table de correspondance que l'on appelle code génétique où 3 nucléotides correspondent à 1 acide aminé. En moyenne, une protéine contient 300 acides aminés, ramenant la taille des gènes à environ 1 kb. Enfin, comme indiqué sur la partie haute de la \autoref{fig:genome_size}, les génomes procaryotes sont majoritairement codants, ce qui veut dire que presque tout l'ADN peut être divisé en gènes, et donc qu'il y a environ entre 100 et 10 000 gènes dans les génomes en fonction de leur taille. En mettant toutes ces informations en perspective, on comprend que la petite taille des génomes procaryotes est compensée par son fort taux de gènes, et qu'ainsi, il contient l'ensemble des protéines nécessaires à la survie de la cellule. 


\begin{figure}[htbp]
    \centering
    \includegraphics[width=\linewidth]{images/gene2prot.jpg}
    \caption[Produit d'un gène]{Produit d'un gène dans la cellule. Un gène est d'abord transcrit en ARN. Si l'ARN transcrit est dit messager (ARNm), il sera ensuite traduit en protéine, sinon l'ARN produit (ARNt, ARNr, miARN, ....) aura un rôle spécifique dans des processus cellulaire. Copié de RNBio, Sorbonne université. \url{https://rnbio.sorbonne-universite.fr/genetique_genotype1}}
    \label{fig:gene2prod}
\end{figure}

L'ADN non codant constitue une part tout de même importante du génome et selon l'adage "la nature a horreur du vide"\footnote{Citation d'Aristote qui, répondant à Démocrite, dit que l'univers ne pouvait être rempli de vide. On sait aujourd'hui que Démocrite avait raison, mais dans le cas des génomes procaryote, l'idée fonctionne.}. Cet ADN non codant, n'est donc pas inutile et renferme également des fonctions essentielles à la vie de la cellule.
Tout d'abord, on retrouve les séquences d'ADN qui seront transcrits en ARN ribosomiques (ARNr) ou ARN de transfert (ARNt). Ces ARN sont indispensables pour la construction de la chaîne d'acide aminée de la protéine. Ces séquences d'ADN sont considérée aussi comme des gènes et selon les sources comme faisant partie du codant. Cependant, d'un point de vue sémantique et biologique ces gènes ne code pas, les nucléotides sont copiés d'une forme d'acide désoxyribonucléique en une forme d'acide ribonucléique. L'ADN non codant renferme d'autres formes d'ARN, comme les microARN et les ARN interférents (miARN et siARN). Ces ARN sont aujourd'hui considérés comme des acteurs clés dans la régulation des fonctions biologiques \cite{backofen_bioinformatics_2014,watkins_regulatory_2019}, mais aussi dans d'autres processus comme le système immunitaire \cite{bobadilla_ugarte_argonaute_2023}.

L'ADN non codant n'a pas uniquement le rôle de contenir les séquences transcrites en ARN, il contient aussi d'autres éléments régulateurs de l'expression des gènes contenus dans l'espace intergénique (cf. \autoref{sec:fn_reg}). On retrouve aussi dans l'ADN non codant des séquences répétées, comme les séquences d'insertion (IS) qui se déplace dans le génome, ou les séquence CRISPR (Régions composées de répétitions palindromiques et d’espacers, impliquées dans le système immunitaire adaptatif des bactéries) \cite{jansen_identification_2002,bolotin_clustered_2005}. Il existe tout de même une partie d'ADN non codant qui n'a aucun rôle, ces séquences sont des vestiges d'anciens gènes qui au cours de l'évolution ont perdu leur fonction (cf. \autoref{sec:dyn_evo}). Pour terminer, c'est aussi dans le non-codant que l'on va retrouver des éléments essentiels dans la réplication et l'évolution des génomes procaryotes : l'origine de réplication (Ori) et les éléments génétique mobile (MGE). 
%Tous les gènes ne sont pas traduits en protéine, une partie de ces gènes seront transcrit dans des formes d'ARN ayant un rôle dans la régulation et le fonctionnement de la cellule. Les ARN ribosomiques (ARNr), sont les constituants fondamentaux de la structure et du fonctionnement des ribosomes. Ils vont interagir avec les ARN de transfert (ARNt), qui acheminent les acides aminés vers les ribosomes pour traduire l'ARNm en protéine. D'autres ARN, comme les microARN et les ARN interférents (miARN et siARN), interviennent dans la régulation de l'expression des gènes. Cette liste non exhaustive montre la diversité des ARN et beaucoup étaient encore considérés il y a peu comme des produits secondaires sans réelle fonction. Aujourd'hui, ils sont considérés comme des acteurs clés dans la régulation des fonctions biologiques \cite{watkins_regulatory_2019,backofen_bioinformatics_2014}, mais aussi dans d'autres processus comme le système immunitaire \cite{bobadilla_ugarte_argonaute_2023}

\subsection{Réplicons et mécanismes de réplication dans les génomes procaryotes}
\label{sec:replicons}
La multiplication des cellules procaryotes s'effectue par division, où une cellule mère donne naissance à deux cellules filles. Afin de transmettre l’information génétique aux cellules nouvellement formées, l’ADN doit être répliqué, \textit{i.e.}, copié de manière exacte. Le terme réplicon désigne l’ensemble des molécules d’ADN capables de se répliquer de façon autonome. Un réplicon contient ainsi tous les éléments nécessaires à l’exécution et à la régulation de la réplication. Chaque réplicon contient une séquence d’ADN spécifique, appelée origine de réplication (Ori), où commence le processus de réplication.

La forme principale de réplicon dans la cellule procaryote est le chromosome. Le chromosome, souvent circulaire et replié, constitue le plus grand réplicon en termes de paires de bases. Chez les procaryotes, le chromosome est généralement unique, bien que d'autres réplicons puissent coexister au sein de la cellule.

Une seconde forme de réplicon, connue pour son rôle dans l’évolution (voir \autoref{sec:evo_hz}), est le plasmide \cite{lederberg_gene_1946,lederberg_sex_1953}. Les plasmides, souvent circulaires et de taille inférieure à celle du chromosome, sont indépendants de ce dernier. En tant que réplicons, ils se répliquent de manière autonome et peuvent être présents en grand nombre dans une cellule. L’origine de réplication des plasmides diffère de celle des chromosomes. Par ailleurs, les plasmides peuvent accumuler de nouvelles séquences et augmenter en taille, prenant alors la forme de mégaplasmides (\autoref{fig:replicon}).

Chez la majorité des procaryotes, le chromosome contient les gènes essentiels, tandis que les plasmides portent des gènes accessoires. Cependant, certaines formes de réplicons oscillent entre chromosome et plasmide. Par exemple, chez \textit{Rhodobacter sphaeroides}\footnote{Bactérie présente dans les lacs profonds et les eaux stagnants. Elle est capable de réaliser la photosynthèse et son métaolisme est très diversifié et donc très utilisé en biotechnologie} et \textit{Vibrio cholerae}\footnote{Bactérie responsable du choléra, on la retrouve dans l'eau et peut se propager entre humain en utilisant la transpiration}, un second chromosome a été identifié \cite{suwanto_physical_1989,trucksis_vibrio_1998}. Aujourd'hui, ces chromosomes secondaires sont distingués d'une forme de réplicons proche, le chromide \cite{harrison_introducing_2010}. Les chromides, de taille intermédiaire entre un plasmide et un chromosome principal, contiennent des gènes essentiels à la cellule. Ces gènes présentent une proximité phylogénétique avec les espèces du même genre, contrairement à ceux du chromosome principal, qui sont conservés au-delà du genre. En revanche, en termes de mécanismes de réplication et de séquences Ori, les chromids utilisent des systèmes de type plasmidique.

L’usage des termes chromosome secondaire, chromid et mégaplasmide demeure actuellement peu standardisé dans la littérature \cite{hall_what_2021}. Plusieurs critères permettent néanmoins de les distinguer. Le premier repose sur le contenu génétique : les mégaplasmides n’abritent pas de gènes essentiels, contrairement aux chromosomes secondaires et aux chromids. Le second critère est la composition en nucléotides, qui est plus proche de celle du chromosome principal pour les chromids et les chromosomes secondaires. Enfin, leur origine évolutive les différencie : le chromosome secondaire résulte de la scission d’un chromosome ancestral en un chromosome principal et un secondaire, tandis que le chromid dérive d’un ancien mégaplasmide ayant perdu sa capacité de mobilité (voir \autoref{sec:evo_hz}) et qui a intégré des gènes essentiels (\autoref{fig:replicon}). Les chromides auraient donc plutôt un rôle de réservoir de gènes d'intérêt et d'adaptation améliorant la \textit{fitness} des organismes. Cette vision vertueuse de l'accumulation de gènes s'oppose directement à la vision plus ancienne des plasmides non mobilisable décrits comme parasitant la cellule \cite{levin_accessory_1993,lili_persistence_2007}.

\begin{figure}[htbp]
    \centering
    \includegraphics[width=0.8\linewidth]{images/replicon.jpg}
    \caption[Évolution d'un plasmide en chromid]{Schéma simplifié de l'évolution d'un plasmide en megaplasmide et de megaplasmide à chromid. Figure extraite de \cite{hall_what_2021}}
    \label{fig:replicon}
\end{figure}



\section{Dynamique évolutive des génomes : mécanismes et impacts}
\label{sec:dyn_evo}

L'étude de l'évolution des génomes se concentre sur les changements affectant la séquence d'ADN des cellules. Ces changements se divisent en deux grandes catégories :  
\begin{enumerate}
    \item Les \textbf{mutations}, qui impliquent une modification de la séquence nucléotidique.  
    \item Les \textbf{réarrangements}, qui réorganisent l'ordre des nucléotides sans altérer leur composition.  
\end{enumerate}

Les mutations et les réarrangements peuvent entraîner un gain, une perte ou une modification de la séquence génétique, selon le mécanisme sous-jacent. Ces mécanismes, complexes et diversifiés, différent radicalement de ce que l'on pourrait imaginer en adoptant un point de vue anthropomorphique. Par exemple, les cellules procaryotes ne s'accouplent pas pour engendrer de nouvelles cellules. Dans la nature, les procaryotes se multiplient par division cellulaire : une cellule mère se divise en deux cellules filles identiques, exception faite des modifications génétiques survenues, comme décrit dans la \autoref{sec:evo_ver}. Lorsque l'ADN est transmis de la cellule mère à ses descendantes, on parle de \textbf{transfert vertical}.  

Cependant, il existe également des processus souvent comparés, par analogie, à une forme de sexualité chez les procaryotes : deux cellules échangent du matériel génétique sans donner naissance à une nouvelle cellule. Dans ce cas, l'ADN est transféré entre une cellule donneuse et une cellule receveuse de la même génération. Ce processus est appelé \textbf{transfert horizontal} (voir \autoref{sec:evo_hz}). Les mécanismes de transfert horizontal observés dans la nature sont largement exploités en microbiologie et en biologie cellulaire pour introduire des modifications spécifiques dans le génome des organismes. Cela permet de créer des espèces chimériques ou hybrides, utilisées à des fins de recherche ou industrielles \cite{baby_chromosomes_2019}. 

Une autre voie évolutive impliquant des procaryotes se base sur leur capacité à vivre en symbiose, voire en endosymbiose\footnote{Une bactérie résidant à l'intérieur d'une autre cellule, qu'elle soit procaryote ou eucaryote}. Cette interaction pourrait être considérée comme une étape préliminaire à une "fusion" évolutive. Cet évènement est rare, mais serait à l'origine d'organites tels que la mitochondrie et le chloroplaste \cite{martin_endosymbiotic_2015}. La fusion de cellules procaryotes peut également être réalisée en laboratoire. En supprimant leur paroi cellulaire, on obtient des \textbf{protoplastes}, qui peuvent être fusionnés à l'aide d'agents chimiques (comme le polyéthylène glycol) ou par chocs électriques (électrofusion) \cite{schaeffer_fusion_1976}.

\subsection{Mécanismes d'évolution par héritage}
\label{sec:evo_ver}
Les mécanismes d'évolution par héritage regroupent les processus menant à une modification du génome entre la cellule mère et la cellule fille. Théoriquement, lors de la division cellulaire, la cellule mère se divise en 2 cellules filles possédant exactement la même information génétique qu'elle. Pourtant, malgré un ensemble de mécanismes de protection et de correction de l'ADN, le génome peut différer entre les cellules mère et filles. Ce sont ces "erreurs" qui vont nous intéresser, car ce sont elles qui sont à l'origine de l'innovation et de la diversité génétique.

\subsubsection{Mutation génétique : un petit changement aux grandes conséquences}
\paragraph{\textit{Single Nucleotid Polymorphism}}

Un \textit{Single Nucleotide Polymorphism} (SNP) est un mécanisme d'évolution qui induit une modification de la séquence par la transformation d'un nucléotide en un autre. Étant donné que le code génétique est dégénéré\footnote{Un acide aminé peut être codé par plusieurs codons différents.}, la mutation peut ne pas avoir d'impact sur la séquence de la protéine, on dit que la mutation est silencieuse ou même sens. Si la modification change la séquence protéique, dans ce cas, on parle de mutation faux-sens. Enfin, Une mutation est qualifiée de non-sens lorsqu'elle affecte un point clé de la séquence protéique, comme le site actif ou un codon STOP, entraînant une perte de fonction de la protéine, ou lorsqu'elle introduit prématurément un codon STOP dans la séquence. Sur la \autoref{fig:mec_evo}, la première mutation implique un changement de glutamine en histidine, des acides aminés aux propriétés de polarité et de charge différente, c'est donc une mutation faux-sens qui aura un impact certainement important sur la structure de la protéine. Les 2 autres SNP redonnent le même acide aminé, elles sont donc silencieuses.

\begin{figure}[htbp]
    \centering
    \includegraphics[width=0.8\textwidth,height=\textheight,keepaspectratio]{images/Mec_evo.jpg}
    \caption[Identification des SNP et indels entre 2 génomes]{SNP et InDels entre deux génomes. On suppose que le premier codon commence par le premier nucléotide. Figure extraite et adaptée de \cite{qi_detection_2014}}
    \label{fig:mec_evo}
\end{figure}

\paragraph{Indels: insertion, délétion et pseudogènes}

Un indel correspond à l'insertion (In) ou la délétion (del)\footnote{On regroupe l'insertion et la délétion, car sans une analyse phylogénétique, il est impossible de les différencier par comparaison de séquence.} d'un ou plusieurs nucléotides dans la séquence d'un gène. 

Lorsque la taille de l'indel est un multiple de 3 (insertion ou délétion d'un codon), la séquence protéique peut soit être allongée soit raccourcie d'un acide aminé, soit coupée de façon précoce si le codon est un codon STOP.

Si la taille de l'indel n'est pas un multiple de 3, il y aura un décalage du cadre de lecture ou \textit{frameshift}. Ce décalage va induire un changement de tous les acides aminés de l'indel à la fin du gène, provoquant avec lui un changement dans la fonction de la protéine ou une inactivation de la fonction. La partie du gène qui n'est pas décalée est alors considérée comme un fragment du gène initial, il est alors qualifié de pseudogène. À nouveau, cette mutation peut être délétère pour la cellule. Sur la \autoref{fig:mec_evo}, les Indels sont de taille 1 et 2, elles ne provoquent pas l'apparition d'un codon STOP précoce, mais l'ensemble des acides aminés est modifié.

Les indels vont donc transformer la séquence protéique traduite, pouvant nuire à la fonction de cette dernière et être délétère pour l'organisme. Pour éviter les problèmes liés aux \textit{frameshifts}, il a été montré qu'il existe un fort taux de codon STOP hors du cadre de lecture \cite{tse_natural_2010}. Cette adaptation permettrait de limiter la traduction des protéines mutantes et d'ainsi limiter le coût énergétique pour la cellule. Il a aussi été montré que les \textit{frameshifts} pourraient être à l'origine d'un réservoir d'adaptation à l'environnement \cite{koch_catastrophe_2004}. Lors d'un changement dans l'environnement créant une nouvelle pression de sélection, un \textit{frameshift} pourrait produire une protéine qui permet à l'organisme de s'adapter à son environnement et donc d'améliorer sa \textit{fitness}\footnote{Le \textit{fitness} correspond à la capacité d'un individu de survivre dans son environnement et à se reproduire}. Une fois que l'élément perturbateur de l'environnement disparaît, un nouveau \textit{frameshift} pourrait ramener le cadre de lecture à sa place d'origine. Ce mécanisme, en accord avec la petite taille des génomes, aurait l'intérêt de ne pas perdre des gènes d'adaptation à l'environnement, même s'ils ne sont nécessaires que ponctuellement.

\subsubsection{Réarrangement génomique : un moteur de l'évolution}
\label{sec:rearragement}
Les mécanismes de réarrangement génomique sont très importants dans l'évolution des génomes, mais qui, par rapport aux mutations, impliquent des segments d'ADN plus important. La forme du génome obtenue, appelée un variant structural (SV pour \textit{Structural variant} en anglais), est plus difficile à détecter que les SNP et les indels \cite{periwal_insights_2015}.

Le mécanisme de recombinaison est à l'origine des réarrangements. Une recombinaison implique l'échange de 2 portions d'ADN entre 2 molécules ou 2 régions d'ADN. La recombinaison peut être homologue, se produisant entre des séquences similaires, ou non homologue, impliquant des séquences différentes. Elle est souvent médiée par des enzymes spécialisées comme RecA ou des intégrases, qui permettent l'intégration, la réparation ou le réarrangement précis des séquences. La recombinaison homologue est cruciale pour la réparation des cassures de l'ADN, les réarrangements et également dans l'acquisition de nouveaux gènes par transfert horizontal (cf. \autoref{sec:evo_hz}) \cite{eisenstark_genetic_1977}.

Les réarrangements de l'ADN correspondent donc globalement à un échange entre 2 segments du génome, induisant une insertion, une délétion ou une modification de l'ordre des nucléotides (\autoref{fig:rearrangement}). Les réarrangements sont fréquents dans les génomes procaryotes \cite{sun_genome-wide_2012}. L'ordre des gènes étant important (cf. \autoref{sec:structure_org}) dans l'expression des gènes et la fonction des protéines, le SV résultant peut conduire à une modification de l'expression génique ou un changement dans la fonction de la protéine. Il existe 3 formes de réarrangement : symétrique, asymétrique et au sein d'un réplicon. Ces formes ne sont pas toutes équiprobables, car elles affectent plus ou moins la structure du génome. Aussi, les réarrangements proches de l'Ori sont plus fréquents que ceux proches du site de terminaison \cite{darling_dynamics_2008}. 

\begin{figure}[htbp]
    \centering
    \includegraphics[width=\textwidth,height=0.45\textheight,keepaspectratio]{images/rearrangement.png}
    \caption[Réarrangement et implication]{Réarrangement et conséquences des variants structuraux. (A) Région génomique sans SV. Les réctangles représentent les gènes et les petits connecteurs à côté représentent le promoteur du gène concerné. (B) Réarrangement intragénique illustrant la délétion et la fusion de gènes à la suite d'une duplication partielle du gène. Les régions codantes modifiées produisent des transcrits aberrants. La délétion ou la duplication peut entraîner une modification du nombre des gènes dans des régions par ailleurs fonctionnellement intactes. (C) Délétion du promoteur, la régulation est modifiée et une duplication/délétion qui modifie le nombre de copis des gènes. (D) Inversions affectant la structure du gène, le gène est inversé, retourné et réarrangé, ce qui éloigne l'un des promoteurs du premier gène (orange). (E) Translocations affectant le contexte génique. Figure extraite et adaptée de \cite{periwal_insights_2015}}
    \label{fig:rearrangement}
\end{figure}

Les recombinaisons peuvent également mener à la copie de gènes ou de régions génomiques. Cet événement de duplication joue un rôle essentiel dans l'évolution des procaryotes en fournissant une redondance génétique : l'une des copies du gène peut conserver sa fonction d'origine, tandis que l'autre peut accumuler des mutations sans affecter la survie de l'organisme. De plus, la duplication peut également avoir un rôle dans l'expression de gènes spécifiques. C'est le cas des gènes codant pour les pompes à efflux qui évacuent les antibiotiques de l'organisme, qui sont fortement dupliqués \cite{maddamsetti_duplicated_2024}. Dans les génomes, les événements de duplication semblent minoritaires par rapport aux transferts horizontaux \cite{tria_gene_2021}, et ceux en partie dus à l'élimination de la redondance dans les génomes. 

\subsection{Mécanismes d'évolution intragénérationnelle}
\label{sec:evo_hz}

Les mécanismes qui viennent d'être décrits apportent donc de l'innovation dans les génomes procaryotes, ces innovations doivent ensuite être transmises dans la population. Si on ne prend en compte que le transfert vertical, l'innovation ne peut être transmise que d'une génération à une autre, or le temps de génération\footnote{temps nécessaire pour que le nombre de cellules double} peut être relativement long (20 minutes chez \textit{E. coli}, 80 min pour \textit{Lactobacillus acidophilus}\footnote{Bactérie probiotique, elle est utilisée dans la composition de lait fermenté et d'anti-infectieux intestinaux.} et 800 min pour \textit{Mycobacterium tuberculosis}\footnote{Bactérie responsable de la tuberculose chez l'homme, on la retrouve principalement dans les voies respiratoire.}). De plus, un temps de génération plus grand semblerait diminuer le taux de mutation spontanée de l'ADN \cite{weller_generation-time_2015}. Les procaryotes possèdent donc d'autres mécanismes pour échanger de l'ADN avec leur environnement (autres procaryotes de la même espèce ou non, virus, eucaryotes, ADN libre\dots) leur permettant d'intégrer un segment d'ADN contenant une potentielle innovation. Ces mécanismes sont regroupés sous le terme de transfert horizontal. 

\subsubsection{Conjugaison : la sexualité des procaryotes}

La conjugaison a été découverte en 1946 par Joshua Lederberg et Edward L. Tatum \cite{lederberg_sex_1953}, qui décrivent ce mécanisme comme la manière sexuée des bactéries d'échanger de l'ADN. En effet, par analogie, la conjugaison demande un contact direct entre une cellule donneuse et une cellule receveuse pour l'échange de matériel génétique\footnote{N.B : Le transfert est unidirectionnel, la cellule donneuse ne peut recevoir de l'ADN et la receveuse ne peut en donner.}. Il existe 2 catégories d'élément génétique mobile (MGE, pour \textit{Mobile Genetic element} en anglais) conjugatif: les plasmides (cf. \autoref{sec:replicons}) et les éléments intégratif et conjugatifs (ICEs, \textit{Intergrative and Conjugative elements} en anglais). Sur la \autoref{fig:conjugaison} est représenté l'échange d'un plasmide par conjugaison. Les ICEs \cite{johnson_integrative_2015}, contrairement aux plasmides, sont directement intégrés au chromosome, ce qui rend leur réplication dépendante du chromosome, mais qui permet leur transfert de manière verticale aussi. Les ICEs pour être échangé doivent suivre un schéma circulaire : excision du chromosome, circularisation, réplication, transfert et réintégration dans le chromosome. Lors de l'étape d'excision, il peut arriver que des gènes flanquant l'ICEs soit excisé aussi, apportant une nouvelle forme à l'ICE \cite{gibbons_genomic_2011}.

\begin{figure}[htbp]
    \centering
    \includegraphics[width=0.8\linewidth]{images/Conjugation.png}
    \caption[Schéma du fonctionnement de la conjugaison]{Schéma du fonctionnement de la conjugaison, dans le cas d'un plasmide conjugatif. (1) Formation d'un pili sexuel par la bactérie donneuse. (2) Contact direct entre les 2 bactéries via le pili. (3) Réplication de l'ADN plasmidique et transfert à la bactérie donneuse. (4) Terminaison de la conjugaison et nouvelle formation d'un pili pour la receveuse devenue donneuse. Image sous licence Creative Commons 3.0 \url{https://commons.wikimedia.org/wiki/File:Conjugation.svg}}
    \label{fig:conjugaison}
\end{figure}

Plasmides et ICEs sont généralement de petite taille, mais ils contiennent des gènes clés d'adaptation aux conditions environnementale. La colocalisation des gènes d'adaptation avec ceux de la conjugaison permet à des colonies de répondre efficacement et rapidement aux nouvelles conditions environnementales, comme la présence de métaux lourds ou d'antibiotique \cite{botelho_role_2021}. Toutefois, tous les MGEs ne sont pas forcément conjugatif \cite{valentine_mobilization_1988}, ils vont profiter de la conjugaison codée par un autre élément pour se transférer. Dans ces conditions, la bactérie receveuse ne devient pas conjugative à son tour, même si elle reçoit l'élément mobile. Ces éléments mobilisables sont appelés des IMEs (élément intégratif mobilisable). Il est d'ailleurs à noter que tous les plasmides ne sont pas mobilisables, il y aurait d'ailleurs autant de plasmide conjugatif que de plasmide non mobilisable \cite{smillie_mobility_2010}.

La conjugaison est un mécanisme majeur de transfert horizontal de matériel génétique, qui a la caractéristique de rapidement répandre les éléments mobiles. Il a toutefois le défaut de limiter le transfert de gènes entre cellule procaryote et donc de limiter le transfert aux innovations génétique déjà intégré par un autre organisme procaryote. Ce n'est pas le cas des prochains mécanismes que nous décrirons.   

\subsubsection{Transformation : recycler l'ADN environnant}

La transformation correspond à l'intégration d'un fragment d'ADN étranger dans le génome de l'organisme. Ce qui  différencie la transformation de la conjugaison, c'est que l'ADN intégré est libre dans l'environnement
\footnote{La découverte de la transformation en 1928 par Fred Griffith \cite{griffith_significance_1928}, précède de nombreuses années celle qui a mis en évidence que l'ADN est le porteur de l'information génétique \cite{avery_studies_1944}. La transformation est donc une preuve anticipée et un socle pour démontrer le rôle de l'ADN.}. Bien que la transformation soit un processus répandu chez les bactéries, elles n'en sont pas toutes capables. Les bactéries pouvant réaliser la transformation sont dites compétentes. De plus, même si les mécanismes de la transformation sont bien décrits \cite{johnston_bacterial_2014,dubnau_mechanisms_2019}, notamment sur l'incorporation de l'ADN dans la cellule (\autoref{fig:transformation}), d'une espèce procaryote à l'autre, ils peuvent varier tout comme la proportion de transformation réalisée \cite{stewart_biology_1986}. Nous ne reviendrons donc pas sur les mécanismes, mais seulement sur des exemples d'application.

\begin{figure}[htbp]
    \centering
    \includegraphics[width=0.8\linewidth]{images/transformation.png}
    \caption[Schéma du mécanisme de transformation]{Schéma du mécanisme de transformation. Extrait de \cite{johnston_bacterial_2014}}
    \label{fig:transformation}
\end{figure}

Les bactéries du genre \textit{Nesseria} et particulièrement \textit{N. gonorrhoeae}\footnote{Ce genre bactérien vit dans les muqueuses des mammifères et est non pathogène à l'exception de \textit{N. meningitidis}, impliqué dans la méningite et \textit{N. gonorrhoeae}, responsable de la gonorrhée, une infection sexuellement transmissible.} reconnaissent préférentiellement une séquence d'ADN non palindromique de leur propre ADN \cite{goodman_identification_1988,duffin_dna_2010}. Ce système permet d'intégrer uniquement l'ADN de souche proche, ainsi des gènes d'adaptation, comme de résistance aux antibiotiques \cite{centers_for_disease_control_and_prevention_cdc_update_2007}, sont distribués préférentiellement dans l'espèce.

Les \textit{Streptococcus pneumoniae}\footnote{Bactérie connue pour son rôle d'agent pathogène dans les pneumonies et responsable de co-infection pendant la grippe espagnole} utilise la transformation comme mécanisme de réparation de l'ADN, car cette espèce ne possède pas de système de réparation SOS. Les \textit{S. pneumoniae} s'engagent alors dans une "guerre fratricide" pour récupérer l'ADN de son espèce.

Pour terminer, chez \textit{Bacillus subtilis}\footnote{Bactérie du sol, mais qu'on retrouve dans de nombreux habitat dû à ses capacités d'adaptation. Elle est utilisé comme modèle d'étude des bactérie Gram+.}, la transformation entre individus de la même espèce, mais de souche éloignée, est privilégiée \cite{lyons_combinatorial_2016}. Les bactéries vont sécréter dans l'environnement des antibiotiques, auxquels elles sont résistantes, pour tuer les autres individus de l'espèce. L'ADN récupéré est donc différent de celui de la bactérie et donc potentiellement source de nouvelles fonctions.

Ces exemples montrent aussi une opposition dans la philosophie des mécanismes de conjugaison et de transformation. La transformation demande que l'ADN soit libre dans l'environnement et donc que les bactéries environnantes soient détruites, alors que la conjugaison laisse les 2 cellules en vie. 


\subsubsection{Transduction : les virus mis à profit}

La transduction est un mécanisme reposant sur l'intervention d'un virus pour transporter et transférer le matériel génétique d'une cellule procaryote à l'autre (\autoref{fig:transduction}). Les virus de bactéries, surnommés (bacterio)phages\footnote{Officiellement bactériophage, mais raccourci dans l'usage en phage}, vont infecter la cellule donneuse pour répliquer leur ADN. Lors de la réplication, de l'ADN de la cellule donneuse peut se trouver intégré à celui du phage. Lorsqu'il infectera une cellule receveuse, la portion d'ADN de la donneuse pourra reprendre une forme plasmidique (si c'est un plasmide qui a été transféré) ou être intégrée au génome de la cellule par recombinaison homologue. La transduction est aujourd'hui largement utilisée en génétique et microbiologie pour transférer de l'ADN et modifier les génomes \cite{wang_phage-based_2024}. 

\begin{figure}[htbp]
    \centering
    \includegraphics[width=0.65\linewidth]{images/transduction.png}
    \caption[Schéma synthétique de la transduction]{Schéma représentant les étapes de transduction. Extrait de \cite{chiang_genetic_2019}}
    \label{fig:transduction}
\end{figure}

La première forme de transduction identifiée décrivait le transfert de n'importe quel gène de la donneuse à la receveuse par le phage. Cette forme a donc été nommé transduction généralisée \cite{zinder_genetic_1952}. Une seconde forme dites spécifique a été découverte en étudiant le phage $\lambda$ infectant les \textit{E. coli} \cite{morse_transduction_1956}. Le transfert se limite à un ensemble de gènes définis. Enfin, une dernière forme, la transduction latérale, a récemment été découverte \cite{chen_genome_2018}. Là où les formes générale et spécifique peuvent être vues comme une erreur et un évènement lié au hasard, la transduction latérale fait partie du cycle de vie du phage, menant à un taux de transfert beaucoup plus important. 

\newpage
\section{Du génome aux processus cellulaires}

\subsection{Gènes : Régulations et fonctions}
\label{sec:fn_reg}

Les réactions qui se produisent dans les cellules procaryotes sont souvent complexes et impliquent une multitude de réactifs et de produits. Toutes ces réactions nécessitent la présence de protéines spécifiques pour être réalisées. Ces protéines sont produites et dégradées par la cellule en fonction des conditions rencontrées. C'est pourquoi l'information est stockée dans une structure durable et transmissible, le gène. Chaque gène sera transcrit en une molécule d'ARN messager (ARNm) par l'ARN polymérase, qui sera traduite en protéine par le ribosome (impliquant l'ARNr et l'ARNt). 


Dans une cellule, les protéines ont un temps de "vie" allant de quelques minutes à quelques heures. Il est donc nécessaire de produire les protéines régulièrement, toutefois cette production a un coût pour la cellule. C'est pourquoi il existe des mécanismes de régulation de l'expression des gènes et donc de la production des protéines. Dans la \autoref{sec:gene}, nous avons vu qu'il existait notamment des petits ARN régulateurs de l'expression. Dans l'ADN non codant, on retrouve également une séquence promotrice (ou promoteur) près d'un gène qui permet la fixation de l'ARN polymérase. La fixation et l'activation de l'ARN polymérase au niveau du promoteur sont régulées par des facteurs de transcription qui se lient spécifiquement à des séquences régulatrices en amont du promoteur, les \textit{enhancer} et \textit{silencer}.


Les protéines peuvent agir en collaboration, soit dans des réactions successives (cas des systèmes biologiques, cf. \autoref{sec:sys_bio}), soit en formant des complexes protéiques interagissant pour métaboliser un produit. Les gènes codant pour des protéines impliquées dans les mêmes processus cellulaires sont situés dans le même contexte génomique (cf. \autoref{sec:gene}). Ils vont alors être régulés par les mêmes éléments de régulation. L'opéron, une structure spécifique des procaryotes découverte par François Jacob et Jacques Monod en 1960\footnote{Découverte qui leur a valu le prix Nobel de médecine en 1965}\cite{jacob_genetic_1961}, permet de produire un seul ARNm pour un ensemble de gènes codant pour des protéines impliquées dans le même processus cellulaire. Dans l'opéron, se trouve une nouvelle séquence de régulation, l'opérateur, où va se lier une molécule régulatrice qui va activer ou inhiber la transcription (\autoref{fig:lac_operon}). L'ensemble de l'opéron permet de synchroniser la régulation et l'expression de gènes qui collaborent dans le même processus cellulaire.


\begin{figure}[htbp]
    \centering
    \includegraphics[width=\linewidth]{images/Lac_operon-2010-21-01.png}
    \caption[Exemple de l'opéron lactose]{\textbf{Schéma du fonctionnement de l'opéron lactose.} Sur la partie haute est représenté la structure génétique de l'opéron. Les 4 lignes suivantes représentent chacune une configuration de réponse à des conditions de présence, absence de glucose et de lactose. si le taux de glucose est faible, une protéine activatrice (CAP) va se fixer en amont du promoteur pour aider à la fixation de l'ARN polymérase, et si du lactose est disponible, les gènes seront alors fortement exprimés. Si le lactose n'est pas disponible, une protéine de répression va se fixer à l'opérateur et elle empêchera l'ARN polymérase de se fixer même si le taux de glucose est faible. 
    Auteur : G3pro. Sous licence Creative Commons 2.0. Disponible à l’adresse : \url{https://commons.wikimedia.org/wiki/File:Lac_operon-2010-21-01.png.}
    }
    \label{fig:lac_operon}
\end{figure}

Pour terminer, l'expression des gènes peut aussi être régulée par le niveau de repliement et de condensation de l'ADN. L'ADN est condensé notamment grâce à des protéines spécialisées et à la méthylation de l’ADN. L'ADN replié ne pourra pas être accessible pour la transcription des gènes et donc ils seront inactifs. Les mécanismes liés à la méthylation de l'ADN sont l'affaire de l’épigénétique. Des études récentes ont mis en lumière le rôle de la méthylation dans la régulation de la virulence bactérienne et dans la capacité des procaryotes à coloniser leurs hôtes \cite{oliveira_bacterial_2021}, soulignant ainsi l'importance de ces mécanismes dans la survie et l’adaptation des bactéries.

\newpage
\subsection{Îlots génomiques et points chauds d'insertion}
\label{sec:ilot}
Les îlots génomiques (GI, pour \textit{Genomic Island en anglais}) sont des régions spécifiques du génome qui jouent un rôle clé dans l'évolution, l'adaptation et l'acquisition de fonctions spécifiques. Les GIs sont retrouvés chez quasiment tous les organismes procaryotes. Ils sont généralement acquis par transfert horizontal (cf. \autoref{sec:evo_hz}) et transportent des gènes accessoires. Ils vont conférer à l'organisme de nouvelles fonctions qui impacteront de façon positive sa \textit{fitness}. Le premier îlot génomique décrit était lié à la capacité de la bactérie \textit{E. coli} de provoquer des maladies et a donc été nommé îlot de pathogénicité \cite{hacker_deletions_1990}.  Depuis, d'autres classes d'îlots ont été découvertes : métabolique, résistance, symbiotique\dots (\autoref{fig:GI}).

Les îlots génomiques sont des régions assez larges, entre 5 et 200 kb (mais certaines sont beaucoup plus grandes) et présentent des caractéristiques spécifiques. (\textit{i}) Les GIs ont un taux de GC qui diffère par rapport au reste du génome, résultant en un biais d'usage des codons\footnote{Un biais d'usage des codons, désigne la fréquence d’utilisation préférentielle de certains codons parmi les codons synonymes pour coder un même acide aminé.} (\autoref{fig:GI}). (\textit{ii}) dans les régions flanquantes des GIs, on retrouve des gènes de mobilité : transposases et intégrases, mais aussi d'IS qui peuvent se dégrader rapidement après l'intégration de l'îlot. (\textit{iii}) Dans les gènes flanquants, on retrouve des gènes codant l'ARNt dont l'origine serait à relier à la prévalence des gènes de phages et des ICEs qui utilisent les ARNt comme site d'intégration dans les génomes \cite{dobrindt_genomic_2004}. (\textit{iv}) Les protéines contenues dans les GIs ont souvent des fonctions inconnues. (\textit{v}) Dans la partie flanquante, on trouve des séquences répétées directes\footnote{Séquences identiques présentes en plusieurs copies dans la même molécule d'ADN et ayant la même orientation.}.

\begin{figure}[htbp]
    \centering
    \includegraphics[width=0.8\linewidth]{images/ilot_genomique.jpg}
    \caption[Îlots génomiques et leur caractéristique]{\textbf{Îlots génomiques et leur caractéristique.} Extrait de  \cite{da_silva_filho_comparative_2018}}
    \label{fig:GI}
\end{figure}

\newpage
Ces GIs sont complexes à étudier, car ils concentrent les variations, même entre génomes proches. L'histoire évolutive est souvent difficile à reconstituer, tant des éléments ont été intégrés et éliminés au cours du temps (\autoref{fig:cycle_IG}). En plus de s'échanger avec d'autres organismes \cite{buchrieser_high-pathogenicity_1998}, les GIs peuvent se déplacer au sein du génome \cite{karaolis_bacteriophage_1999}.

\begin{figure}[htbp]
    \centering
    \includegraphics[width=0.65\linewidth]{images/cycle_GI.png}
    \caption[Cycle de vie d'un îlot génomique]{Cycle de vie d'un îlot génomique. Extrait de \cite{dobrindt_genomic_2004}}
    \label{fig:cycle_IG}
\end{figure}

Les GIs ne s'insèrent pas n'importe où dans les génomes. On les retrouve fréquemment dans des zones où de nombreux éléments se sont insérés au cours de l'évolution d'un taxon. Ces régions sont appelées : point chaud d'insertion (\textit{hotspot} en anglais).
À l'intérieur des \textit{hotspots}, on retrouve une grande variabilité du contenu génique entre les génomes. Les \textit{hotspots} sont également caractérisés par des bordures composées de gènes communs à l'ensemble des génomes. 

Ils présentent également une recombinaison homologue accrue dans les gènes flanquant les \textit{hotspots}, avec 50 \% d’événements de recombinaison et 30 \% d’incongruence phylogénétique\footnote{L'incongruence phylogénétique désigne une discordance entre l’arbre phylogénétique d’un gène spécifique et l’arbre phylogénétique global construit à partir d'un grand nombre de gènes conservés.} par rapport à l'arbre des espèces. Ces \textit{hotspots} contiennent 50 \% des gènes acquis par HGT \cite{oliveira_chromosomal_2017}. Ils sont enrichis en gènes liés à la motilité, à la défense, à la transcription, à la réplication et à la réparation de l’ADN \cite{flores_ramos_genomic_2021}.

Le contenu génique du \textit{hotspot} provient d'une accumulation progressive de gènes, comme le suggère le faible pourcentage (8 \%) de \textit{hotspots} composés uniquement de gènes spécifiques à une souche \cite{oliveira_chromosomal_2017}. Cette accumulation peut se faire par bloc de gènes. Ces blocs, conservés dans le \textit{hotspot}, sont appelés modules \cite{lescat_module_2009}. Cette modularité pourrait expliquer l'organisation complexe des îlots génomiques \cite{touchon_organised_2009}.
Les \textit{hotspots}, sont donc communs à un groupe d'organismes, et définis au niveau d'un taxon. Il est donc nécessaire de mener des études de comparaison des génomes pour les identifier.