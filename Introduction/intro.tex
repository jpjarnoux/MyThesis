\chapter*{Introduction}
\addcontentsline{toc}{chapter}{Introduction}

Cette introduction a pour objectifs de faire le panorama scientifique et historique des différents sujets qui seront abordés dans ce manuscrit de thèse. Elle fait aussi place d'entrée en matière pour les personnes non expertes qui liront ce manuscrit. Pour ces quelques lignes, je me permettrais donc quelques facilités et imprécisions scientifiques, que j'espère me seront pardonnées.
\newline

L'origine de la vie sur Terre est encore pleine de question et de débat scientifique passionnant. Néanmoins, les premières traces de vie remontent à 4 milliards d'années et correspondent à des microorganismes, des êtres invisibles à l'{\oe}il nu. Ces microorganismes colonisent la Terre depuis des milliards d'années et représentent aujourd'hui la proportion d'être vivant la plus importante en termes de nombre et de diversité. Ils jouent un rôle crucial dans les écosystèmes, les cycles biogéochimiques et la santé de la planète comme de la nôtre. Pourtant, la microbiologie, l'étude des microorganismes, reste une science relativement récente. Même s'il existe bien, dans l'Antiquité, certains savants et philosophes qui avaient déjà imaginé ces "animaux invisibles", marquant une compréhension primitive de la transmission des maladies infectieuses, il faudra attendre l'invention du microscope par Leeuwenhoek, au XVIIe siècle, pour qu'il fasse les premières observations d'\textit{animaculum}, marquant la naissance de la microbiologie. La microbiologie du XVIIe au XXe siècle à amener de grandes découvertes et révolution scientifique, notamment en médecine. Nous pouvons citer les travaux de Louis Pasteur qui a prouvé que les maladies infectieuses étaient causées par des microorganismes, ou encore d'Alexander Fleming qui découvrit la pénicilline, le premier antibiotique. 


En s'éloignant quelque peu de la microbiologie, toujours entre le XVIIe et XXe siècle, les chimistes s'intéressent aux molécules du vivant. Le Français Antoine Fourcroy va faire la première description de substances azotées dans les organismes vivants, qu'il appelait "substances animales". Ce sont ensuite les chimistes suédois et néerlandais Jöns Jacob Berzelius et Gérardus Johannes Mulder, qui ont introduit le terme "protéine". Le mot vient du grec \textit{proteios}, qui signifie "de première importance", soulignant l'importance fondamentale de ces molécules composées de carbone, hydrogène, azote et oxygène, avec des proportions spécifiques, dans les organismes vivants. Enfin, en 1894, le chimiste allemand, Emil Fischer, démontra que les protéines sont composées d'acides aminés, unité de base des protéines, liés par des liaisons peptidiques. Il déterminera la composition et la structure de plusieurs d'entre eux. La fin du XIXe siècle voit aussi la découverte d'un autre molécule du vivant, l'acide désoxyribonucléique, mieux connue sous l'acronyme ADN. C'est le biologiste suisse Friedrich Miescher qui découvre une substance riche en phosphore dans les cellules du pus, qu'il appelle "nucléine". Il faudra attendre près d'un demi-siècle pour que Phoebus Levene, biochimiste russe-américain, identifie les composants de base de l'ADN : les nucléotides. Plus tard, en pleine Seconde Guerre mondiale, les chercheurs Oswald Avery, Colin MacLeod et Maclyn McCarty, confirment l'hypothèse de Miescher, en montrent que l’ADN est la substance qui transfère les caractères héréditaires et que l'ADN est le support de l’information génétique. Pour terminer, les travaux de Watson et Crick, sans oublier la contribution de Rosalind Franklin, a permis de décrire la structure de la molécule d'ADN. Toutes ces découvertes ont ouvert la voie à de nombreuses autres dans tous les domaines : médecine, agro-alimentaire, biotechnologie, et sont le socle de la génétique moderne.


Les développements technologiques du XXe siècle, et notament l'apparition de l'informatique, amènent les chercheurs à créer une nouvelle discipline pour l'étude de la structure et de la composition des molécules du vivant : la bioinformatique. Margaret Dayhoff, une pionnière de la bioinformatique des années 50, développe l'un des premiers programmes informatiques pour analyser les séquences de protéines. Elle publie d'ailleurs le premier atlas de séquences protéiques, jetant les bases de l'analyse des séquences biologiques. En 1970, Saul Needleman et Christian Wunsch introduisent un algorithme pour l'alignement global des séquences, qui est toujours utilisé aujourd'hui. En 1977, Frederick Sanger développe une méthode de séquençage de l'ADN qui portera son nom. Elle devient rapidement la méthode de référence en raison de sa précision. Dans les années 80, la méthode s'automatise, devient plus rapide et précise. On voit alors se développer les premières bases de données accessibles au public pour stocker des séquences génétiques et protéiques. En 1990, est lancé le Projet du Génome Humain (HGP), un effort international visant à séquencer l'intégralité du génome humain. Ce projet catalyse de nombreux développements en bioinformatique, notamment dans la gestion et l'analyse des grandes quantités de données générées. Enfin, au début des années 2000, les technologies de séquençage sont de plus en plus performantes et abordables, faisant entrer la bioinformatique dans l'âge du \textit{Big Data}, la rendant essentielle dans de nombreux domaines d'étude en biologie.


La microbiologie, et pour ce qui va nous intéresser ici l'étude de la génétique des microorganismes, profitent de toutes ces nouvelles technologies pour développer ces connaissances. Néanmoins, elle va aussi subir cette explosion de la quantité d'informations disponible dans les bases de données. C'est pourquoi, microbiologistes et bioinformaticiens sont toujours à la recherche de nouvelles méthodes pour l'analyse de ces données. Alors que les programmes bioinformatique s'attachaient à représenter et à étudier un génome en tant qu'une séquence indépendante des autres, un nouveau concept de représentation des génomes est apparue : le pangénome. Il permet de regrouper l'ensemble des génomes en une seule entité et donc rendre une représentation globale de l'ensemble de l'information contenu dans les génomes. Le pangénome garantie une meilleure représentation de la diversité des génomes, tout en étant plus adapté à l'analyse de grande quantité de données. C'est dans ce cadre que j'ai effectué mon travail de thèses, avec pour objectifs de proposer de nouvelles méthodes en pangénomique pour l'analyse comparé des génomes de microorganismes a l'ère du big data.
\newline

Cette riche introduction permet, je l'espère, de montrer le caractère multidisciplinaire et internationale de la bioinformatique, mais aussi de comprendre les origines sur lesquelles reposent nos connaissances. Elle permet également de procurer une vision claire du cadre dans lequel s'inscrit ma thèse, ainsi que son esprit novateur, pour un public non expert en bioinformatique et génomique des microorganismes.
\newline

Ce manuscrit sera divisé en plusieurs parties comme suit. Une première partie sera consacrée à contextualiser et à rendre compte des problématiques auquel répond ce travail de thèse. Dans cette partie, je reviendrai sur la définition précise des termes que j'utiliserai et je reviendrais sur l'état de l'art en génomique comparé des procaryotes. Je poursuivrai par une seconde partie sur les développements méthodologique que j'ai pu apporter en pangénomique, notamment dans la suite logicielle PPanGGOLiN. La troisième partie sera consacrée au c{\oe}ur de mon sujet de thèse, c.-à-d., aux développements de méthode pour la comparaison de pangénome. Enfin, je présenterai une nouvelle approche utilisant les bases de données orientées graphe comme solution pour le stockage et l'étude des pangénomes. Pour terminer, je présenterai une discussion critique sur le travail réalisé pendant ces trois ans. Je me dois également de rappeler aux lecteurs que la nature est extraordinairement diverse et que certaines règles et affirmations généralement vérifiées peuvent connaitre leurs exceptions.
\newline

Je vous souhaite bonne lecture de ce manuscrit, qui, je souhaite, fait preuve de toute la rigueur scientifique attendu et rend compte du travail réalisé pendant ces trois ans de manière authentique.
