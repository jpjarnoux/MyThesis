\part{Génomique comparée des procaryotes}
\chapter{Cellule procaryote : structure, physiologie et phylogénie}

La classification du vivant est encore marqué de nombreux débats. Dans ce chapitre, nous nous baserons sur la classification communément adoptée, c.-à-d., une division des êtres vivants en trois domaines : bactérie, archées et eucaryote. Dans cette vision de la classification, les virus ne sont pas intégrés, étant donné que leur appartenance au vivant est toujours débattue.

\section{Caractéristique phénotypique et structure de la cellule procaryote}

Les premières classifications des microorganismes se sont appuyées sur des critères phénotypiques. Bien que ces premières tentatives aient été limitées par la petite taille des organismes et les technologies de l'époque, elles ont permis de distinguer plusieurs grands groupes.

Pour commencer, certains microorganismes sont pluricellulaires, comme les champignons du genre \textit{Penicillium}, tandis que d'autres, tels que la bactérie \textit{Escherichia coli}, ne sont constitués que d'une seule cellule et sont qualifiés d'unicellulaires. Dans la suite, nous nous concentrerons exclusivement sur les organismes unicellulaires. La première distinction majeure qui a été établie pour diviser le vivant en deux grands domaines repose sur la présence ou l'absence de noyau. Le noyau est une structure interne de la cellule qui va contenir l'ensemble du matériel génétique. Les organismes (unicellulaire ou non) qui ont un noyau sont qualifiés d'eucaryote. Pour ceux dont le matériel génétique est librement dispersé dans le cytoplasme, ils sont catégorisés dans le domaine des procaryotes. Ce sont ces derniers qui vont nous intéresser, et, sauf précision ou volontaire imprécision, ce qui sera dit s'appliquera uniquement aux procaryotes.


L'avènement de la biologie moléculaire a permis d'affiner et de corriger les classifications précédentes en comparant : la structure moléculaire des parois de la cellule, la composition des molécules présente dans le cytoplasme et les séquences d'ADN entre les microorganismes. C'est notamment en étudiants les gènes codant l'ARN 16S, qu'il a été mis en évidence que l'ensemble des procaryotes ne formait pas un groupe monophylétique, mais qu'ils étaient séparés en deux domaines, Bactérie et Archée. Longtemps considéré comme un type de bactérie extrémophile, il est aujourd'hui clair que les archées sont un domaine à part entière avec toute sa singularité. Malgré toute la fascination que nous pouvons avoir pour les archées, et que toutes les méthodes qui seront présentées peuvent s'appliquer aux espèces Archée, nous ne présenterons que très peu de résultats les concernant. C'est pourquoi dans la suite, ce qui sera dit concernera uniquement le domaine des bactéries, mais pourra être étendu aux archées.

\section{Anatomie de la cellule bactérienne}

Pour continuer à classifier les bactéries, nous pouvons étudier les structures qui composent la cellule. Parmi ces structures, nous retrouvons des structures essentielles, commune à toutes les bactéries. Nous retrouvons tout d'abord autour de la cellule une \textbf{paroi cellulaire} composé de peptidoglycane. Chez certaines bactéries, cette couche va être plus épaisse et

1. Membrane plasmique
Description : Une double couche de phospholipides qui entoure la cellule.
Fonction : Elle régule les échanges entre l'intérieur de la cellule et son environnement externe, permettant le passage de nutriments et d'ions tout en rejetant les déchets.
2. Paroi cellulaire
Description : Structure rigide entourant la membrane plasmique, composée principalement de peptidoglycane chez les bactéries Gram-positives et d'une couche plus fine chez les Gram-négatives.
Fonction : Fournit une protection mécanique à la cellule, maintient sa forme et prévient l'éclatement en cas de variations de pression osmotique.
3. Cytoplasme
Description : Substance gélatineuse contenant de l'eau, des sels, des nutriments, des enzymes et d'autres molécules nécessaires aux activités cellulaires.
Fonction : Il est le lieu où se déroulent les réactions métaboliques et où sont contenus les organites et les molécules.
4. Ribosomes
Description : Petites structures dispersées dans le cytoplasme, constituées d'ARN ribosomique et de protéines.
Fonction : Les ribosomes sont responsables de la synthèse des protéines en traduisant l'information génétique contenue dans l'ARN messager.
5. Nucléoïde
Description : Région du cytoplasme où est localisé l'ADN chromosomique circulaire, sans membrane nucléaire.
Fonction : Contient le matériel génétique de la bactérie, qui code pour les protéines et les fonctions cellulaires essentielles.
6. Plasmides
Description : Petits fragments d'ADN circulaire présents en plus du chromosome principal.
Fonction : Ils codent souvent pour des gènes non essentiels à la survie de la cellule, mais pouvant conférer des avantages, comme la résistance aux antibiotiques.
7. Flagelle (présent chez certaines bactéries)
Description : Longue structure filamenteuse en hélice qui s'étend à l'extérieur de la cellule.
Fonction : Assure la motilité de la bactérie, lui permettant de se déplacer dans son environnement en réponse à des stimuli.
8. Pili ou fimbriae
Description : Petits filaments courts et fins qui recouvrent la surface de certaines bactéries.
Fonction : Impliqués dans l'adhésion aux surfaces, à d'autres cellules bactériennes ou à des cellules hôtes, et peuvent jouer un rôle dans la conjugaison (échange de matériel génétique entre bactéries).
9. Capsule (présente chez certaines bactéries)
Description : Couche externe de polysaccharides ou de protéines entourant la paroi cellulaire.
Fonction : Protège la bactérie contre la phagocytose, aide à l'adhérence à des surfaces et peut être un facteur de virulence.
10. Endospore (chez certaines bactéries, comme les Bacillus et Clostridium)
Description : Structure dormante et résistante formée à l'intérieur de la cellule en réponse à des conditions environnementales défavorables.
Fonction : Permet à la bactérie de survivre dans des conditions extrêmes (chaleur, dessiccation, radiation).

\section{Physiologie de la cellule bactérienne}

\chapter{Génomique des procaryotes}
\section{Organisation et structure des génomes}
\section{Dynamique évolutive des génomes procaryotes}
\section{Du génome au processus cellulaire}
\subsection{Métabolisme}
\subsection{système anti-viraux chez les procaryotes}
\chapter{La génomique des procaryotes à l'ère du big data}
\section{Base de données génomique}
\section{Études comparées de génomes}
\section{Pangénomique: état des lieux, enjeux et ambitions}