\part{Procaryotes : de la biologie cellulaire à la génomique moderne}

Ce chapitre marque le début du manuscrit et posera les bases conceptuelles et méthodologiques, biologiques et bioinformatiques essentielles à la compréhension des travaux menés dans le cadre de cette thèse. Il s’agit ici de contextualiser les enjeux de la génomique des procaryotes et de la pangénomique, tout en abordant les principaux concepts et méthodes utilisés dans ce domaine.

Nous commencerons par un rapide retour sur ce qu'est un procaryote, un élément clé pour définir les bornes et le contexte d’application de nos recherches. Cette partie est essentielle pour comprendre les spécificités de ces organismes et leurs impacts sur les approches méthodologiques adoptées. Cette introduction permettra également de les situer dans la classification du vivant, notamment en revenant sur la structure cellulaire et l'organisation du génome, tout en apportant les éléments pour discuter de la notion d'espèce procaryote.

Une fois ce cadre biologique posé, nous aborderons les bases de la génomique comparée, en se focalisant sur l'application aux procaryotes. Ce moment sera l’occasion de clarifier l’utilisation de simplifications ou de choix algorithmiques, souvent nécessaires en raison des caractéristiques propres à ces génomes. Ces éléments permettront de mieux comprendre l’approche bioinformatique qui sous-tend la comparaison des génomes, et ce, de la comparaison de séquences en allant jusqu'à l'approche par graphe, en passant par les modèles statistiques et les méthodes d'intelligence artificielle.

Le chapitre poursuivra en contextualisant la pangénomique, un domaine en pleine expansion qui permet de saisir la diversité génétique des populations microbiennes et qui est le centre des travaux de recherche ici réalisés. Nous mettrons en lumière l’évolution des données biologiques, tant sur le plan quantitatif que qualitatif, et soulignerons les défis posés par la gestion et l’analyse de ces données, en particulier dans le cadre de leur représentation, pour conclure par la manière dont la pangénomique a pu répondre à ces difficultés.

À la fin de ce chapitre, le lecteur aura tous les éléments théoriques et méthodologiques pour aborder les travaux de recherche développés, tout en disposant du cadre dans lequel s'inscrit la thèse et des enjeux actuels de la génomique des procaryotes et de la pangénomique.

\chapter{Caractérisation et classification des procaryotes : de la cellule au génome}

Avant d'étudier la génomique des procaryotes, il convient de revenir sur ce qu'est un procaryote et comment le placer dans l'arbre du vivant. La classification du vivant est encore marquée de nombreux débats et donc elle est en perpétuel changement \cite{chun_integrating_2014}\cite{adl_revisions_2019}. De plus, il faut prendre en compte comment regrouper les individus en groupes (appelé taxon), c'est la taxonomie, mais aussi comment reconstruire les relations évolutives reliant les individus entre eux, c'est la systématique. Nous nous baserons sur la classification communément adoptée, c.-à-d., une division des êtres vivants en trois domaines : Bactéries, Archées et Eucaryotes\footnote{Dans cette vision de la classification, les virus ne sont pas intégrés, étant donné que leur appartenance au vivant est toujours débattue.}. Cette classification permet (dans de nombreux cas) de concilier une classification des espèces selon des critères phénotypiques et des critères génomiques.

\section{La classification des microbes : des critères phénotypiques à la biologie moléculaire}

Les premières classifications des microorganismes se sont appuyées sur des critères phénotypiques, c.-à-d., des caractéristiques observables. Bien que ces premières tentatives aient été limitées par la petite taille des organismes et les technologies, elles ont permis de distinguer plusieurs grands groupes.

Pour commencer, certains microorganismes sont pluricellulaires, comme les champignons du genre \textit{Penicillium}, tandis que d'autres, tels que la bactérie \textit{Escherichia coli}, ne sont constitués que d'une seule cellule et sont qualifiés d'unicellulaires. Dans la suite, nous nous concentrerons exclusivement sur les organismes unicellulaires\footnote{certains procaryotes montrent des formes de coopération et de différenciation cellulaire, suggérant une forme de multicellularité primitive. Cependant, elles ne sont pas multicellulaires au sens strict, car leurs cellules restent indépendantes sur le plan fonctionnel et structurel.}. 
La première distinction majeure qui a été établie pour diviser le vivant en deux grands domaines repose sur la présence ou l'absence de noyau. Le noyau est une structure interne de la cellule qui va contenir l'ensemble du matériel génétique. Les organismes (unicellulaires ou non) qui ont un noyau sont qualifiés d'eucaryotes. Pour ceux dont le matériel génétique est librement dispersé dans le cytoplasme, ils sont catégorisés dans le domaine des procaryotes. Ce sont ces derniers qui vont nous intéresser, et, sauf précision, ce qui sera dit s'appliquera à tous les procaryotes.

Le développement de la biologie moléculaire a permis d'affiner et de corriger les classifications précédentes en analysant la morphologie, la physiologie et la biochimie des cellules procaryotes, ainsi que les séquences d'ADN des génomes. C'est notamment en étudiant les gènes codant l'ARN 16S, qu'il a été mis en évidence que l'ensemble des procaryotes ne formait pas un groupe monophylétique, mais qu'ils étaient séparés en deux domaines, Bactérie et Archée\cite{woese_phylogenetic_1977}. Longtemps considéré comme des bactéries extrêmophiles, il est aujourd'hui clair que les Archées représentent un domaine à part entière avec toute sa singularité, comme la composition de leur membrane par exemple \cite{albers_archaeal_2011}. Malgré toute la fascination que nous pouvons avoir pour les archées, et que toutes les méthodes qui seront présentées peuvent s'appliquer aux espèces Archée, nous ne présenterons que très peu de résultats les concernant. C'est pourquoi dans la suite, même si nous parlerons de procaryote, nous considérerons plutôt le domaine des bactéries avec un prolongement possible aux archées.

\section{Taxonomie des procaryotes : un problème non résolu ?}

La classification des procaryotes et la définition d'espèce procaryote ne fait pas consensus dans la communauté des microbiologistes. Toutefois, les méthodes de classification se basent sur le même principe de relation entre les individus \cite{aldhebiani_species_2018}. Ces relations peuvent être soit phénétique, c.-à.-d, reposant sur la similarité d'un trait, sans s'intéresser au lien évolutif qui pourrait les relier, soit phylogénétique, c.-à.-d, reposant sur l'hérédité du caractère indépendamment de son état actuel.

Les premières tentatives de classification des bactéries reposaient sur des approches phénétiques, utilisant des critères basés sur les caractéristiques observables de ces organismes. Ces classifications s'appuyaient sur des caractères morphologiques, physiologiques et biochimiques.
D'un point de vue morphologique, les microbiologistes examinaient des paramètres tels que la taille des cellules, leur mode de croissance et leur capacité à former des agrégats spécifiques. La présence ou l'absence de structures spécialisées, telles que les flagelles, était également un critère de différenciation. 
Les caractéristiques physiologiques permettaient, quant à elles, de classer les bactéries selon leur mode de vie, leurs mécanismes métaboliques (anabolisme et catabolisme) et leurs réponses aux conditions environnementales.
L'étude de la composition cellulaire offrait par ailleurs de nouveaux outils pour affiner ces classifications sur le plan biochimique. Par exemple, la coloration de Gram, méthode emblématique, permet de différencier les bactéries en deux grands groupes : les Gram-positives, caractérisées par une paroi épaisse de peptidoglycane, et les Gram-négatives, qui présentent une paroi plus fine associée à une membrane externe lipidique.
Enfin, selon le contexte d’étude, d'autres critères peuvent être intégrés. Dans le domaine médical, la pathogénicité (capacité à induire une maladie) et le sérogroupage (basé sur la composition antigénique de la capsule bactérienne) sont particulièrement utilisés pour identifier et classifier les bactéries d'intérêt clinique.

Avec l'arrivée de la génomique, du séquençage et de la bioinformatique, ces classifications ont peu à peu laissé leur place à des classifications basées sur la phylogénie. Néanmoins, l'ADN a aussi été utilisé comme un critère phénétique définissant des critères biochimique comme similarité entre les souches. 
Dans ces critères, il y a d'abord le pourcentage de guanine-cytosine (GC) qui permet de différencier 2 souches appartenant à 2 genres différents si elles possèdent plus de 10 mol \%\footnote{1 équivalent molaire équivaut à 100 \% en mole, donc 10 \% en mole équivaut à 0,1 équivalent molaire.}, mais il faut noter qu'une composition en GC proche n'implique pas forcément que les souches soient proches. 
Une approche visant à définir formellement une espèce procaryote a été adoptée en 1987 par un comité d'expert \cite{moore_report_1987}. Il propose que des souches appartiennent à une même espèce si l'ADN s'hybride\footnote{Appariement de 2 brins d'ADN par complémentarité des bases} a plus de 70 \% et que le $\Delta T_m$\footnote{température à laquelle la moitié de l'ADN est dénaturés} diffère de 5 degrés ou moins.

Toutes ces approches ont permis de classer les procaryotes en taxon et dans la nomenclature de la taxonomie actuel, il reste des traces de ces méthodes. Elles sont d'ailleurs toujours utilisées et font partie des traits visible dans les classifications. Il faut d'ailleurs souligner qu'il n'est pas toujours possible d'obtenir des génomes de bonne qualité pour réaliser des phylogénies.

\section{Espèce procaryote : le génome complet et la phylogénie peuvent-ils trancher ?}

Les approches phénétiques présentées précédemment ont l'intérêt de s'appliquer directement au laboratoire et donc de regrouper et d'identifier les souches rapidement. Néanmoins, elles restent relativement approximatives et sont parfois coûteuses (en temps et en moyens). De plus, même si elles répondent aux problèmes de la taxonomie, et donc de ranger les bactéries dans des taxons, elles ne répondent pas à la question du lien entre les différents taxons et comment représenter ce lien, c.-à-d., à la question de la systématique.

Pour pallier les limites des approches précédentes, une nouvelle méthode a été développée et reste encore largement utilisée en routine aujourd'hui : la comparaison des souches à partir d'un gène marqueur. Il s'agit d'un gène présentant des variations spécifiques parmi les différentes souches d'intérêt, toutes dérivant d'une forme ancestrale commune ayant évolué différemment au fil du temps. Ainsi, le gène marqueur reflète à la fois la similarité entre les souches, permettant leur regroupement, et les événements dits de spéciation ayant conduit à leur séparation en espèces distinctes. On va privilégier l'utilisation de gènes hautement exprimés qui assurent une fonction essentielle à la vie de l'organisme : les gènes de ménage (\textit{house-keeping genes}). Un gène marqueur en particulier est utilisé : l'ADNr 16S, qui a la particularité d'être présent chez tous les procaryotes. En 2007, un arbre du vivant de toutes les espèces, a été reconstruit à partir d'un arbre d'ADNr 16S comprenant toutes les souches types séquencées d'espèces de bactéries et d'archées publiées jusqu'à la fin de l'année 2007 \cite{yarza_all-species_2008}. 
En allant encore plus loin, des analyses \textit{multilocus sequence analysis} MLSA ont été proposés \cite{glaeser_multilocus_2015}. Ces analyses prennent en compte plusieurs gènes marqueurs pour réaliser la taxonomie. L'utilisation de plusieurs gènes augmente le niveau d'information et réduit les biais. Toutefois, il n'y a pas de recommandation universelle pour réaliser l'analyse et chaque MLSA est réalisé en fonction des souches de départ. La sélection des gènes et leur nombre sont des paramètres qui ont un impact encore peu évalué sur la taxonomie. Il en va de même pour la taille des fragments considérés pour chaque gène, qui ne représente qu'une partie de la séquence du gène. Enfin, expérimentalement, il est souvent difficile, voire impossible, de concevoir des amorces facilitant l'amplification des gènes dans toutes les souches prises en compte. Malgré ces critiques, l'utilisation de gènes marqueurs est encore aujourd'hui utilisée, mais est peu à peu remplacée par des méthodes prenant en compte l'ensemble du génome.


Au début des années 2000 et avec les nombreux projets autour du séquençage et de l'analyse des génomes, comme le projet génome humain \cite{lander_initial_2001}, les technologies de séquençage sont de plus en plus précises et de moins en moins couteuse, amenant dans la génomique "moderne" : une augmentation exponentielle du nombre de séquences et des séquences plus longues et de meilleure qualité. C'est l'arrivée du \textit{Whole Genome Sequencing} (WGS) et de l'analyse de génomes complets de procaryote. En réalité, les premiers génomes complets ont été séquencés et assemblés il y a longtemps (1995), mais pour les utiliser en génomique comparée et en phylogénie, il fallait aussi que les technologies et les algorithmes bioinformatiques se développent à leur tour, c'est pourquoi les méthodes présentées précédemment étaient privilégiées.

Grâce aux nouvelles méthodes de génomique comparée, que nous présenterons dans le chapitre suivant, il est désormais possible de considérer le génome complet pour faire l'assignation taxonomique d'une bactérie. Une de ces approches est l'\textit{Average Nucleotide Identity} (ANI), qui rend compte de la similarité entre 2 séquences nucléotidiques. Le score d'ANI va d'ailleurs remplacer celui de l'hybridation, où un ANI inférieur 95 \% permet de différencier les espèces à la place d'une hybridation à 70 \% \cite{goris_dnadna_2007}. Plus récemment, le seuil de 95 \% a été confirmé par les auteurs de FastANI \cite{jain_high_2018}, utilisant plus de 90000 génomes. Ils ont montré l'existence d'un \textit{gap}, espace où l'ANI diminue fortement avant 95 \% (\autoref{fig:ANI_gap_sp}).


\begin{figure}[htbp]
    \centering
    % Première image
    \subfloat{%
        \includegraphics[width=0.48\textwidth]{images/ANI_gap.jpg}
    }
    \hfill % Espace flexible entre les deux images
    % Deuxième image
    \subfloat{%
        \includegraphics[width=0.48\textwidth]{images/ANI_sp.jpg}
    }
    \caption[Variation du score d'ANI au niveau de l'espèce]{Variation du score d'ANI au niveau de l'espèce. (A-B) Les histogrammes sont basés sur des comparaisons par paire effectuées avec FastANI. (A) Le Score d'ANI représenté au niveau de l'espèce se base sur les données de Jain \textit{et al}. On y retrouve un \textit{gap} entre 84 et 95 \% d'ANI. (B) Score d'ANI représenté au niveau intra-espèce sur les données de Rodrigues-R \textit{et al}. On retrouve un \textit{gap} entre 99,2 et 99,8 \% d'ANI. (C-D) Score d'ANI au niveau du groupe \textit{Escherichia coli}. Le nombre de génomes utilisés est le suivant : \textit{E. coli} : 2815 ; \textit{Salmonella enterica} : 1351 ; \textit{Escherichia fergusonii} : 57 ; \textit{Escherichia albertii} : 70 ; et \textit{Shigella flexneri} : 93 (tous les génomes complets disponibles au NCBI en juillet 2023). (C) Comparaison de l'ANI entre \textit{E.Coli} et d'autres espèces. Le seuil de 95 \% délimitant l'espèce est retrouvé. Un \textit{gap} à 97 \% existe entre \textit{E.coli} et \textit{Shigella flexneri} (une espèce d'\textit{E.Coli} particulière pour ces propriétés infectieuse). (D) Analyse de l'ANI au sein des génomes de \textit{E.Coli}. L'écart d'ANI de 99,5 \% est aussi prononcé, par rapport aux barres adjacentes, que l'écart d'ANI de 98 \%-97 \% qui correspond à l'écart entre les phylogroupes d'\textit{E. coli}, un groupe distinct et bien reconnu au sein d'\textit{E. coli}. Figures et légende adaptées de \cite{konstantinidis_sequence-discrete_2023}}
    \label{fig:ANI_gap_sp}
\end{figure}


Pourtant, la communauté n'est toujours pas arrivée à un consensus sur la classification des procaryotes en espèces et même sur l'existence d'espèces procaryotes. On peut d'abord critiquer l'approche et les résultats des études utilisant l'ANI, qui se limitent aux génomes de bonnes qualité et complets, ce qui \textit{de facto} limite le nombre de génomes et d'espèces potentielles pris en compte, tout en augmentant la redondance et limitant la diversité et la variabilité. De plus, la démarche apporte le biais d'utiliser une taxonomie déjà existante. Il faut aussi prendre en compte que la dynamique évolutive des procaryotes, que nous détaillerons dans le chapitre suivant (\autoref{sec:dyn_evo}), n'est pas linéaire et héréditaire, mais que les procaryotes sont capables de recevoir et d'échanger de l'ADN. C'est pourquoi des auteurs soutiennent une définition plus écologique de l'espèce bactérienne \cite{luo_genome_2011}, prenant en compte ces échanges agissant sur le \textit{fitness} des bactéries dans leur environnement.


On peut donc convenir qu'il n'est pas encore communément admis de parler d'espèce procaryote. Il existe toutefois des caractéristiques communes et spécifiques aux procaryotes ainsi que des traits propres à chaque taxon. De nombreuses méthodes et démarches scientifiques parviennent à construire une phylogénie des procaryotes, mais celle-ci doit être replacée dans son contexte d'étude pour prendre sens. Notamment en pangénomique, on étudie régulièrement le pangénome d'une espèce, il est donc nécessaire de se baser sur une classification des génomes en espèce. Dans le contexte de nos travaux, la similarité des séquences l'emporte comme critère de classification, nous utiliserons donc des génomes provenant de bases de données utilisant des critères comme l'ANI ou des gènes marqueurs pour construire des pangénomes.

%Enfin, avec l'explosion du nombre de séquences disponible, nous voyons l'émergence d'un paradoxe : de plus en plus de données sont disponibles, mais alors que l'on pensait pouvoir ranger les procaryotes dans des boites bien précises qui se verraient valider au cours du temps, une nouvelle exception vient renverser l'ordre actuel et la phylogénie doit être revue.

\chapter{Génomique des procaryotes : organisation, évolution et fonctions}

Avant d'aborder la génomique comparée des procaryotes, il convient de revenir sur ce qu'est un génome procaryote. Les génomes procaryotes sont souvent décrits comme plus simple et plus facile à étudier que les génomes eucaryotes. Pourtant, sous cette simplicité apparente, il reste encore de nombreuses parts d'ombre sur l'organisation et la régulation des génomes procaryotes. Quant à la dynamique évolutive de ces génomes, nous avons vu qu'elle pose encore de nombreux problèmes aux spécialistes de la phylogénie. Enfin, les procaryotes sont toujours autant étudiés, car ce sont des réservoirs d'enzyme et processus chimique qui peuvent être utilisés dans de nombreux domaines. Des molécules et des réactions qui nous sont parfois encore inconnu et que nous sommes incapables de reproduire. Dans cette partie, je décrirais les mécanismes les plus connus et les plus répandus qui seront également des principes fondamentaux de nos hypothèses de développement méthodologique et d'analyse pangénomique. Je laisserai donc à chacun se faire une idée de la simplicité des génomes procaryotes. 

\section{Structure et organisation des génomes procaryotes}

Avant de décrire le génome, revenons rapidement sur sa définition. Le génome, c'est l'ensemble du matériel génétique, c.-à-d., des éléments qui seront hérités par les cellules de la génération suivante. Le génome, c'est aussi la structure de base qui va contenir l'ensemble des informations nécessaires au fonctionnement et à la survie de la cellule. Ces informations sont contenues dans la molécule d'ADN, ce qui nous amène à la structure primaire du génome, la séquence nucléotidique. Cette séquence est souvent circulaire chez les procaryotes et est de petite taille, quelques centaines de milliers de bases, mais certains génomes peuvent atteindre plusieurs millions de bases\footnote{En bioinformatique, on utilise l'unité base (b) ou paire de base (pb), pour mesurer la taille d'un génome. Un génome procaryote sera donc compris entre 100 kb et 10 Mb. Pour comparaison, le génome humain mesure environs 3 Gb.} (\autoref{fig:genome_size}).


\begin{figure}[htbp]
    \centering
    \includegraphics[width=\linewidth]{images/genome_size.png}
    \caption[Tailles des génomes pour différents groupes taxonomiques]{Variation de la taille des génomes (en paire de base) pour différents groupes taxonomiques. Copié de \cite{milo_cell_2015}}
    \label{fig:genome_size}
\end{figure}

Le génome est divisé en sous-unité que l'on appelle gène. Le gène contient l'information nécessaire pour produire une protéine qui réalisera une fonction dans la cellule (\autoref{fig:gene2prod}). Ces protéines correspondent à une chaîne d'acide aminé, que l'on peut représenter sous forme de séquence. Pour passer d'un gène à une protéine, on utilise une table de correspondance que l'on appelle code génétique où 3 nucléotides correspondent à 1 acide aminé. En moyenne, une protéine contient 300 acides aminés, ramenant la taille des gènes à environs 1 kb. Enfin, comme indiqué sur la partie haute de la \autoref{fig:genome_size}, les génomes procaryotes sont majoritairement codants, ce qui veut dire que presque tous l'ADN peut être divisé en gènes, et donc qu'il y a environs entre 100 et 10 000 gènes dans les génomes en fonction de leur taille. En mettant toutes ces informations en perspective, la petite taille des génomes procaryotes est compensé par son fort taux de gènes, ainsi, il contient l'ensemble des protéines nécessaires à la survie de la cellule. 


\begin{figure}[htbp]
    \centering
    \includegraphics[width=\linewidth]{images/gene2prot.jpg}
    \caption[Produit d'un gène]{Produit d'un gène dans la cellule. Un gène est d'abord transcrit en ARN. Si l'ARN transcrit est dit messager (ARNm), il sera ensuite traduit en protéine, sinon l'ARN produit (ARNt, ARNr, miARN, ....) aura un rôle spécifique dans des processus cellulaire. Copié de RNBio, Sorbonne université. \url{https://rnbio.sorbonne-universite.fr/genetique_genotype1}}
    \label{fig:gene2prod}
\end{figure}


Dans la cellule, l'ADN ne reste pas sous cette forme primaire de séquence, il va se replier par différent mécanisme pour arriver dans une forme plus compacte qu'on appelle le chromosome (\autoref{fig:structure_dna}). L'ADN commence par se replier dans une structure secondaire, notamment la célèbre double hélice décrite par Watson, Crick et Franklin \cite{watson_molecular_1953}\footnote{Ces travaux sont souvent cités comme exemple dans la lutte pour la reconnaissance des femmes en sciences, Rosalind Franklin ayant joué un rôle essentiel, mais souvent sous-estimé dans cette découverte.}. Bien que la double hélice soit la forme la plus connue, d'autres conformations secondaires, telles que les structures en triple hélice ou en Z, ont également été identifiées, comme illustré dans la \autoref{fig:structure_dna}. L'organisation de l'ADN va au-delà de cette structure secondaire : il est ensuite soumis à des mécanismes de superenroulement induits par des enzymes spécifiques comme les topoisomérases et les gyrases. Ce superenroulement permet de réduire davantage la taille de l'ADN et de favoriser son organisation en boucles maintenues par des protéines structurales telles que HU, IHF ou H-NS \cite{williams_molecular_1997,prieto_genomic_2012}. Pour terminer des protéines appelé histones vont terminer de replier l'ADN en formant des nucléosomes, les procaryotes utilisent ces protéines pour compacter leur ADN en une structure appelée nucléoïde. Cette forme, au-delà d'optimiser l'espace dans la cellule, permet aussi de stabiliser et de protéger l'ADN, ainsi que la régulation de l’expression des gènes. Par exemple, la méthylation de l’ADN, ainsi que les modifications des protéines associées, sont des mécanismes clés de l’épigénétique. Ces processus influencent la transcription des gènes et ont des implications fonctionnelles majeures. Des études récentes ont mis en lumière le rôle de la méthylation dans la régulation de la virulence bactérienne et dans la capacité des procaryotes à coloniser leurs hôtes \cite{oliveira_bacterial_2021}, soulignant ainsi l'importance de ces mécanismes dans la survie et l’adaptation des bactéries.

\begin{figure}[htbp]
    \centering
    \includegraphics[width=0.8\linewidth]{images/structureDNA.jpg}
    \caption[Structure de l'ADN]{Représentation de la structure primaire, secondaire, tertiaire et quaternaire d'un acide nucléique. PDB ID : 1EQZ et 4R4V. Tiré de \cite{kumar_biomolecular_2019}}
    \label{fig:structure_dna}
\end{figure}

Un génome procaryote est donc en résumé un génome de petite taille, souvent circulaire et majoritairement codant. Il est donc essentiel de comprendre que la moindre modification dans la séquence d'ADN peut amener soit à un changement dans la séquence protéique, et donc son incapacité à fonctionner correctement, soit à l'impossibilité de produire la protéine. Une vision plus positive sera aussi d'imaginer que des changements dans la séquence d'ADN permettrons de produire une nouvelle protéine d'intérêt pour la cellule. Dans la suite, avec une vision darwinienne\footnote{Vision de l'évolution proposée par Charles Darwin, qui propose que les espèces évolue perpétuellement de façon hasardeuse et que les innovations génétiques sont ensuite maintenues ou perdues dans les populations par pression de sélection}, nous verrons par quels mécanismes la séquence d'ADN va évoluer, mais aussi comment ces évolutions seront transmises aux autres cellules procaryotes. 

\section{Dynamique évolutive des génomes : mécanismes et impacts}
\label{sec:dyn_evo}

Lorsqu'on étudie l'évolution des génomes, on s'intéresse aux changements apportés à la séquence d'ADN de la cellule : les mutations. Les mutations peuvent induire soit un gain, une perte ou une modification de la séquence génétique en fonction du mécanisme sous-jacent. Ces mécanismes sont complexes et bien différents de ceux que l'on pourrait concevoir avec une vision anthropomorphique. En effet, les cellules procaryotes ne s'accouplent pas pour produire une nouvelle cellule. Dans la nature, les procaryotes vont se multiplier par division cellulaire où une cellule mère donnera 2 cellules filles possédant le même matériel génétique que la mère, moins les possibles changements que nous décrirons dans la \autoref{sec:evo_ver}. Lorsque l'ADN est hérité de la cellule mère par la cellule fille, on va parler de \textbf{transfert vertical}. Il existe également (toujours par anthropomorphisme) une forme de sexualité des procaryotes, où 2 cellules vont échanger du matériel génétique sans qu'une nouvelle cellule ne soit créée. Dans ce cas, l'ADN est échangé entre 2 cellules dites de la même génération et on parle de \textbf{transfert horizontal} (voir \autoref{sec:evo_hz}). 

Ces mécanismes présents dans la nature sont exploités en microbiologie et en biologie cellulaire pour introduire des changements de gènes spécifiques dans une cellule et ainsi obtenir des espèces chimériques hybrident qui pourront être utilisées dans la recherche ou l'industrie \cite{baby_chromosomes_2019}. On peut aussi penser aux cellules procaryotes vivant en symbiose, voire en endosymbiose\footnote{une bactérie réside à l'intérieur d'une autre cellule (procaryote ou eucaryote)}, qui pourrait être considéré comme une étape préliminaire à une "fusion" évolutive. Ce mécanisme serait d'ailleurs à l'origine d'organites comme la mitochondrie et le chloroplaste\cite{martin_endosymbiotic_2015}. La fusion de cellules procaryotes est d'ailleurs possible et réalisée en laboratoire en enlevant leur paroi cellulaire pour obtenir des protoplastes. Les protoplastes peuvent être fusionnés grâce à des agents chimiques (comme le polyéthylène glycol) ou des chocs électriques (électrofusion) \cite{schaeffer_fusion_1976}.

\subsection{Mécanismes d'évolution par héritage}
\label{sec:evo_ver}
Les mécanismes d'évolution par héritages regroupent les processus menant à une modification du génome entre la cellule mère et la cellule fille. Théoriquement, lors de la division cellulaire, la cellule mère se divise en 2 cellules filles possédant exactement la même information génétique qu'elle. Pourtant, malgré un ensemble de mécanisme de protection et de correction de l'ADN, le génome peut différer entre les cellules mère et filles. Ce sont ces "erreurs" qui vont nous intéresser, car ce sont elles qui sont à l'origine de l'innovation et de la diversité génétique.

\subsubsection{Mutation génétique : un petit changement aux grandes conséquence}
\paragraph{\textit{Single Nucleotid Polymorphism}}

Un \textit{Single Nucleotide Polymorphism} (SNP) est un mécanisme d'évolution qui induit une modification de la séquence par la transformation d'un nucléotide en un autre. Étant donné que le code génétique est dégénéré\footnote{Un acide aminé peut être codé par plusieurs codons différents.}, la mutation peut ne pas avoir d'impact sur la séquence de la protéine, on dit que la mutation est silencieuse ou même sens. Si la modification change la séquence protéique, dans ce cas, on parle de mutation faux-sens. Enfin, Une mutation est qualifiée de non-sens lorsqu'elle affecte un point clé de la séquence protéique, comme le site actif ou un codon STOP, entraînant une perte de fonction de la protéine, ou lorsqu'elle introduit prématurément un codon STOP dans la séquence.

\paragraph{Indels: insertion, délétion et pseudogènes}

Un indel correspond à l'insertion (In) ou la délétion (del)\footnote{On regroupe l'insertion et la délétion, car sans une analyse phylogénétique, il est impossible de les différencier par comparaison de séquence.} d'un ou plusieurs nucléotides dans la séquence d'un gène. 

Lorsque la taille de l'indel est un multiple de 3 (insertion ou délétion d'un codon), la séquence protéique peut soit être allongé ou raccourci d'un acide aminé, soit coupé de façon précoce si le codon est un codon STOP.

Si la taille de l'indel n'est pas un multiple de 3, il y aura un décalage du cadre de lecture ou \textit{frameshift}. Ce décalage va induire un changement de tous les acides aminés de l'indel à la fin du gène, provoquant avec lui un changement dans la fonction de la protéine ou une inactivation de la fonction. La partie du gène qui n'est pas décalé est alors considéré comme un fragment du gène initial, il est alors qualifié de pseudogène. À nouveau, cette mutation peut être délétère pour la cellule. 

Les indels vont donc transformer la séquence protéique traduite, pouvant nuire à la fonction de cette dernière et être délétère pour l'organisme. Pour éviter les problèmes liés au \textit{frameshift}, il a été montré qu'il existe un fort taux de codon STOP hors du cadre de lecture \cite{tse_natural_2010}. Cette adaptation permettrait de limiter la traduction des protéines mutante et d'ainsi limiter le coût énergétique pour la cellule. Il a aussi été montré que les \textit{frameshift} pourrait être à l'origine d'un réservoir d'adaptation à l'environnement \cite{koch_catastrophe_2004}. Lors d'un changement dans l'environnement créant une nouvelle pression de sélection, un \textit{frameshift} pourrait améliorer le fitness de certains organismes. Une fois que l'élément perturbateur de l'environnement disparait, un nouveau frameshift ramènerait le cadre de lecture à sa place d'origine. Ce mécanisme, en accord avec la petite taille des génomes, aurait l'intérêt de ne pas perdre des gènes d'adaptation à l'environnement, même s'ils ne sont nécessaires que ponctuellement.

\subsubsection{Réarrangement génomique : un moteur de l'évolution}
\paragraph{Réarrangement}
\paragraph{Recombinaison}
\paragraph{Duplication}

\subsection{Mécanismes d'évolution intragénérationnelle}
\label{sec:evo_hz}

\subsubsection{Conjugaison : la sexualité des procaryotes}

\subsubsection{Transformation : recycler l'ADN environnant}

\subsubsection{Transduction : un sacrifice pour le bien commun}

\subsection{Interprétation des évolutions : l'homologie et ses déclinaisons}


\section{Du génome aux processus cellulaires : exploration fonctionnelle}

\subsection{Gènes et fonctions}

\subsection{Îlots génomiques}

\chapter{Génomique comparées des procaryotes}
\section{Analyse comparative des génomes : méthodes et applications}
\label{sec:comp_gen}
\subsection{Comparaison des séquences}
\subsection{Statistique et séquence}
\subsection{Utilisation des graphes}
\subsection{Application de la génomique comparée pour l'étude des procaryotes}
\subsection{Intelligence artificielle : machine learning et deep learning}
\section{Système biologique}
\subsection{Définition et intérêt}
\subsection{Méthode de détections}
\subsection{Systèmes de défense aux phages}
\chapter{Pangénomique: état des lieux, enjeux et ambitions}

 La pangénomique est un domaine d'étude en plein essor, qui a permis d'explorer et d'analyser les génomes procaryotes sous un nouveau point de vue. Mon travail de thèse s'est concentré sur l'analyse et la comparaison de pangénomes. Dans cette partie, je reviendrai d'abord sur l'origine, les concepts et les défis que pose la pangénomique.  Je présenterai ensuite les différentes modélisations permettant de représenter les génomes en pangénomique, pour poursuivre sur les méthodes de construction de pangénome. Pour terminer, je développerai les méthodes d'analyse existantes en pangénomique. Cette partie sera aussi l'occasion de faire l'état de l'art des outils en pangénomique et de présenter l'outil PPanGGOLiN sur lequel j'ai pu travailler et que j'ai utilisé dans mes développements de thèse.

\section{Origine et concept}

Bien que le terme "pangénome" soit utilisé dans des articles avant 2005, en microbiologie, on s'accorde sur une origine du concept de pangénome proposé dans 2 articles fondateurs \cite{medini_microbial_2005,tettelin_genome_2005}.
L'idée est de ne pas représenter chaque génome individuellement, mais d'utiliser une structure mathématique permettant de les représenter tous simultanément.
Le pangénome représente l'union de toutes les séquences présentes dans un ensemble de génomes. En bioinformatique, la structure, les algorithmes, les méthodes d'analyses des pangénomes, ont constitué un nouveau champ de recherche, la pangénomique.

%\medskip

À partir du pangénome, Tettelin \textit{et al.} proposent de séparer les gènes en 2 catégories, les gènes "\textit{core}" communs à tous les génomes, des gènes "\textit{dispensable}" (ou \textit{accessory}) trouvés dans un sous-ensemble de génomes. En généralisant, le pangénome permet de distinguer l'ensemble des séquences communes à tous les organismes des variations présentes chez certains groupes d'individus, voire spécifiques à un organisme. De ce postulat a émergé l'idée de remplacer les génomes de référence dans les bases de données par des pangénomes de référence \cite{the_computational_pan-genomics_consortium_computational_2018}. Toutefois, ce changement de paradigme n'a pas encore été opéré, car aucune méthode n'a encore réussi à s'imposer comme solution optimale. Trouver une méthode globale est un défi, car la pangénomique est appliquée dans de nombreux domaines de recherche, pour répondre à une grande diversité de questions.

%\medskip

En 2018, le "Computational Pan-Genomics Consortium" met en avant le rôle de la pangénomique dans le développement de solutions applicatives répondant à des problématiques communes à plusieurs disciplines \cite{the_computational_pan-genomics_consortium_computational_2018}. En retour, la pangénomique bénéficie des avancées en phylogénie, métagénomique et intelligence artificielle.
En phylogénie, les méthodes de comparaison génomique à grande échelle et les techniques de construction d'arbres phylogénétiques ont été intégrées aux approches pangénomiques. Réciproquement, la pangénomique permet une meilleure prise en compte des variations génétiques à l'échelle de l'ensemble des génomes, plutôt que de se limiter à un génome de référence, offrant ainsi une vision plus fine de la dynamique évolutive \cite{bazinet_pan-genome_2017}.
Les données métagénomiques représentent un challenge pour la pangénomique. À partir des métagénomes, le pangénome doit être construit en étudiant les relations de co-occurrence des gènes, et non les relations évolutives. Ce changement représente un défi, notamment lorsque les lectures sont courtes. Toutefois, la pangénomique permet d'approfondir l’analyse de la diversité génétique des communautés microbiennes, et de mettre en évidence des adaptations communes à l'environnement ou des co-évolutions et des interactions entre les organismes \cite{the_computational_pan-genomics_consortium_computational_2018}.
L’intelligence artificielle joue également un rôle clé en améliorant l’annotation et la prédiction fonctionnelle des gènes. L’apprentissage automatique est appliqué à la pangénomique pour détecter des motifs génétiques pertinents, prédire des phénotypes et identifier des associations entre mutations et traits phénotypiques \cite{her_pan-genome-based_2018}. Ces méthodes, souvent développées pour d’autres disciplines, ont donc favorisé l’essor de la pangénomique en optimisant l’analyse des données, la reconstruction des génomes et l’interprétation des résultats.

%\medskip

La pangénomique représente une solution à l'analyse de grands volumes de données, à l'heure où le nombre de génomes disponibles dans les banques augmente de façon exponentielle. Entre 2006 et 2024, ce ne sont pas moins de 3 500 articles qui référencent le terme\footnote{Ce chiffre doit être revu à la baisse dû à l'utilisation erronée du terme dans certaines études et une utilisation parfois abusive pour profiter de l'intérêt croissant pour ces analyses}, dont près de 800 concernant les procaryotes (\autoref{fig:panCite}).

\begin{figure}[htbp]
    \centering
    \includegraphics[width=\linewidth]{images/pangenomeCitation.png}
    \caption[Bibliométrie pangénome]{\textbf{Nombre d'articles, référencés dans PubMed, par année, à propos de pangénome du 1er janvier 2004 au 10 février 2025}. La courbe bleue représente le nombre d'articles contenant le terme pangénome dans le titre ou l'abstract : Query=("pan-genome"[Title/Abstract] OR "pangenome"[Title/Abstract] OR "pan-genome"[Title/Abstract]) AND (2004:2025[pdat]). La courbe rouge limite aux publications concernant les procaryotes : Query=("procaryote"[Title/Abstract] OR "bacteria"[Title/Abstract] OR "archeae"[Title/Abstract]) AND ("pan-genome"[Title/Abstract] OR "pangenome"[Title/Abstract] OR "pan-genome"[Title/Abstract]) AND (2004:2025[pdat]). La courbe verte représente tous les articles ou le terme pangénome est trouvé : Query=((pangenome) OR (pan genome)) OR (pan-genome) AND (2004:2025[pdat]). La courbe violette filtre les publications concernant les procaryotes : Query=(((procaryote) OR (bacteria)) OR (archeae)) AND (((pan-genome) OR (pangenome)) OR (pan genome)) AND (2004:2025[pdat]).}
    \label{fig:panCite}
\end{figure}

\newpage
\subsection{Modélisation de la croissance des pangénomes}
\label{sec:croissance_pan}

Dans l'article original de Tettelin \textit{et al.} \cite{tettelin_genome_2005}, les auteurs se sont intéressés à la distribution \textit{core/dispensable} en fonction du nombre de génomes de \textit{Streptococcus agalactiae}\footnote{Bactérie du microbiote intestinale humain et animal, qui est également associé à des infections graves.} que contient le pangénome. Ils observent que lorsque le nombre de génomes augmente, la part de \textit{core genome} décroît de façon exponentielle. Ce résultat les amène à modéliser la croissance du \textit{core genomes} selon une équation exponentielle décroissante. Le modèle permet alors d'estimer la taille du \textit{core genome} pour un nombre de génomes en théorie infinie. Il est alors possible d'estimer la taille du \textit{core genome} d'une espèce à partir d'un échantillon de génome. 

À partir de ce modèle, il est également possible d'estimer la taille du pangénome, \textit{i.e.}, le nombre de gènes unique que contient le pangénome. Ils définissent alors 2 types de pangénomes en fonction de l'estimation de la taille : les \textbf{pangénomes ouverts} et les \textbf{pangénomes fermés}. Les pangénomes sont considérés comme ouverts lorsque l'on ajoute un génome, le nombre de gènes ajouté au pangénome augmentent. Le nombre de gènes est donc théoriquement infini pour un pangénome ouvert avec une infinité de génomes. Les pangénomes fermés quant à eux voient le nombre de nouveaux gènes progressivement diminuer lors de l'ajout de nouveaux génomes. La courbe de prédiction permet d'identifier un plateau théorique du nombre maximal de familles que contiendra le pangénome avec un nombre de génomes infinis. Biologiquement, le pangénome ouvert est attendu pour les espèces sympatriques\footnote{Espèces vivant dans le même environnement que d'autres espèces.} et qui présentent un fort taux de transferts horizontaux, tandis que les espèces vivant dans des niches écologiques ou qui ont une faible capacité d'acquisition de gènes extérieurs vont avoir un pangénome fermé.

\begin{figure}[htbp]
    \centering
    \includegraphics[width=0.85\linewidth]{images/panOpenClose.png}
    \caption[Schéma de croissance du pangénome]{\textbf{Schéma de croissance du pangénome.}}
    \label{fig:panOpenClose}
\end{figure}

Le modèle proposé par Tettelin \textit{et al.}, repose sur l'hypothèse que pour un nombre suffisant de génomes, le nombre de nouveaux gènes apportés par un génome devient constant à partir d'un certain nombre de génomes \cite{tettelin_genome_2005}. Cette hypothèse implique alors que la taille du pangénome est infinie. Cette hypothèse sera questionnée par Hogg \textit{et al.} dans leur étude du pangénome de \textit{Haemophilus influenzae} \cite{hogg_characterization_2007}. Ils vont alors proposer une modélisation basée sur l'hypothèse que le pangénome est fini. Dans leur modèle, chaque gène est associé à une variable aléatoire de Bernoulli, dont la probabilité correspond à la fréquence du gène dans les génomes. Un génome est ainsi généré en observant ces variables : un gène est présent si l’essai est un succès, sinon il est absent. Bien que certains gènes ne soient pas indépendants en raison de structures comme les îlots génomiques, cette hypothèse est conservée pour simplifier le modèle. Les fréquences réelles des gènes étant inconnues, elles sont modélisées de manière probabiliste en répartissant les gènes en $K$ classes distinctes, chacune ayant une fréquence de présence spécifique. À partir de ce modèle, sur le pangénome de \textit{H. influenzae} avec $K=7$, la taille du pangénome est estimée à 5 000 gènes (contre 2 800 gènes dans les 13 génomes de base). Ce modèle sera ensuite amélioré par Snipen \textit{et al.} \cite{snipen_microbial_2009}, qui proposeront une détermination automatique du nombre de classes $K$ et de la fréquence théorique des gènes pour chaque classe. Les modèles binomiaux proposent une perspective dans laquelle la diversité en gènes est finie et qu'il existe un nombre de génomes suffisamment grand pour que tout le répertoire génique soit connu. Cette vision semble de prime abord logique, car le nombre de combinaisons possibles de nucléotides ou d'acides aminés est fini. Pourtant, on peut y opposer que ce nombre, sans le calculer, semble démesuré et qu'il peut être considéré comme infini. De plus, les génomes évoluent continuellement et de nouveaux gènes apparaissent sans cesse. L'utilisation des modèles binomiaux semble alors plus appropriée à des espèces de niche, isolées et présentant un faible taux de transferts horizontaux.

En 2008, Tettelin \textit{et al.} vont proposer une nouvelle modélisation basée sur la loi de Heaps\footnote{Définit de manière empirique en linguistique, cette loi permet de décrire le nombre de mots d'une langue à partir d'un ensemble de documents.} \cite{tettelin_comparative_2008}. On estime le nombre $n$ de gènes distincts, en fonction du nombre $N$ de génomes étudiés, selon la relation :
\begin{equation}
    n=kN^\gamma, 0<\gamma<1,k\geq1
\end{equation}

Le paramètre $k$ est une constante de proportionnalité tandis que $\gamma$ reflète la tendance de la fonction. Ainsi, plus $\gamma$ est proche de 0 plus la croissance en gènes distincts est lente, et plus $\gamma$ est proche de 1 plus la croissance est rapide (\autoref{fig:HeaplawGamma}).

\begin{figure}[htbp]
    \centering
    \subfloat[Courbe de croissance selon la loi de Heap]{\includegraphics[width=0.48\linewidth]{images/HeapsLawgamma.png}
    \label{fig:HeaplawGamma}}
    \hfill
    \subfloat[Courbe de raréfaction selon la loi de Heap]{\includegraphics[width=0.48\linewidth]{images/HeapsLawAlpha.png}
    \label{fig:HeaplawAlpha}}
    \caption[Évolution du pangénome : visualisation de la croissance et de la raréfaction du contenue génique selon la loi de Heap]{\textbf{Évolution du pangénome : visualisation de la croissance et de la raréfaction du contenue génique selon la loi de Heap.}}
    \label{fig:Heaplaw}
\end{figure}

Selon la loi de Heap, le nombre de nouveaux gènes découverts diminue à mesure que l'on ajoute des génomes. On peut formuler ceci selon l'équation : 

\begin{equation}
    p(n)=kN^{(\gamma-1)}=kN^{-\alpha}, \alpha=1-\gamma
\end{equation}

Ainsi, sur la \autoref{fig:HeaplawAlpha}, lorsque $0<\alpha<1$, le taux de nouveaux gènes décroît en ajoutant des génomes, sans jamais être nul. Dans ce cas, le nombre de gènes distincts est croissant. Ce qui implique que si $0<\alpha<1$, le pangénome est ouvert. À partir d'un ensemble de génomes, il est possible d'estimer k et $\alpha$ (ou $\gamma$) et donc de caractériser si le pangénome est ouvert. Si $\alpha\geq1$, alors le taux de nouveaux gènes atteint 0, ce qui correspond à un pangénome fermé. 

\newpage

\subsection{Les différents types de pangénomes}

Les pangénomes peuvent être divisés en 2 catégories en fonction de l'unité choisie pour les construire. Le premier type, celui présenté par Tettelin \textit{et al.} \cite{tettelin_genome_2005}, utilise les gènes comme unité de base du pangénome (\autoref{fig:panType}.B). En regroupant les gènes par homologie (appelé famille de gènes, cf. \autoref{sec:clustering}), il est possible d'obtenir la présence/absence de gènes similaires dans les génomes. Ces pangénomes ont l'avantage d'être moins coûteux en ressources de calcul pour être construits. De plus, ils sont faciles à interpréter, car les gènes sont des unités déjà bien définies et parfois, ils sont même annotés fonctionnellement. Néanmoins, en utilisant les gènes, la méthode d'annotation a un impact important sur le pangénome et il est sensible aux erreurs d'annotation. De plus, les régions non codantes ne sont pas prises en compte dans cette approche. Enfin, les SNPs peuvent passer inaperçus après le regroupement, ainsi que les variants structuraux (SV).

L'autre type de pangénome est basé sur les séquences brutes des génomes. Bien que le terme pangénome n’ait pas encore été employé à l’époque, Chiapello \textit{et al.} \cite{chiapello_systematic_2005} ont proposé une méthode de segmentation des génomes en deux composantes : la "colonne vertébrale", représentant les régions conservées, et les "boucles", qui correspondent aux parties variables. Plus tard, l’outil GenomeMapper \cite{schneeberger_simultaneous_2009}, a explicitement introduit la notion de pangénome de séquence. Son approche repose sur un alignement global des séquences, analysées à travers des k-mers pour différencier les segments conservés des segments variables (\autoref{fig:panType}.C,D). Cette approche a l'intérêt de prendre en compte toute la diversité des génomes (codant, non codant, SNPs et SV). Toutefois, la construction de ces pangénomes est plus coûteuse en ressources. De plus, l'interprétation est plus complexe, car le pangénome n'est pas annoté au départ. Pour terminer, certaines méthodes de construction, utilisent un génome de référence comme séquence de base (\autoref{fig:panType}.C), ce qui peut aussi introduire un biais.

\begin{figure}[htbp]
    \centering
    \includegraphics[width=0.8\linewidth]{images/pangenomeTypes.jpeg}
    \caption[Différents types de pangénomes]{\textbf{Différents types de pangénomes.} Extrait de \cite{matthews_gentle_2024}}
    \label{fig:panType}
\end{figure}

\newpage
\section{Pangénome de séquences}

\subsection{Pangénome basé sur une séquence représentative}

Un pangénome basé sur les séquences correspond à un ensemble de génomes dont l'alignement minimise le nombre de régions homologues tout en rendant compte de toute la diversité. L'objectif derrière ces pangénomes est d'obtenir une séquence pangénomique de référence. De façon contre-intuitive (par rapport à la définition "sans-référence" des pangénomes), pour construire ces pangénomes, on utilise une séquence représentative comme base. Toutes les séquences seront alignées à partir de cette base, et les segments non redondants détectés dans au moins un génome seront intégrés à la référence non redondante (NRR, Non-Redundant Reference en anglais). L'ensemble, séquence représentante et NRRs, forme alors la séquence pangénomique de référence.

\subsubsection{Méthode de construction}

Pour construire ces pangénomes, il faut d'abord identifier une séquence représentative. Les autres séquences, en général des séquences non assemblées (lectures ou \textit{reads} en anglais), sont alignées contre la représentante et les séquences non alignables sont considérées comme des NRRs potentiels. Les NRRs de taille inférieure à 500 pb sont exclues, ainsi que celles dont la similarité avec la représentante est supérieure à un seuil (90 \% en général). Les NRRs restantes sont comparées à des bases de données pour retirer tous les contaminants potentiels. De ce schéma général, on peut identifier 4 méthodes différentes pour l'identification des NRRs potentiels :
\begin{itemize}
    \item \textbf{Assemblage de type métagénomique} : les lectures non alignées sur la référence sont regroupées et assemblées \textit{de novo}. Les contigs obtenus sont ajoutés à la séquence représentante. Cette méthode est efficace même avec une faible couverture des lectures.
    \item \textbf{Assemblage itératif} : Dans un premier temps, les lectures non alignées du premier échantillon sont assemblées et ajoutées au génome de référence. Ce génome mis à jour sert ensuite de base pour l’assemblage des échantillons suivants. Ce processus est répété pour tous les échantillons.
\end{itemize}
\newpage
\begin{itemize}
    \item \textbf{Assemblage indépendant des \textit{reads} non alignés} : Toutes les lectures non alignées sont séparées par échantillon\footnote{Ensemble de lectures obtenues simultanément} et assemblées \textit{de novo} indépendamment. Les contigs obtenus sont regroupés selon leur similarité. Dans chaque groupe, un contig référent est identifié et est intégré à la séquence référente. Cette méthode nécessite une couverture d’au moins 10×, pour obtenir des contigs de taille suffisante.
    \item \textbf{Assemblage génomique indépendant} : chaque échantillon est assemblé indépendamment, et les contigs obtenus sont alignés à la référence. Les contigs non alignés sont regroupés par similarité et un contig référent est ajouté à la séquence référente.
\end{itemize}

Le choix de la méthode dépend du type et de la quantité des données disponibles. Avec une faible couverture (<10×) et un grand nombre d’échantillons, l’approche métagénomique est recommandée, bien qu’elle puisse produire des contigs chimériques. Avec une couverture plus élevée (>10×), l’assemblage indépendant ou l’approche itérative sont préférables. Cette dernière est plus lente, mais facilite l’ajout de nouveaux échantillons. Enfin, si plusieurs assemblages de haute qualité existent déjà, l’assemblage génomique indépendant est la meilleure option. Ces méthodes peuvent être combinées pour optimiser l’utilisation des données disponibles.
 
\subsubsection{Domaines d'application des pangénomes basés sur une séquence représentative}

Ces pangénomes sont particulièrement utiles lorsque les données de départ sont des \textit{lectures}. En utilisant ces modèles, il est possible de revenir à une séquence linéaire qui peut être utilisée dans les outils classiques de génomique. De plus, il peut également être utilisé comme étape préliminaire à la construction d'autres types de pangénomes, en réduisant rapidement la redondance dans un sous-ensemble proche de génomes. 

L'outil NGSPanPipe \cite{kulsum_ngspanpipe_2018} est un pipeline intégré conçu pour l'identification du pangénome à partir de lectures courtes (short reads) issues du séquençage de nouvelle génération (NGS). Contrairement à d'autres méthodes nécessitant des prétraitements des lectures, NGSPanPipe permet une analyse directe des reads bruts pour identifier le pangénome. Il ne génère pas de séquence pangénomique linéaire, mais il permet de reconstruire des contigs à partir des lectures en utilisant un génome de référence. Les contigs obtenus à partir des lectures alignées, permettent de calculer la couverture du génome par rapport au pangénome. Les lectures non alignées sont comparées à des bases de données de \textit{reads} pour identifier de nouveaux \textit{reads}, puis ils sont assemblés en contigs. L'ensemble des contigs (de lectures alignées et non alignées) sont annotés et utilisés pour construire une matrice binaire représentant la présence ou l'absence des gènes dans la séquence de référence.

\subsection{Pangénome graphe}

Les graphes de séquences sont un modèle de pangénome permettant de visualiser la diversité génomique, qu’elle soit basée sur une séquence de référence ou non. Dans tous les cas, des segments de séquences vont constituer les n\oe uds du pangénome et les arêtes seront étiquetées par des informations permettant de retrouver le lien entre les segments (comme l'organisation dans les génomes). Ce modèle pangénomique a l'intérêt de représenter toute la diversité, codant et non codant. 

\subsubsection{Méthodes et outils de construction}

\paragraph{Graphe de variant prédéterminé}

La première méthode de construction des graphes de pangénome se base sur l'utilisation d'une séquence référente et d'un fichier contenant les variations connues dans les autres séquences par rapport à cette référence. Cette méthode a l'intérêt de demander peu de ressources, car les variations sont prédéterminées et données en entrée. Toutefois, pour obtenir un graphe fiable et précis, un génome complet de bonne qualité est requis.

L'outil VG (\textit{Variation Graph toolkit}) \cite{garrison_variation_2018}, contient un ensemble d'outils permettant de générer un graphe de variants. À partir de ce graphe, qui peut être assimilé à un graphe de pangénome, il est possible de détecter les variants structuraux (SVs) et les SNPs rapidement. Le graphe est indexé, rendant les recherches et l'alignement plus efficaces, notamment dans l'alignement de lectures ou dans la recherche de variants génétiques (\textit{variant calling}). L'outil a d'abord été développé pour la génomique humaine, mais il est tout à fait possible de l'utiliser avec des génomes procaryotes.

L'outil Minigraph \cite{li_design_2020}, lui aussi développé pour le variant calling sur le génome humain, propose une méthode demandant moins de ressources que VG. Le graphe est plus léger, sans annotation, permettant de construire des graphes de pangénome de grande taille, en utilisant peu de mémoire de calcul et de stockage. Minigraph permet de capturer les grandes variations génomiques, mais est moins performant sur la détection des SNPs par rapport à VG.

\paragraph{Graphe d'alignement multiple}

Une méthode, proche de la précédente, est celle basée sur l'alignement multiple des séquences (MSA\footnote{cf. \autoref{paragraph:MSA}}) entre elles. Cette méthode n'est pas dépendante d'une séquence référente. Le MSA permet de déterminer les variations entre les séquences, ce qui augmente le coût en ressources par rapport au graphe de variants prédéterminé. Toutefois, cette méthode est plus adaptée dans le cas où plusieurs séquences de bonne qualité sont disponibles pour construire le pangénome. En effet, le MSA permet de se passer du biais de la séquence référente dans la construction du graphe et d'ainsi mieux représenter la diversité génomique.

L'outil Harvest \cite{treangen_harvest_2014}, permet de comparer des génomes étroitement apparentés. Pour optimiser l'étape d'alignement, il utilise l'outil progressiveMauve \cite{darling_progressivemauve_2010}, qui fait un alignement progressif des séquences. Après l'alignement, il identifie le \textit{core genome} dans le pangénome et génère une phylogénie basée sur une matrice des SNPs. Bien qu'étant rapide et efficace, il n'est pas adapté aux génomes très divergents et il ne permet pas d'analyser les éléments mobiles (MGE).

PGGB \cite{garrison_building_2024}, utilise des algorithmes de graphes de préfixes minimaux (MPHF\footnote{Minimal Perfect Hash Function (MPHF) est une fonction qui associe de manière unique chaque élément d’un ensemble sans collisions et avec un espace mémoire minimal}), pour compresser le graphe et optimiser l'alignement. Il est capable d'identifier et de représenter les SNPs, SV, et les MGEs de manière efficace. PGGB est conçu pour mener des études pangénomiques à grande échelle, prenant en compte de grandes quantités de séquences, ce qui demande des ressources disponibles importantes. De plus, c'est un outil assez complet pour les analyses, ce qui peut le rendre difficile d'accès.

\paragraph{Graphe de De Bruijn}

Les graphes de De Bruijn (De Bruijn Graph : DBG) sont des graphes orientés dont les n\oe uds représentent des k-mers et les arêtes le chevauchement entre le suffixe et le préfixe (de taille k-1) des k-mers (\autoref{fig:deBruijn}). Ainsi, en suivant un chemin, il est possible de reconstituer une séquence. C'est pourquoi les DBG sont utilisés dans de nombreuses applications en bioinformatique (assemblage, correction des erreurs de séquençage, métagénomique\dots) et notamment en pangénomique.

\begin{figure}[htbp]
    \centering
    \includegraphics[width=.9\linewidth]{images/DBG.png}
    \caption[Exemple d'un graphe de De Bruijn]{\textbf{Exemple d'un graphe de De Bruijn.} Ici $k=3$, ce graphe permet de représenter et de reconstruire 3 séquences.}
    \label{fig:deBruijn}
\end{figure}

Les DBG, permettent d'avoir une structure compacte des séquences du pangénome. Les n\oe uds et les arêtes sont colorées en fonction des génomes dans lesquels ils sont retrouvés. Les DBG peuvent être compactés en cDBG, en fusionnant chaque région \textit{core}, \textit{i.e.} chaque suite de n\oe uds avec une seule arête entre chaque n\oe ud. Ces nouveaux n\oe uds fusionnés sont appelés "\textit{unitig"} et seront étiquetés avec la séquence combinée des k-mers.

L'une des premières méthodes développées utilisant des DBG est la méthode Cortex \cite{iqbal_novo_2012}, qui construit un DBG "coloré" (les arêtes et les n\oe uds sont étiquetés par les échantillons dans lesquels ils sont trouvés). À partir de ce DBG coloré, il est possible d'identifier les variants et de les associer à un génotype. Des outils plus récents, comme Bifrost \cite{holley_bifrost_2020}, améliorent les méthodes de coloration de DBG, permettant d'augmenter le volume de données pris en compte et supportant la mise à jour du graphe. Les auteurs de Bifrost ont notamment appliqué leur méthode sur une collection de plus de 100 000 génomes de \textit{Salmonella} \cite{luhmann_blastfrost_2021}, leur permettant d'identifier des gènes reliés à des îlots de pathogénicité et à une résistance aux fluoroquinolones\footnote{Classe d'antibiotique utilisée pour traiter les infections bactériennes graves.}.

\newpage
SplitMEM \cite{marcus_splitmem_2014}, permet de construire rapidement et efficacement des cDBG en intégrant une méthode appelée "saut de suffixe"\footnote{Le cDBG est relié à des arbres de suffixes, un saut de suffixe permet depuis un suffixe à l'extrémité d'une branche de l'arbre de sauté vers un même suffixe plus proche de la racine. Les sauts se poursuivent jusqu'à atteindre le suffixe le plus proche de la racine. Le chemin restant correspond au chemin le plus court sans ramification, entre la racine et le suffixe.}, qui permet de construire le cDBG sans passer par un DBG. L'outil permet ensuite de détecter dans l'ensemble des génomes ou dans un sous-ensemble de génomes les régions compressées (appelées \textit{Maximum Exact Matches} : MEMs), correspondant au \textit{core genome}. Cet outil est linéaire en temps et en espace pour identifier le \textit{core genome}, mais ne permet pas de mener d'autres analyses. De plus, la méthode a été testée sur un jeu de 62 génomes de \textit{E. coli}, le caractère linéaire est donc à vérifier sur de plus grands jeux de données.

PanTools \cite{sheikhizadeh_pantools_2016}, est un outil complet qui a largement évolué depuis sa publication. Il permet la construction de pangenomes basés sur des cDBG généralisés. PanTools est robuste à l'utilisation de grands volumes de données, que ce soit en temps, en mémoire ou en stockage. Il intègre également des méthodes d'annotation structurale et fonctionnelle, de partitionnement, d'alignement, de phylogénie, d'identification du \textit{core genome} et de visualisation. 

DBGWAS \cite{jaillard_fast_2018}, construit également les pangénomes avec des cDBG. Son originalité réside dans l'association de phénotypes (\textit{Genome Wild Association Study} : GWAS). L'intérêt d'utiliser le graphe de pangénome est qu'il n'est pas nécessaire d'utiliser une séquence de référence, contrairement aux approches classiques de GWAS. De plus, les phénotypes ne sont pas associés à des SNPs mais à des sous-graphes, permettant d'extraire des variants plus longs ou plus complexes. DBGWAS intègre une partie visualisation, permettant d'explorer les variants associés au phénotype dans leur contexte génomique pour identifier des régions variables plus larges comme les îlots génomiques. 

Le DBG (et ses dérivés) est une structure de données puissante, permettant de calculer, analyser et stocker, rapidement et efficacement, de grandes quantités de données. Néanmoins, ce qui fait la force de cette approche (l'utilisation de k-mer) est aussi sa faiblesse. Le choix de la taille du k-mer va influencer le graphe et donc la découverte des variations. De plus, cette structure est limitée dans l'identification et l'étude des régions répétées. C'est pourquoi des auteurs proposent des méthodes pour lier des informations au DBG \cite{turner_integrating_2018}.

\subsubsection{Application des graphes de pangénome.}

L'utilisation de pangénomes de séquence est très utile à partir de lectures courtes pour améliorer le génotypage. En utilisant le pangénome, contenant des variants connus, on améliore la couverture des lectures et donc on améliore le génotypage de ces lectures. Par rapport aux méthodes utilisant un génome de référence, le pangénome réduit le biais en faveur de la séquence de référence, particulièrement pour les grandes insertions/délétions et les SV. Le pangénome améliore aussi le \textit{variant calling} (VC) en augmentant sa précision, et à partir des DBG de faire du VC sans référence.

Les graphes de séquences sont également utilisés en métagénomique. L'outil MetaKallisto \cite{schaeffer_pseudoalignment_2017} utilise notamment une base de données de séquences représentantes qu'il représente sous forme de DBG coloré afin de faire de l'assignation taxonomique et de la quantification de séquences métagénomiques.

L'utilisation des graphes de séquences pour les GWAS permet de détecter finement des variations dans les populations associées à un phénotype. Chaguza \textit{et al.} \cite{chaguza_bacterial_2020} ont mené une étude sur 909 échantillons de souche hyper virulente de \textit{Streptococcus pneumoniae} (serotype 1). Ils ont pu identifier, grâce à l'outil DBGWAS, des mutations de certaines protéines associées à des phénotypes spécifiques (âge de l'hôte, géographie, structure des populations\dots). L'utilisation de graphes de pangénome a permis de mener une étude à large échelle, tout en prenant en compte toute la diversité sans nécessiter de référence.

\begin{table}[htbp]
    \centering
    \begin{tabular}{|p{.25\textwidth}|p{.3\textwidth}|p{.35\textwidth}|}
        \hline
        Nom & Méthode & Référence \\
        \hline
        NGSPanPipe & Séquence représentative & \cite{kulsum_ngspanpipe_2018} \\
        \hline
        Spine & Séquence représentative & \cite{ozer_characterization_2014}\\
        \hline
        VG toolkit & Variant prédéterminé & \cite{garrison_variation_2018} \\
        \hline
        Minigraph & Alignement sur graphe & \cite{li_design_2020} \\
        \hline
        PanVC & Variant prédéterminé & \cite{norri_founder_2021} \\
        \hline
        Minigraph-Cactus & MSA & \cite{hickey_pangenome_2024}\\
        \hline
        Harvest & MSA & \cite{treangen_harvest_2014} \\
        \hline
        PGGB & MSA & \cite{garrison_building_2024}\\
        \hline
        Cortex & graphe de De Bruijn & \cite{iqbal_novo_2012} \\
        \hline
        Bifrost & graphe de De Bruijn & \cite{holley_bifrost_2020} \\
        \hline
        SplitMEM & graphe de De Bruijn & \cite{marcus_splitmem_2014} \\
        \hline
        PanTools & graphe de De Bruijn & \cite{sheikhizadeh_pantools_2016} \\
        \hline
        twoPaCo & graphe de De Bruijn & \cite{minkin_twopaco_2017}\\
        \hline
        DBGWas & graphe de De Bruijn & \cite{jaillard_fast_2018}\\
        \hline
        PanVA & Visualisation & \cite{van_den_brandt_panva_2024} \\
        \hline
    \end{tabular}
    \caption[Outils de pangénomique basés sur les séquences]{\textbf{Liste non exhaustive d'outils de pangénomique basés sur les séquences.}}
    \label{tab:pangenomicToolsSeq}
\end{table}
\section{Pangenome de gènes}

\subsection{Famille de gènes homologues}
\label{sec:fam}

\newpage
\section{Conclusion sur les pangénomes et éléments de réflexions}

Nous avons présenté une grande variété de méthodes et d'outils de pangénomique (\textit{cf.} tableaux \ref{tab:pangenomicToolsSeq} et \ref{tab:pangenomicToolsFam}) et exploré plusieurs champs d'application des pangénomes. Cependant, plusieurs défis majeurs restent à relever en pangénomique.

Un premier défi concerne la représentation et la visualisation des pangénomes. En effet, ceux-ci doivent être visualisables par l'\oe il humain tout en intégrant l’ensemble de l’information génomique. Différentes approches ont été développées pour répondre à cette problématique, notamment des outils de visualisation interactifs. À titre d’exemple, Pan-Tetris\cite{hennig_pan-tetris_2015} permet de visualiser et de modifier la composition des groupes de gènes grâce à une technique inspirée du jeu Tetris. PanViz\cite{pedersen_panviz_2017} facilite la comparaison des génomes individuels aux pangénomes avec une navigation basée sur l'ontologie des gènes. PanVa\cite{van_den_brandt_panva_2024}, utilisant les pangénomes de PanTools\cite{sheikhizadeh_pantools_2016}, propose une approche centrée sur la variabilité structurale, permettant une analyse flexible des pangénomes en tenant compte des variations génomiques complexes. Enfin, PANACHE\cite{durant_panache_2021} propose une visualisation linéarisée des pangénomes, affichant la présence ou l’absence des blocs de séquences sous forme de navigateur génomique.

Un second défi crucial est celui du stockage et de la gestion des données pangénomiques. Contrairement aux génomes individuels, généralement stockés sous forme de texte et reliés à des bases de données, les pangénomes sont des structures plus complexes qui ne peuvent être stockées sous un format linéaire. Les BD doivent donc permettre un accès rapide et efficace aux informations, tout en étant mises à jour régulièrement pour suivre l’augmentation du volume des données génomiques. Des BD comme panKB \cite{sun_pankb_2025}, permettent d'avoir accès à des informations sur des pangénomes précalculés, reliés à des métadonnées comme : les publications de référence, les informations sur l'origine des génomes\dots 
Cependant, le pangénome en lui-même n'est pas disponible et seuls les résultats d'analyse sont disponibles.

Enfin, un défi essentiel concerne la construction du pangénome. Le choix des méthodes influence fortement le résultat final, mais les données d’entrée jouent également un rôle fondamental. L’objectif étant de représenter au mieux la diversité génomique, il est important de maximiser cette diversité tout en évitant les biais, tels que la surreprésentation de génomes pathogènes dans les bases de données. Un équilibre est nécessaire : trop de variations dans les données d’entrée peuvent complexifier l’analyse, en rendant par exemple les graphes de pangénome difficilement exploitables. La qualité des génomes, des annotations et des bases de données, sont également des facteurs déterminants pour garantir la robustesse des analyses pangénomiques.

Pour répondre à ces défis, les méthodes et les outils de pangénomique sont en constante évolution. Avec ces progrès, divers domaines de la microbiologie voient progressivement s'intégrer des analyses pangénomiques en routine. Cette intégration repose en partie sur des plateformes qui rendent ces outils accessibles et exploitables par la communauté scientifique. Par exemple, MicroScope\cite{vallenet_microscope_2020} permet de construire des pangénomes procaryotes à l’aide de PPanGGOLiN \cite{gautreau_ppanggolin_2020} et d’identifier les régions de plasticité génomique (RGP) grâce à la méthode panRGP\cite{bazin_panrgp_2020}. MicroScope constitue alors un point d'intersection entre les données génomiques, la production de pangénomes et leur utilisation effective, contribuant ainsi à relever progressivement les défis liés à la visualisation, au stockage et à la construction des pangénomes.