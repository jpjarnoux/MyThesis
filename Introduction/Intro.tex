\part{Procaryotes : de la biologie cellulaire à la génomique moderne}

Ce chapitre marque le début du manuscrit et posera les bases conceptuelles et méthodologiques,  biologiques et bioinformatiques essentielles à la compréhension des travaux menés dans le cadre de cette thèse. Il s’agit ici de contextualiser les enjeux de la génomique des procaryotes et de la pangénomique, tout en abordant les principaux concepts et méthodes utilisés dans ce domaine.

Nous commencerons par un rapide retour sur ce qu'est un procaryote, un élément clé pour définir les bornes et le contexte d’application de nos recherches. Cette partie est essentielle pour comprendre les spécificités des procaryotes et leur impact sur les approches méthodologiques adoptées. Cette introduction permettra également de situer les procaryotes dans la classification du vivant, notamment en revenant sur la structure cellulaire et l'organisation du génome, tout en apportant les éléments pour discuter de la notion d'espèce procaryote.

Une fois ce cadre biologique posé, nous aborderons les bases de la génomique comparée, en se focalisant sur l'application aux procaryotes. Ce moment sera l’occasion de clarifier l’utilisation de simplifications ou de choix algorithmiques, souvent nécessaires en raison des caractéristiques propres des génomes procaryotes. Ces éléments permettront de mieux comprendre l’approche bioinformatique qui sous-tend la comparaison des génomes, et ce, de la comparaison de séquences en allant jusqu'à l'approche par graphe en passant par les modèles statistiques et les méthodes d'intelligence artificielle.

Le chapitre poursuivra en contextualisant la pangénomique, un domaine en pleine expansion qui permet de saisir la diversité génétique des populations microbiennes et qui est le centre des travaux de recherche ici réalisés. Nous mettrons en lumière l’évolution des données biologiques, tant sur le plan quantitatif que qualitatif ou encore sémantique, et soulignerons les défis posés par la gestion et l’analyse de ces données, en particulier dans le cadre de l’évolution des connaissances ontologiques et syntaxiques, pour conclure par la manière dont la pangénomique a pu répondre à ces difficultés.  

À la fin de ce chapitre, le lecteur aura tous les éléments théorique et méthodologique pour aborder les travaux de recherche développés, tout en disposant du cadre dans lequel s'inscrit la thèse et des enjeux actuels de la génomique des procaryotes et de la pangénomique.

\chapter{Caractérisation et classification des procaryotes : de la cellule au génome}

Avant d'étudier la génomique des procaryotes, il convient de revenir sur ce qu'est un procaryote et comment le placer dans l'arbre du vivant. La classification du vivant est encore marquée de nombreux débats et donc elle est en perpétuel changement \cite{chun_integrating_2014}\cite{adl_revisions_2019}. De plus, il faut prendre en compte comment regrouper les individus en groupes (appelé taxon), c'est la taxonomie, mais aussi comment reconstruire les relations évolutives reliant les individus entre eux, c'est la systématique. Nous nous baserons sur la classification communément adoptée, c.-à-d., une division des êtres vivants en trois domaines : Bactéries, Archées et Eucaryotes\footnote{Dans cette vision de la classification, les virus ne sont pas intégrés, étant donné que leur appartenance au vivant est toujours débattue.}. Cette classification permet (dans de nombreux cas) de concilier une classification des espèces selon des critères phénotypiques et des critères génomiques.

\section{La classification des microbes : des critères phénotypiques à la biologie moléculaire}

Les premières classifications des microorganismes se sont appuyées sur des critères phénotypiques, c.-à-d., des caractéristiques observables. Bien que ces premières tentatives aient été limitées par la petite taille des organismes et les technologies, elles ont permis de distinguer plusieurs grands groupes.

Pour commencer, certains microorganismes sont pluricellulaires, comme les champignons du genre \textit{Penicillium}, tandis que d'autres, tels que la bactérie \textit{Escherichia coli}, ne sont constitués que d'une seule cellule et sont qualifiés d'unicellulaires. Dans la suite, nous nous concentrerons exclusivement sur les organismes unicellulaires\footnote{certains procaryotes montrent des formes de coopération et de différenciation cellulaire, suggérant une forme de multicellularité primitive. Cependant, elles ne sont pas multicellulaires au sens strict, car leurs cellules restent indépendantes sur le plan fonctionnel et structurel.}. 
La première distinction majeure qui a été établie pour diviser le vivant en deux grands domaines repose sur la présence ou l'absence de noyau. Le noyau est une structure interne de la cellule qui va contenir l'ensemble du matériel génétique. Les organismes (unicellulaires ou non) qui ont un noyau sont qualifiés d'eucaryotes. Pour ceux dont le matériel génétique est librement dispersé dans le cytoplasme, ils sont catégorisés dans le domaine des procaryotes. Ce sont ces derniers qui vont nous intéresser, et, sauf précision, ce qui sera dit s'appliquera à tous les procaryotes.

Le développement de la biologie moléculaire a permis d'affiner et de corriger les classifications précédentes en analysant la morphologie, la physiologie et la biochimie des cellules procaryotes, ainsi que les séquences d'ADN des génomes. C'est notamment en étudiant les gènes codant l'ARN 16S, qu'il a été mis en évidence que l'ensemble des procaryotes ne formait pas un groupe monophylétique, mais qu'ils étaient séparés en deux domaines, Bactérie et Archée\cite{woese_phylogenetic_1977}. Longtemps considéré comme des bactéries extrêmophiles, il est aujourd'hui clair que les Archées représentent un domaine à part entière avec toute sa singularité, comme la composition de leur membrane par exemple \cite{albers_archaeal_2011}. Malgré toute la fascination que nous pouvons avoir pour les archées, et que toutes les méthodes qui seront présentées peuvent s'appliquer aux espèces Archée, nous ne présenterons que très peu de résultats les concernant. C'est pourquoi dans la suite, même si nous parlerons de procaryote, nous considérerons plutôt le domaine des bactéries avec un prolongement possible aux archées.

\section{Taxonomie des procaryotes : un problème non résolu ?}

La classification des procaryotes et la définition d'espèce procaryote ne fait pas consensus dans la communauté des microbiologistes. Toutefois, les méthodes de classification se basent sur le même principe de relation entre les individus \cite{aldhebiani_species_2018}. Ces relations peuvent être soit phénétique, c.-à.-d, reposant sur la similarité d'un trait, sans s'intéresser au lien évolutif qui pourrait les relier, soit phylogénétique, c.-à.-d, reposant sur l'hérédité du caractère indépendamment de son état actuel.

Les premières tentatives de classification des bactéries reposaient sur des approches phénétiques, utilisant des critères basés sur les caractéristiques observables de ces organismes. Ces classifications s'appuyaient sur des caractères morphologiques, physiologiques et biochimiques.
D'un point de vue morphologique, les microbiologistes examinaient des paramètres tels que la taille des cellules, leur mode de croissance et leur capacité à former des agrégats spécifiques. La présence ou l'absence de structures spécialisées, telles que les flagelles, était également un critère de différenciation. 
Les caractéristiques physiologiques permettaient, quant à elles, de classer les bactéries selon leur mode de vie, leurs mécanismes métaboliques (anabolisme et catabolisme) et leurs réponses aux conditions environnementales.
L'étude de la composition cellulaire offrait par ailleurs de nouveaux outils pour affiner ces classifications sur le plan biochimique. Par exemple, la coloration de Gram, méthode emblématique, permet de différencier les bactéries en deux grands groupes : les Gram-positives, caractérisées par une paroi épaisse de peptidoglycane, et les Gram-négatives, qui présentent une paroi plus fine associée à une membrane externe lipidique.
Enfin, selon le contexte d’étude, d'autres critères peuvent être intégrés. Dans le domaine médical, la pathogénicité (capacité à induire une maladie) et le sérogroupage (basé sur la composition antigénique de la capsule bactérienne) sont particulièrement utilisés pour identifier et classifier les bactéries d'intérêt clinique.

Avec l'arrivée de la génomique, du séquençage et de la bioinformatique, ces classifications ont peu à peu laissé leur place à des classifications basées sur la phylogénie. Néanmoins, l'ADN a aussi été utilisé comme un critère phénétique définissant des critères biochimique comme similarité entre les souches. 
Dans ces critères, il y a d'abord le pourcentage de guanine-cytosine (GC) qui permet de différencier 2 souches appartenant à 2 genres différents si elles possèdent plus de 10 mol \%\footnote{1 équivalent molaire équivaut à 100 \% en mole, donc 10 \% en mole équivaut à 0,1 équivalent molaire.}, mais il faut noter qu'une composition en GC proche n'implique pas forcément que les souches soient proches. 
Une approche visant à définir formellement une espèce procaryote a été adoptée en 1987 par un comité d'expert \cite{moore_report_1987}. Il propose que des souches appartiennent à une même espèce si l'ADN s'hybride\footnote{Appariement de 2 brins d'ADN par complémentarité des bases} a plus de 70 \% et que le $\Delta T_m$\footnote{température à laquelle la moitié de l'ADN est dénaturés} diffère de 5 degrés ou moins.

Toutes ces approches ont permis de classer les procaryotes en taxon et dans la nomenclature de la taxonomie actuel, il reste des traces de ces méthodes. Elles sont d'ailleurs toujours utilisées et font partie des traits visible dans les classifications. Il faut d'ailleurs souligner qu'il n'est pas toujours possible d'obtenir des génomes de bonne qualité pour réaliser des phylogénies.

\section{Espèce procaryote : le génome complet et la phylogénie peuvent-ils trancher ?}

Les approches phénétiques présentées précédemment ont l'intérêt de s'appliquer directement au laboratoire et donc de regrouper et d'identifier les souches rapidement. Néanmoins, elles restent relativement approximatives et sont parfois coûteuses (en temps et en moyens). De plus, même si elles répondent aux problèmes de la taxonomie, et donc de ranger les bactéries dans des taxons, elles ne répondent pas à la question du lien entre les différents taxons et comment représenter ce lien, c.-à-d., à la question de la systématique.

Pour pallier les limites des approches précédentes, une nouvelle méthode a été développée et reste encore largement utilisée en routine aujourd'hui : la comparaison des souches à partir d'un gène marqueur. Il s'agit d'un gène présentant des variations spécifiques parmi les différentes souches d'intérêt, toutes dérivant d'une forme ancestrale commune ayant évolué différemment au fil du temps. Ainsi, le gène marqueur reflète à la fois la similarité entre les souches, permettant leur regroupement, et les événements dits de spéciation ayant conduit à leur séparation en espèces distinctes. On va privilégier l'utilisation de gènes hautement exprimés qui assurent une fonction essentielle à la vie de l'organisme : les gènes de ménage (\textit{house-keeping genes}). Un gène marqueur en particulier est utilisé : l'ADNr 16S, qui a la particularité d'être présent chez tous les procaryotes. En 2007, un arbre du vivant de toutes les espèces, a été reconstruit à partir d'un arbre d'ADNr 16S comprenant toutes les souches types séquencées d'espèces de bactéries et d'archées publiées jusqu'à la fin de l'année 2007 \cite{yarza_all-species_2008}. 
En allant encore plus loin, des analyses \textit{multilocus sequence analysis} MLSA ont été proposés \cite{glaeser_multilocus_2015}. Ces analyses prennent en compte plusieurs gènes marqueurs pour réaliser la taxonomie. L'utilisation de plusieurs gènes augmente le niveau d'information et réduit les biais. Toutefois, il n'y a pas de recommandation universelle pour réaliser l'analyse et chaque MLSA est réalisé en fonction des souches de départ. La sélection des gènes et leur nombre sont des paramètres qui ont un impact encore peu évalué sur la taxonomie. Il en va de même pour la taille des fragments considérés pour chaque gène, qui ne représente qu'une partie de la séquence du gène. Enfin, expérimentalement, il est souvent difficile, voire impossible, de concevoir des amorces facilitant l'amplification des gènes dans toutes les souches prises en compte. Malgré ces critiques, l'utilisation de gènes marqueurs est encore aujourd'hui utilisée, mais est peu à peu remplacée par des méthodes prenant en compte l'ensemble du génome.


Au début des années 2000 et avec les nombreux projets autour du séquençage et de l'analyse des génomes, comme le projet génome humain \cite{lander_initial_2001}, les technologies de séquençage sont de plus en plus précises et de moins en moins couteuse, amenant dans la génomique "moderne" : une augmentation exponentielle du nombre de séquences et des séquences plus longues et de meilleure qualité. C'est l'arrivée du \textit{Whole Genome Sequencing} (WGS) et de l'analyse de génomes complets de procaryote. En réalité, les premiers génomes complets ont été séquencés et assemblés il y a longtemps (1995), mais pour les utiliser en génomique comparée et en phylogénie, il fallait aussi que les technologies et les algorithmes bioinformatiques se développent à leur tour, c'est pourquoi les méthodes présentées précédemment étaient privilégiées.

Grâce aux nouvelles méthodes de génomique comparée, que nous présenterons dans le chapitre suivant, il est désormais possible de considérer le génome complet pour faire l'assignation taxonomique d'une bactérie. Une de ces approches est l'\textit{Average Nucleotide Identity} (ANI), qui rend compte de la similarité entre 2 séquences nucléotidiques. Le score d'ANI va d'ailleurs remplacer celui de l'hybridation, où un ANI inférieur 95 \% permet de différencier les espèces à la place d'une hybridation à 70 \% \cite{goris_dnadna_2007}. Plus récemment, le seuil de 95 \% a été confirmé par les auteurs de FastANI \cite{jain_high_2018}, utilisant plus de 90000 génomes. Ils ont montré l'existence d'un \textit{gap}, espace où l'ANI diminue fortement avant 95 \% (\autoref{fig:ANI_gap_sp}).


\begin{figure}[htbp]
    \centering
    % Première image
    \subfloat{%
        \includegraphics[width=0.48\textwidth]{images/ANI_gap.jpg}
    }
    \hfill % Espace flexible entre les deux images
    % Deuxième image
    \subfloat{%
        \includegraphics[width=0.48\textwidth]{images/ANI_sp.jpg}
    }
    \caption[Variation du score d'ANI au niveau de l'espèce]{Variation du score d'ANI au niveau de l'espèce. (A-B) Les histogrammes sont basés sur des comparaisons par paire effectuées avec FastANI. (A) Le Score d'ANI représenté au niveau de l'espèce se base sur les données de Jain \textit{et al}. On y retrouve un \textit{gap} entre 84 et 95 \% d'ANI. (B) Score d'ANI représenté au niveau intra-espèce sur les données de Rodrigues-R \textit{et al}. On retrouve un \textit{gap} entre 99,2 et 99,8 \% d'ANI. (C-D) Score d'ANI au niveau du groupe \textit{Escherichia coli}. Le nombre de génomes utilisés est le suivant : \textit{E. coli} : 2815 ; \textit{Salmonella enterica} : 1351 ; \textit{Escherichia fergusonii} : 57 ; \textit{Escherichia albertii} : 70 ; et \textit{Shigella flexneri} : 93 (tous les génomes complets disponibles au NCBI en juillet 2023). (C) Comparaison de l'ANI entre \textit{E.Coli} et d'autres espèces. Le seuil de 95 \% délimitant l'espèce est retrouvé. Un \textit{gap} à 97 \% existe entre \textit{E.coli} et \textit{Shigella flexneri} (une espèce d'\textit{E.Coli} particulière pour ces propriétés infectieuse). (D) Analyse de l'ANI au sein des génomes de \textit{E.Coli}. L'écart d'ANI de 99,5 \% est aussi prononcé, par rapport aux barres adjacentes, que l'écart d'ANI de 98 \%-97 \% qui correspond à l'écart entre les phylogroupes d'\textit{E. coli}, un groupe distinct et bien reconnu au sein d'\textit{E. coli}. Figures et légende adaptées de \cite{konstantinidis_sequence-discrete_2023}}
    \label{fig:ANI_gap_sp}
\end{figure}


Pourtant, la communauté n'est toujours pas arrivée à un consensus sur la classification des procaryotes en espèces et même sur l'existence d'espèces procaryotes. On peut d'abord critiquer l'approche et les résultats des études utilisant l'ANI, qui se limitent aux génomes de bonnes qualité et complets, ce qui \textit{de facto} limite le nombre de génomes et d'espèces potentielles pris en compte, tout en augmentant la redondance et limitant la diversité et la variabilité. De plus, la démarche apporte le biais d'utiliser une taxonomie déjà existante. Il faut aussi prendre en compte que la dynamique évolutive des procaryotes, que nous détaillerons dans le chapitre suivant (\autoref{sec:dyn_evo}), n'est pas linéaire et héréditaire, mais que les procaryotes sont capables de recevoir et d'échanger de l'ADN. C'est pourquoi des auteurs soutiennent une définition plus écologique de l'espèce bactérienne \cite{luo_genome_2011}, prenant en compte ces échanges agissant sur le \textit{fitness} des bactéries dans leur environnement.


On peut donc convenir qu'il n'est pas encore communément admis de parler d'espèce procaryote. Il existe toutefois des caractéristiques communes et spécifiques aux procaryotes ainsi que des traits propres à chaque taxon. De nombreuses méthodes et démarches scientifiques parviennent à construire une phylogénie des procaryotes, mais celle-ci doit être replacée dans son contexte d'étude pour prendre sens. Notamment en pangénomique, on étudie régulièrement le pangénome d'une espèce, il est donc nécessaire de se baser sur une classification des génomes en espèce. Dans le contexte de nos travaux, la similarité des séquences l'emporte comme critère de classification, nous utiliserons donc des génomes provenant de bases de données utilisant des critères comme l'ANI ou des gènes marqueurs pour construire des pangénomes.

%Enfin, avec l'explosion du nombre de séquences disponible, nous voyons l'émergence d'un paradoxe : de plus en plus de données sont disponibles, mais alors que l'on pensait pouvoir ranger les procaryotes dans des boites bien précises qui se verraient valider au cours du temps, une nouvelle exception vient renverser l'ordre actuel et la phylogénie doit être revue.

\chapter{Génomique des procaryotes : organisation, évolution et fonctions}

Avant d'aborder la génomique comparée des procaryotes, il convient de revenir sur ce qu'est un génome procaryote. Les génomes procaryotes sont souvent décrits comme plus simple et plus facile à étudier que les génomes eucaryotes. Pourtant, sous cette simplicité apparente, il reste encore de nombreuses parts d'ombre sur l'organisation et la régulation des génomes procaryotes. Quant à la dynamique évolutive de ces génomes, nous avons vu qu'elle pose encore de nombreux problèmes aux spécialistes de la phylogénie. Enfin, les procaryotes sont toujours autant étudiés, car ce sont des réservoirs d'enzyme et processus chimique qui peuvent être utilisés dans de nombreux domaines. Des molécules et des réactions qui nous sont parfois encore inconnu et que nous sommes incapables de reproduire. Dans cette partie, je décrirais les mécanismes les plus connus et les plus répandus qui seront également des principes fondamentaux de nos hypothèses de développement méthodologique et d'analyse pangénomique. Je laisserai donc à chacun se faire une idée de la simplicité des génomes procaryotes. 

\section{Structure et organisation des génomes procaryotes}

Avant de décrire le génome, revenons rapidement sur sa définition. Le génome, c'est l'ensemble du matériel génétique, c.-à-d., des éléments qui seront hérités par les cellules de la génération suivante. Le génome, c'est aussi la structure de base qui va contenir l'ensemble des informations nécessaires au fonctionnement et à la survie de la cellule. Ces informations sont contenues dans la molécule d'ADN, ce qui nous amène à la structure primaire du génome, la séquence nucléotidique. Cette séquence est souvent circulaire chez les procaryotes et est de petite taille, quelques centaines de milliers de bases, mais certains génomes peuvent atteindre plusieurs millions de bases\footnote{En bioinformatique, on utilise l'unité base (b) ou paire de base (pb), pour mesurer la taille d'un génome. Un génome procaryote sera donc compris entre 100 kb et 10 Mb. Pour comparaison, le génome humain mesure environs 3 Gb.} (\autoref{fig:genome_size}).


\begin{figure}[htbp]
    \centering
    \includegraphics[width=\linewidth]{images/genome_size.png}
    \caption[Tailles des génomes pour différents groupes taxonomiques]{Variation de la taille des génomes (en paire de base) pour différents groupes taxonomiques. Copié de \cite{milo_cell_2015}}
    \label{fig:genome_size}
\end{figure}

Le génome est divisé en sous-unité que l'on appelle gène. Le gène contient l'information nécessaire pour produire une protéine qui réalisera une fonction dans la cellule (\autoref{fig:gene2prod}). Ces protéines correspondent à une chaîne d'acide aminé, que l'on peut représenter sous forme de séquence. Pour passer d'un gène à une protéine, on utilise une table de correspondance que l'on appelle code génétique où 3 nucléotides correspondent à 1 acide aminé. En moyenne, une protéine contient 300 acides aminés, ramenant la taille des gènes à environs 1 kb. Enfin, comme indiqué sur la partie haute de la \autoref{fig:genome_size}, les génomes procaryotes sont majoritairement codants, ce qui veut dire que presque tous l'ADN peut être divisé en gènes, et donc qu'il y a environs entre 100 et 10 000 gènes dans les génomes en fonction de leur taille. En mettant toutes ces informations en perspective, la petite taille des génomes procaryotes est compensé par son fort taux de gènes, ainsi, il contient l'ensemble des protéines nécessaires à la survie de la cellule. 


\begin{figure}[htbp]
    \centering
    \includegraphics[width=\linewidth]{images/gene2prot.jpg}
    \caption[Produit d'un gène]{Produit d'un gène dans la cellule. Un gène est d'abord transcrit en ARN. Si l'ARN transcrit est dit messager (ARNm), il sera ensuite traduit en protéine, sinon l'ARN produit (ARNt, ARNr, miARN, ....) aura un rôle spécifique dans des processus cellulaire. Copié de RNBio, Sorbonne université. \url{https://rnbio.sorbonne-universite.fr/genetique_genotype1}}
    \label{fig:gene2prod}
\end{figure}


Dans la cellule, l'ADN ne reste pas sous cette forme primaire de séquence, il va se replier par différent mécanisme pour arriver dans une forme plus compacte qu'on appelle le chromosome (\autoref{fig:structure_dna}). L'ADN commence par se replier dans une structure secondaire, notamment la célèbre double hélice décrite par Watson, Crick et Franklin \cite{watson_molecular_1953}\footnote{Ces travaux sont souvent cités comme exemple dans la lutte pour la reconnaissance des femmes en sciences, Rosalind Franklin ayant joué un rôle essentiel, mais souvent sous-estimé dans cette découverte.}. Bien que la double hélice soit la forme la plus connue, d'autres conformations secondaires, telles que les structures en triple hélice ou en Z, ont également été identifiées, comme illustré dans la \autoref{fig:structure_dna}. L'organisation de l'ADN va au-delà de cette structure secondaire : il est ensuite soumis à des mécanismes de superenroulement induits par des enzymes spécifiques comme les topoisomérases et les gyrases. Ce superenroulement permet de réduire davantage la taille de l'ADN et de favoriser son organisation en boucles maintenues par des protéines structurales telles que HU, IHF ou H-NS \cite{williams_molecular_1997,prieto_genomic_2012}. Pour terminer des protéines appelé histones vont terminer de replier l'ADN en formant des nucléosomes, les procaryotes utilisent ces protéines pour compacter leur ADN en une structure appelée nucléoïde. Cette forme, au-delà d'optimiser l'espace dans la cellule, permet aussi de stabiliser et de protéger l'ADN, ainsi que la régulation de l’expression des gènes. Par exemple, la méthylation de l’ADN, ainsi que les modifications des protéines associées, sont des mécanismes clés de l’épigénétique. Ces processus influencent la transcription des gènes et ont des implications fonctionnelles majeures. Des études récentes ont mis en lumière le rôle de la méthylation dans la régulation de la virulence bactérienne et dans la capacité des procaryotes à coloniser leurs hôtes \cite{oliveira_bacterial_2021}, soulignant ainsi l'importance de ces mécanismes dans la survie et l’adaptation des bactéries.

\begin{figure}[htbp]
    \centering
    \includegraphics[width=0.8\linewidth]{images/structureDNA.jpg}
    \caption[Structure de l'ADN]{Représentation de la structure primaire, secondaire, tertiaire et quaternaire d'un acide nucléique. PDB ID : 1EQZ et 4R4V. Tiré de \cite{kumar_biomolecular_2019}}
    \label{fig:structure_dna}
\end{figure}

Un génome procaryote est donc en résumé un génome de petite taille, souvent circulaire et majoritairement codant. Il est donc essentiel de comprendre que la moindre modification dans la séquence d'ADN peut amener soit à un changement dans la séquence protéique, et donc son incapacité à fonctionner correctement, soit à l'impossibilité de produire la protéine. Une vision plus positive sera aussi d'imaginer que des changements dans la séquence d'ADN permettrons de produire une nouvelle protéine d'intérêt pour la cellule. Dans la suite, avec une vision darwinienne\footnote{Vision de l'évolution proposée par Charles Darwin, qui propose que les espèces évolue perpétuellement de façon hasardeuse et que les innovations génétiques sont ensuite maintenues ou perdues dans les populations par pression de sélection}, nous verrons par quels mécanismes la séquence d'ADN va évoluer, mais aussi comment ces évolutions seront transmises aux autres cellules procaryotes. 

\section{Dynamique évolutive des génomes : mécanismes et impacts}
\label{sec:dyn_evo}

Lorsqu'on étudie l'évolution des génomes, on s'intéresse aux changements apportés à la séquence d'ADN de la cellule : les mutations. Les mutations peuvent induire soit un gain, une perte ou une modification de la séquence génétique en fonction du mécanisme sous-jacent. Ces mécanismes sont complexes et bien différents de ceux que l'on pourrait concevoir avec une vision anthropomorphique. En effet, les cellules procaryotes ne s'accouplent pas pour produire une nouvelle cellule. Dans la nature, les procaryotes vont se multiplier par division cellulaire où une cellule mère donnera 2 cellules filles possédant le même matériel génétique que la mère, moins les possibles changements que nous décrirons dans la \autoref{sec:evo_ver}. Lorsque l'ADN est hérité de la cellule mère par la cellule fille, on va parler de \textbf{transfert vertical}. Il existe également (toujours par anthropomorphisme) une forme de sexualité des procaryotes, où 2 cellules vont échanger du matériel génétique sans qu'une nouvelle cellule ne soit créée. Dans ce cas, l'ADN est échangé entre 2 cellules dites de la même génération et on parle de \textbf{transfert horizontal} (voir \autoref{sec:evo_hz}). 

Ces mécanismes présents dans la nature sont exploités en microbiologie et en biologie cellulaire pour introduire des changements de gènes spécifiques dans une cellule et ainsi obtenir des espèces chimériques hybrident qui pourront être utilisées dans la recherche ou l'industrie \cite{baby_chromosomes_2019}. On peut aussi penser aux cellules procaryotes vivant en symbiose, voire en endosymbiose\footnote{une bactérie réside à l'intérieur d'une autre cellule (procaryote ou eucaryote)}, qui pourrait être considéré comme une étape préliminaire à une "fusion" évolutive. Ce mécanisme serait d'ailleurs à l'origine d'organites comme la mitochondrie et le chloroplaste\cite{martin_endosymbiotic_2015}. La fusion de cellules procaryotes est d'ailleurs possible et réalisée en laboratoire en enlevant leur paroi cellulaire pour obtenir des protoplastes. Les protoplastes peuvent être fusionnés grâce à des agents chimiques (comme le polyéthylène glycol) ou des chocs électriques (électrofusion) \cite{schaeffer_fusion_1976}.

\subsection{Mécanismes d'évolution par héritage}
\label{sec:evo_ver}
Les mécanismes d'évolution par héritages regroupent les processus menant à une modification du génome entre la cellule mère et la cellule fille. Théoriquement, lors de la division cellulaire, la cellule mère se divise en 2 cellules filles possédant exactement la même information génétique qu'elle. Pourtant, malgré un ensemble de mécanisme de protection et de correction de l'ADN, le génome peut différer entre les cellules mère et filles. Ce sont ces "erreurs" qui vont nous intéresser, car ce sont elles qui sont à l'origine de l'innovation et de la diversité génétique.

\subsubsection{Mutation génétique : un petit changement aux grandes conséquence}
\paragraph{\textit{Single Nucleotid Polymorphism}}

Un \textit{Single Nucleotide Polymorphism} (SNP) est un mécanisme d'évolution qui induit une modification de la séquence par la transformation d'un nucléotide en un autre. Étant donné que le code génétique est dégénéré\footnote{Un acide aminé peut être codé par plusieurs codons différents.}, la mutation peut ne pas avoir d'impact sur la séquence de la protéine, on dit que la mutation est silencieuse ou même sens. Si la modification change la séquence protéique, dans ce cas, on parle de mutation faux-sens. Enfin, Une mutation est qualifiée de non-sens lorsqu'elle affecte un point clé de la séquence protéique, comme le site actif ou un codon STOP, entraînant une perte de fonction de la protéine, ou lorsqu'elle introduit prématurément un codon STOP dans la séquence.

\paragraph{Indels: insertion, délétion et pseudogènes}

Un indel correspond à l'insertion (In) ou la délétion (del)\footnote{On regroupe l'insertion et la délétion, car sans une analyse phylogénétique, il est impossible de les différencier par comparaison de séquence.} d'un ou plusieurs nucléotides dans la séquence d'un gène. 

Lorsque la taille de l'indel est un multiple de 3 (insertion ou délétion d'un codon), la séquence protéique peut soit être allongé ou raccourci d'un acide aminé, soit coupé de façon précoce si le codon est un codon STOP.

Si la taille de l'indel n'est pas un multiple de 3, il y aura un décalage du cadre de lecture ou \textit{frameshift}. Ce décalage va induire un changement de tous les acides aminés de l'indel à la fin du gène, provoquant avec lui un changement dans la fonction de la protéine ou une inactivation de la fonction. La partie du gène qui n'est pas décalé est alors considéré comme un fragment du gène initial, il est alors qualifié de pseudogène. À nouveau, cette mutation peut être délétère pour la cellule. 

Les indels vont donc transformer la séquence protéique traduite, pouvant nuire à la fonction de cette dernière et être délétère pour l'organisme. Pour éviter les problèmes liés au \textit{frameshift}, il a été montré qu'il existe un fort taux de codon STOP hors du cadre de lecture \cite{tse_natural_2010}. Cette adaptation permettrait de limiter la traduction des protéines mutante et d'ainsi limiter le coût énergétique pour la cellule. Il a aussi été montré que les \textit{frameshift} pourrait être à l'origine d'un réservoir d'adaptation à l'environnement \cite{koch_catastrophe_2004}. Lors d'un changement dans l'environnement créant une nouvelle pression de sélection, un \textit{frameshift} pourrait améliorer le fitness de certains organismes. Une fois que l'élément perturbateur de l'environnement disparait, un nouveau frameshift ramènerait le cadre de lecture à sa place d'origine. Ce mécanisme, en accord avec la petite taille des génomes, aurait l'intérêt de ne pas perdre des gènes d'adaptation à l'environnement, même s'ils ne sont nécessaires que ponctuellement.

\subsubsection{Réarrangement génomique : un moteur de l'évolution}
\paragraph{Réarrangement}
\paragraph{Recombinaison}
\paragraph{Duplication}

\subsection{Mécanismes d'évolution intragénérationnelle}
\label{sec:evo_hz}

\subsubsection{Conjugaison : la sexualité des procaryotes}

\subsubsection{Transformation : recycler l'ADN environnant}

\subsubsection{Transduction : un sacrifice pour le bien commun}

\subsection{Interprétation des évolutions : l'homologie et ses déclinaisons}


\section{Du génome aux processus cellulaires : exploration fonctionnelle}

\subsection{Gènes et fonctions}

\subsection{Îlots génomiques}

\chapter{Génomique comparées des procaryotes}
\section{Analyse comparative des génomes : méthodes et applications}
\label{sec:comp_gen}
\subsection{Comparaison des séquences}
\subsection{Statistique et séquence}
\subsection{Utilisation des graphes}
\subsection{Application de la génomique comparée pour l'étude des procaryotes}
\subsection{Intelligence artificielle : machine learning et deep learning}
\section{Système biologique}
\subsection{Définition et intérêt}
\subsection{Méthode de détections}
\subsection{Systèmes de défense aux phages}
\chapter{Pangénomique: état des lieux, enjeux et ambitions}
\section{Origine et concept}
\section{Modélisation des pangénomes}
\section{méthodes de construction}
\section{Analyse pangénomique}
\chapter{Génomique à l'ère du BigData}
\section{Base de données génomiques : ressources et exploitation}
\section{La pangénomique}
\section{Base de données orienté graphe et données biologique}