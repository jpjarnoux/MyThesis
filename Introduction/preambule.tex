\chapter*{Introduction}
\addcontentsline{toc}{chapter}{Introduction}

Cette introduction a pour objectif de faire le panorama scientifique et historique des différents sujets qui seront abordés dans ce manuscrit de thèse. Elle fait aussi office d'entrée en matière pour les personnes non expertes qui liront ce manuscrit. Pour ces quelques lignes, je me permettrai donc quelques facilités et imprécisions scientifiques.%, en espérant qu'elles me seront excusées.


Les conditions qui ont permis à la vie de naître sur Terre suscitent encore de nombreuses questions et sont à l'origine de débat scientifique passionnant. Néanmoins, les premières traces de vie retrouvées remontent à 4 milliards d'années et correspondent à des microorganismes, des êtres invisibles à l'{\oe}il nu. Ces microorganismes colonisent la Terre depuis des milliards d'années et représentent aujourd'hui la proportion d'êtres vivants la plus importante en termes de nombre et de diversité. Ils jouent un rôle crucial dans les écosystèmes, les cycles biogéochimiques et la santé de la planète comme de la nôtre. En effet, ces microbes sont connus pour poser des problèmes de santé publique (épidémie, hygiène\dots), de contamination des plantes et des sols, ou encore de dégradation des matériaux. En contrepartie, ils peuvent aussi améliorer notre santé (les probiotiques par exemple), fertiliser les sols et épurer les eaux et être utile dans l'industrie et les biotechnologies (fermentation des fromages et des bières). Pourtant, la microbiologie, l'étude des microorganismes, reste une science relativement récente. Même s'il existe bien, dans l'Antiquité, certains savants et philosophes qui avaient déjà imaginé ces "animaux invisibles", marquant une compréhension primitive de la transmission des maladies infectieuses, il faudra attendre l'invention du microscope par Leeuwenhoek, au XVIIe siècle, pour qu'il fasse les premières observations d'\textit{animaculum}, marquant la naissance de la microbiologie. La microbiologie du XVIIe au XXe siècle a amené de grandes découvertes et révolution scientifique, notamment en médecine. Nous pouvons citer les travaux de Louis Pasteur qui a prouvé, en 1877, que les maladies infectieuses étaient causées par des microorganismes (staphylocoque, pneumocoque et streptocoque), ou encore d'Alexander Fleming qui découvrit, en 1928, la pénicilline, le premier agent antibiotique. 


En s'éloignant quelque peu de la microbiologie, toujours entre le XVIIe et XXe siècle, les chimistes s'intéressent aux molécules du vivant. Autour des années 1800, Le Français Antoine Fourcroy va faire la première description de substances azotées dans les organismes vivants, qu'il appelait "substances animales". C'est ensuite, en 1835, que le chimiste néerlandais Gérardus Johannes Mulder découvre des chaînes de substance azotées, qui seront introduites sous le terme de protéine par le chimiste suédois Jöns Jacob Berzelius en 1838. Le mot vient du grec \textit{proteios}, qui signifie "de première importance", soulignant l'intérêt fondamentale de ces molécules composées de carbone, hydrogène, azote et oxygène, avec des proportions spécifiques, dans les organismes vivants. Enfin, en 1894, le chimiste allemand, Emil Fischer, démontra que les protéines sont composées d'acides aminés, unité de base des protéines, liés par des liaisons peptidiques. Il déterminera la composition et la structure de plusieurs d'entre eux. La fin du XIXe siècle voit aussi la découverte d'une autre molécule du vivant, l'acide désoxyribonucléique, mieux connue sous l'acronyme ADN. C'est le biologiste suisse Friedrich Miescher qui découvre, en 1869, une substance riche en phosphore dans les cellules du pus, qu'il appelle "nucléine". Il faudra attendre près d'un demi-siècle (1929) pour que Phoebus Levene, biochimiste russe-américain, identifie les composants de base de l'ADN : les nucléotides. Plus tard, en pleine Seconde Guerre mondiale, les chercheurs Oswald Avery, Colin MacLeod et Maclyn McCarty, confirment l'hypothèse de Miescher, en montrent que l’ADN est la substance qui transfère les caractères héréditaires et que l'ADN est le support de l’information génétique. Pour terminer, les travaux de James Watson, Francis Crick  et Rosalind Franklin, ont permis de décrire la structure de la molécule d'ADN. Toutes ces découvertes ont ouvert la voie à de nombreuses autres dans tous les domaines : médecine, agro-alimentaire, biotechnologie, et sont le socle de la génétique moderne.

Les développements technologiques de la seconde moitié du XXe siècle, et notamment l'apparition du séquençage et de l'informatique, amènent les chercheurs à créer une nouvelle discipline pour l'étude de la structure et de la composition des molécules du vivant : la bioinformatique. En 1955, Frederick Sanger séquencera la première protéine, l'insuline. Cette découverte, récompensée par un prix Nobel, a établi la base du séquençage. Peu de temps après, Margaret Dayhoff, une pionnière de la bioinformatique, développe l'un des premiers programmes informatiques pour analyser les séquences de protéines. Elle publiera d'ailleurs, en 1969, le premier atlas de séquences protéiques, jetant les bases de l'analyse des séquences biologiques. À partir de là, la bioinformatique ne cessera d'évoluer avec les techniques de séquençage. En 1970, Saul Needleman et Christian Wunsch introduisent un algorithme pour l'alignement global des séquences, qui est toujours utilisé aujourd'hui. En 1977, Sanger va à nouveau révolutionner le domaine de la biochimie en proposant une méthode de séquençage de l'ADN qui portera son nom. Elle devient rapidement la méthode de référence en raison de sa précision. Dans les années 80, la méthode s'automatise, devient plus rapide et précise. On voit alors se développer les premières bases de données accessibles au public pour stocker des séquences génétiques et protéiques. En 1990, est lancé le Projet du Génome Humain (HGP), un effort international visant à séquencer l'intégralité du génome humain. Ce projet catalyse de nombreux développements en bioinformatique, notamment dans la gestion et l'analyse des grandes quantités de données générées. Enfin, au début des années 2000, les technologies de séquençage sont de plus en plus performantes et abordables, faisant entrer la bioinformatique dans l'âge du \textit{Big Data}, la rendant essentielle dans de nombreux domaines d'étude en biologie.


La microbiologie et, pour ce qui va nous intéresser ici, l'étude de la génétique des microorganismes profitent de toutes ces nouvelles technologies pour développer ces connaissances. Elle va aussi subir cette explosion de la quantité d'informations disponible dans les bases de données. C'est pourquoi, microbiologistes et bioinformaticiens sont toujours à la recherche de nouvelles méthodes pour l'analyse de ces données. Alors que les programmes bioinformatiques s'attachaient à représenter et à étudier un génome en tant qu'une séquence indépendante des autres, un nouveau concept de représentation des génomes est apparu : le pangénome. Il permet de regrouper l'ensemble des génomes en une seule entité et donc rendre une représentation globale de l'ensemble de l'information contenue dans les génomes. Le pangénome garantit une meilleure représentation de la diversité des génomes, tout en étant plus adapté à l'analyse de grandes quantités de données. C'est dans ce cadre que j'ai effectué mon travail de thèse, avec pour objectif de proposer de nouvelles méthodes en pangénomique pour l'analyse comparé des génomes de microorganismes a l'ère du \textit{Big Data}.

%Cette courte introduction permet, je l'espère, de montrer le caractère multidisciplinaire et international de la bioinformatique, mais aussi de comprendre les bases sur lesquelles reposent nos connaissances.

Ce manuscrit sera divisé en plusieurs parties comme suit. Une première partie sera consacrée à contextualiser et à rendre compte des problématiques auxquelles répond ce travail de thèse. Dans cette partie, je donnerai la définition précise des termes que j'utiliserai et je reviendrai sur l'état de l'art en génomique comparée des procaryotes. Je poursuivrai par une seconde partie sur les développements méthodologiques que j'ai pu réaliser en pangénomique, notamment dans la suite logicielle PPanGGOLiN. La troisième partie sera consacrée au c{\oe}ur de mon sujet de thèse, c.-à-d., aux développements de méthodes pour la comparaison de pangénomes. Enfin, je présenterai une nouvelle approche utilisant les bases de données orientées graphe comme solution pour le stockage et l'étude des pangénomes. Pour terminer, je présenterai une discussion critique sur le travail réalisé pendant ces trois ans et demi. Je me dois également de rappeler aux lecteurs que la nature, \textit{mirabile dictu}, se distingue par une diversité extraordinaire, et que certaines règles ou affirmations généralement vérifiées peuvent souffrir d'exceptions.
\newline

Je vous souhaite bonne lecture de ce manuscrit, qui, je l'espère, fait preuve de toute la rigueur scientifique attendu et rend compte du travail réalisé pendant ces trois ans et demi de manière authentique.
