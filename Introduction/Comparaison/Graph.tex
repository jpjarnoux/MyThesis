\subsection{Utilisation des graphes en génomique comparé}

Les graphes sont largement utilisés en bioinformatique \cite{pavlopoulos_using_2011} et ceux dans des domaines très divers : interaction protéine-protéine, expression des gènes, modélisation du métabolisme\dots  Durant ma thèse, les graphes ont largement été utilisés, il est donc essentiel de revenir sur la terminologie et les concepts liés à la théorie des graphes.

\subsubsection{Définitions et concepts}

Un graphe est constitué d'un ensemble de n\oe uds relié par un ensemble d'arêtes. Pour exemple, prenons un ensemble de molécules métaboliques qui constitueront nos n\oe uds et les arêtes représentent les réactions biologiques. Ce graphe représente alors le réseau métabolique.

Mathématiquement, tous les graphes ne possèdent pas les mêmes propriétés et donc les théorèmes associé change. Dans la suite, nous utiliserons les symboles mathématiques suivants :
\begin{itemize}
    \item $V$ : ensemble de n\oe uds
    \item $E$ : ensemble d'arêtes
    \item $G(V, E)$ : un graphe composé d'un ensemble de n\oe uds $V$ et d'arête $E$
    \item $u$ et $v$ : 2 n\oe uds distincts dans le graphe
    \item $e_{(u,v)}$ : une arête reliant $u$ et $v$.
\end{itemize}

\paragraph{Orientation du graphe}

Un graphe peut être orienté, \textit{i.e.}, que les arêtes ont une direction. Dans ce cas, il peut exister une arête de $u$ vers $v$ ($e_{(u, v)}$) sans qu'il n'y ait nécessairement une arête $e_{(v, u)}$. Si le graphe est non orienté, si $e_{(u, v)}$ existe, $e_{(v, u)}$ également. Dans notre exemple, entre 2 métabolites, l'un est le réactif et le second le produit, le graphe est donc orienté. Les graphes représentant les interactions protéine-protéine sont un exemple de graphe non orienté.

\paragraph{graphe pondéré et étiqueté}

En bioinformatique, il est courant d'ajouter de l'information sur le graphe. Ces informations peuvent servir à modifier le graphe, le filtrer ou l'analyser par exemple. On peut ajouter un poids aux n\oe uds ($w_u$) et aux arêtes ($w_{(u,v)}$), le graphe est alors dit pondéré. Le poids est quantifiable et correspond généralement à un nombre. Dans notre exemple, on peut ajouter l'énergie nécessaire pour passer d'une molécule à l'autre sur les arêtes, ainsi, on pourra filtrer les réactions trop énergivores. 

D'autres informations peuvent être ajoutées aux n\oe uds et aux arêtes sous forme d'annotation. Dans ce cas, l'annotation peut être qualitative et on dira que le graphe est étiqueté\footnote{Dans la littérature bioinformatique, on retrouve aussi le terme "coloré", mais qui est utilisé à tort si on se réfère à la théorie des graphes.}. Dans notre exemple, on peut ajouter aux n\oe uds l'identifiant unique de la molécule et aux arêtes les protéines impliquées dans la réaction.

\paragraph{Voisinage et chemin dans le graphe}

Dans cette thèse, nous parlerons de n\oe uds voisins, \textit{i.e.}, des n\oe uds qui sont reliés par un ensemble d'arêtes ($E_{(u,v)}$), cet ensemble d'arêtes est appelé chemin. Lorsque $u$ et $v$ sont reliés par une seule arête, on dit qu'ils sont dans un voisinage direct. 

\paragraph{Sous-ensemble du graphe}

Lorsqu'on va analyser un graphe, on peut chercher à retrouver des structures d'intérêt. Pour commencer, on peut chercher à identifier un sous graphe. Le sous graphe est une fraction du graphe qui contient un sous-ensemble de n\oe uds de $G(V, E)$ et les arêtes reliant ces n\oe uds. Dans notre exemple, on peut vouloir un sous graphe contenant une liste restreinte de molécules. 

Une autre structure est la clique, qui correspond à un sous-ensemble de n\oe uds tous connectés entre eux. La détection et l'analyse de cliques a de nombreuses applications en bioinformatique, notamment l'identification de groupes de gènes coexprimés.

Pour terminer, une forme de sous-ensemble que j'ai largement utilisée, la composante connexe qui correspond à un ensemble de n\oe uds  tel que, quel que soit $u$, $v$, il existe un chemin  qui les relie. Dans notre exemple, les composantes connexes pourraient être utilisées dans le but d'isoler les voies métaboliques les unes des autres. 

\paragraph{Partitionnement du graphe}

Partitionner un graphe consiste diviser les n\oe uds du graphe en groupe. Chacun de ces groupes est appelé une partie et l'ensemble des parties est appelé partition. En fonction de l'algorithme utilisé, la partition sera alors différente. Dans ce manuscrit, nous utiliserons cette notion de partition à  de nombreuses reprises. 

\subsubsection{Application dans la comparaison des génomes}

L'utilisation des graphes en comparaison de génomes est de plus en plus courante. Le principe, commun à toutes les méthodes, est de représenter des segments de séquences des génomes sous forme de n\oe uds, puis de relier ces segments pour représenter les relations entre différentes variantes des génomes. 


Une première application possible est d'améliorer les méthodes de MSA. Des outils comme MUSCLE ou 
MAFFT \cite{katoh_mafft_2013} utilisent des arbres guides pour améliorer les performances de l'alignement. Ces arbres sont des graphes particuliers, où les séquences sont des n\oe uds et les relations de similarité sont des arêtes.

Une seconde utilisation des graphes concerne l'étude des SNPs, indels et SVs. Ces graphes, appelés graphe de variants, représentent d'une manière flexible les différences entre les génomes. Chaque n\oe ud représente une séquence ou un k-mer, les arêtes vont représenter la colocalisation dans le génome. Ainsi, chaque chemin permet de reconstruire un génome, tout en ayant toutes les variations génétiques. Des outils comme VG toolkit \cite{garrison_variation_2018} et Minigraph \cite{li_design_2020}, permettent notamment d'améliorer l'alignement des lectures en sortie de séquençage, mais aussi d'enrichir la représentation des génomes procaryotes présentant une forte diversité.

Une autre application proche des graphes de variants est celle des graphes de réarrangements. Dans ces graphes, les n\oe uds représentent des synténies conservées, les arêtes vont relier ces synténies en fonction de l'ordre et de l'orientation dans les génomes. L'outil Sibelia \cite{minkin_sibelia_2013} est un outil d'alignement et d'analyse des réarrangements de génomes procaryotes. Il permet  d'étudier les différences évolutives et de reconstruire l'histoire des réplicons.

Deux autres applications, qui m'ont particulièrement intéressées dans ma thèse et que je décrirais ensuite, est l'utilisation pour le regroupement de séquences et pour la construction de pangénome. 
