\subsection{Alignement des séquences}

L'alignement des séquences peut être : pair ou multiple. Dans les deux cas, l'objectif est de trouver l'alignement qui maximise la correspondance entre les résidus. Pour cela, les séquences ne seront pas majoritairement pas alignées entre leur début et leur fin, elles seront décalées. Il existe alors 2 stratégies pour l'alignement : global et local. Dans un alignement global, on fait l'hypothèse que les séquences sont relativement similaires et donc on peut les aligner sur toute leur longueur.% Cette stratégie est bien adaptée pour les séquences proches
Pour un alignement local, on ne fait pas cette hypothèse et on cherche les régions dans la séquence qui ont le plus de similarité sans prendre en compte le reste de la séquence. Les algorithmes et outils que je vais présenter ensuite peuvent généralement s'appliquer au 2 types de stratégies, qu'on choisit en fonction du contexte.

\paragraph{Alignement par paire}

Dès les années 1960, on commence à voir des développements autour de l'idée de comparer 2 séquences (protéiques), mais c'est en 1970 que Needleman et Wunch présentent leur algorithme fondateur des approches de génomique comparée \cite{needleman_general_1970}. Leur algorithme d'alignement global repose sur la construction d'une matrice de similarité, représentant en ligne une séquence et en colonne la seconde, et inclut une pénalité de trou (\textit{gap} en anglais). Ainsi, il est possible de déterminer l'alignement optimal en considérant tous les \textit{gap} sans énumérer toutes les possibilités. Cet algorithme sera revu par Smith et Waterman qui, en 1981, proposent un nouvel algorithme, cette fois pour l'alignement local \cite{smith_identification_1981}. Ces 2 algorithmes ont l'intérêt de donner un résultat de comparaison exacte et sont d'ailleurs encore utilisés aujourd'hui. Toutefois, avec l'augmentation du volume de séquences, la comparaison de paire de séquences utilisant des algorithmes exhaustifs\footnote{Un algorithme exhaustif recherche toutes les solutions possibles pour trouver celle qui exacte ou optimale} pose un problème de complexité quadratique\footnote{La complexité d'un algorithme mesure la consommation de ressources (temps ou espace) nécessaire pour son exécution.}.

En 1985 et 1988, sont publiés les programmes FASTP et FASTA\footnote{Le format de données de l'outil FASTA est aujourd'hui utilisé comme format standard pour écrire les séquences. Les fichiers ont donc pris l'extension .fasta.} \cite{lipman_rapid_1985,pearson_improved_1988} qui marqueront un tournant en utilisant une approche heuristique\footnote{Un algorithme heuristique fournit un résultat rapidement, mais qui n'est pas nécessairement optimal ou exact.}. Le principe est de chercher quelles séquences peuvent être similaires en comparant des mots de tailles k (\textit{k-mer}), pour ensuite ne faire l'alignement exact que sur ce sous-ensemble de séquences. Dans la suite, en 1990, apparait le programme BLAST \cite{altschul_basic_1990}, qui utilise aussi cette approche heuristique et qui sera intégré comme outil dans les bases de données du NCBI, faisant sa renommée.  

Toujours lié à l'augmentation du volume de données, les outils utilisant ces approches heuristiques vont se perfectionner pour permettre l'alignement de paire de séquence de manière rapide et efficace, comme LAST \cite{kielbasa_adaptive_2011} ou DIAMOND \cite{buchfink_fast_2015}. 

\paragraph{Alignement multiple}
\label{paragraph:MSA}

L'alignement multiple des séquences (MSA pour \textit{Multiple Sequence Alignment} en anglais) vise à aligner plusieurs séquences simultanément. C'est une extension de l'alignement en paire. Ces alignements ont l'intérêt de révéler des régions conservées et ainsi d'identifier des relations évolutives, par contre la complexité est accrue et il est donc nécessaire d'introduire des algorithmes plus puissants.

Les premiers algorithmes étaient des algorithmes exhaustifs \cite{stoye_multiple_1998}, et tout comme pour l'alignement de paire de séquences, rapidement des algorithmes heuristiques ont été publiés. En 1988, Higgins et Sharp publient CLUSTAL \cite{higgins_clustal_1988}, une méthode d'alignement progressive pour obtenir un alignement multiple. Elle construit l’alignement en assemblant progressivement les séquences selon une hiérarchie basée sur une matrice de distance ou un arbre guide. D'autres méthodes adopterons cette approche, comme MUSCLE \cite{edgar_muscle_2004} mais ce dernier apporte un côté itératif. Ces méthodes sont rapides, mais peuvent être sensibles aux erreurs accumulées dans les étapes initiales.

D'autres méthodes, que je décrirai ensuite, s'appuient sur des éléments de statistique ou sur des algorithmes de graphes pour être plus efficaces. Il est à noter que toutes ces méthodes ont leur avantage et leurs inconvénients, qui doivent être évalués en fonction du contexte.