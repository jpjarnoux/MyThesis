\chapter{Pangénomique: état des lieux, enjeux et ambitions}

 La pangénomique est un domaine d'étude en plein essor, qui a permis d'explorer et d'analyser les génomes procaryotes sous un nouveau point de vue. Mon travail de thèse s'est concentré sur l'analyse et la comparaison de pangénomes. Dans cette partie, je reviendrai d'abord sur l'origine, les concepts et les défis que pose la pangénomique.  Je présenterai ensuite les différentes modélisations permettant de représenter les génomes en pangénomique, pour poursuivre sur les méthodes de construction de pangénome. Pour terminer, je développerai les méthodes d'analyse existantes en pangénomique. Cette partie sera aussi l'occasion de faire l'état de l'art des outils en pangénomique et de présenter l'outil PPanGGOLiN sur lequel j'ai pu travailler et que j'ai utilisé dans mes développements de thèse.

\section{Origine et concept}

Bien que le terme "pangénome" soit utilisé dans des articles avant 2005, en microbiologie, on s'accorde sur une origine du concept de pangénome proposé dans 2 articles fondateurs \cite{medini_microbial_2005,tettelin_genome_2005}.
L'idée est de ne pas représenter chaque génome individuellement, mais d'utiliser une structure mathématique permettant de les représenter tous simultanément.
Le pangénome représente l'union de toutes les séquences présentes dans un ensemble de génomes. En bioinformatique, la structure, les algorithmes, les méthodes d'analyses des pangénomes, ont constitué un nouveau champ de recherche, la pangénomique.

%\medskip

À partir du pangénome, Tettelin \textit{et al.} proposent de séparer les gènes en 2 catégories, les gènes "\textit{core}" communs à tous les génomes, des gènes "\textit{dispensable}" (ou \textit{accessory}) trouvés dans un sous-ensemble de génomes. En généralisant, le pangénome permet de distinguer l'ensemble des séquences communes à tous les organismes des variations présentes chez certains groupes d'individus, voire spécifiques à un organisme. De ce postulat a émergé l'idée de remplacer les génomes de référence dans les bases de données par des pangénomes de référence \cite{the_computational_pan-genomics_consortium_computational_2018}. Toutefois, ce changement de paradigme n'a pas encore été opéré, car aucune méthode n'a encore réussi à s'imposer comme solution optimale. Trouver une méthode globale est un défi, car la pangénomique est appliquée dans de nombreux domaines de recherche, pour répondre à une grande diversité de questions.

%\medskip

En 2018, le "Computational Pan-Genomics Consortium" met en avant le rôle de la pangénomique dans le développement de solutions applicatives répondant à des problématiques communes à plusieurs disciplines \cite{the_computational_pan-genomics_consortium_computational_2018}. En retour, la pangénomique bénéficie des avancées en phylogénie, métagénomique et intelligence artificielle.
En phylogénie, les méthodes de comparaison génomique à grande échelle et les techniques de construction d'arbres phylogénétiques ont été intégrées aux approches pangénomiques. Réciproquement, la pangénomique permet une meilleure prise en compte des variations génétiques à l'échelle de l'ensemble des génomes, plutôt que de se limiter à un génome de référence, offrant ainsi une vision plus fine de la dynamique évolutive \cite{bazinet_pan-genome_2017}.
Les données métagénomiques représentent un challenge pour la pangénomique. À partir des métagénomes, le pangénome doit être construit en étudiant les relations de co-occurrence des gènes, et non les relations évolutives. Ce changement représente un défi, notamment lorsque les lectures sont courtes. Toutefois, la pangénomique permet d'approfondir l’analyse de la diversité génétique des communautés microbiennes, et de mettre en évidence des adaptations communes à l'environnement ou des co-évolutions et des interactions entre les organismes \cite{the_computational_pan-genomics_consortium_computational_2018}.
L’intelligence artificielle joue également un rôle clé en améliorant l’annotation et la prédiction fonctionnelle des gènes. L’apprentissage automatique est appliqué à la pangénomique pour détecter des motifs génétiques pertinents, prédire des phénotypes et identifier des associations entre mutations et traits phénotypiques \cite{her_pan-genome-based_2018}. Ces méthodes, souvent développées pour d’autres disciplines, ont donc favorisé l’essor de la pangénomique en optimisant l’analyse des données, la reconstruction des génomes et l’interprétation des résultats.

%\medskip

La pangénomique représente une solution à l'analyse de grands volumes de données, à l'heure où le nombre de génomes disponibles dans les banques augmente de façon exponentielle. Entre 2006 et 2024, ce ne sont pas moins de 3 500 articles qui référencent le terme\footnote{Ce chiffre doit être revu à la baisse dû à l'utilisation erronée du terme dans certaines études et une utilisation parfois abusive pour profiter de l'intérêt croissant pour ces analyses}, dont près de 800 concernant les procaryotes (\autoref{fig:panCite}).

\begin{figure}[htbp]
    \centering
    \includegraphics[width=\linewidth]{images/pangenomeCitation.png}
    \caption[Bibliométrie pangénome]{\textbf{Nombre d'articles, référencés dans PubMed, par année, à propos de pangénome du 1er janvier 2004 au 10 février 2025}. La courbe bleue représente le nombre d'articles contenant le terme pangénome dans le titre ou l'abstract : Query=("pan-genome"[Title/Abstract] OR "pangenome"[Title/Abstract] OR "pan-genome"[Title/Abstract]) AND (2004:2025[pdat]). La courbe rouge limite aux publications concernant les procaryotes : Query=("procaryote"[Title/Abstract] OR "bacteria"[Title/Abstract] OR "archeae"[Title/Abstract]) AND ("pan-genome"[Title/Abstract] OR "pangenome"[Title/Abstract] OR "pan-genome"[Title/Abstract]) AND (2004:2025[pdat]). La courbe verte représente tous les articles ou le terme pangénome est trouvé : Query=((pangenome) OR (pan genome)) OR (pan-genome) AND (2004:2025[pdat]). La courbe violette filtre les publications concernant les procaryotes : Query=(((procaryote) OR (bacteria)) OR (archeae)) AND (((pan-genome) OR (pangenome)) OR (pan genome)) AND (2004:2025[pdat]).}
    \label{fig:panCite}
\end{figure}

\newpage
\subsection{Modélisation de la croissance des pangénomes}
\label{sec:croissance_pan}

Dans l'article original de Tettelin \textit{et al.} \cite{tettelin_genome_2005}, les auteurs se sont intéressés à la distribution \textit{core/dispensable} en fonction du nombre de génomes de \textit{Streptococcus agalactiae}\footnote{Bactérie du microbiote intestinale humain et animal, qui est également associé à des infections graves.} que contient le pangénome. Ils observent que lorsque le nombre de génomes augmente, la part de \textit{core genome} décroît de façon exponentielle. Ce résultat les amène à modéliser la croissance du \textit{core genomes} selon une équation exponentielle décroissante. Le modèle permet alors d'estimer la taille du \textit{core genome} pour un nombre de génomes en théorie infinie. Il est alors possible d'estimer la taille du \textit{core genome} d'une espèce à partir d'un échantillon de génome. 

À partir de ce modèle, il est également possible d'estimer la taille du pangénome, \textit{i.e.}, le nombre de gènes unique que contient le pangénome. Ils définissent alors 2 types de pangénomes en fonction de l'estimation de la taille : les \textbf{pangénomes ouverts} et les \textbf{pangénomes fermés}. Les pangénomes sont considérés comme ouverts lorsque l'on ajoute un génome, le nombre de gènes ajouté au pangénome augmentent. Le nombre de gènes est donc théoriquement infini pour un pangénome ouvert avec une infinité de génomes. Les pangénomes fermés quant à eux voient le nombre de nouveaux gènes progressivement diminuer lors de l'ajout de nouveaux génomes. La courbe de prédiction permet d'identifier un plateau théorique du nombre maximal de familles que contiendra le pangénome avec un nombre de génomes infinis. Biologiquement, le pangénome ouvert est attendu pour les espèces sympatriques\footnote{Espèces vivant dans le même environnement que d'autres espèces.} et qui présentent un fort taux de transferts horizontaux, tandis que les espèces vivant dans des niches écologiques ou qui ont une faible capacité d'acquisition de gènes extérieurs vont avoir un pangénome fermé.

\begin{figure}[htbp]
    \centering
    \includegraphics[width=0.85\linewidth]{images/panOpenClose.png}
    \caption[Schéma de croissance du pangénome]{\textbf{Schéma de croissance du pangénome.}}
    \label{fig:panOpenClose}
\end{figure}

Le modèle proposé par Tettelin \textit{et al.}, repose sur l'hypothèse que pour un nombre suffisant de génomes, le nombre de nouveaux gènes apportés par un génome devient constant à partir d'un certain nombre de génomes \cite{tettelin_genome_2005}. Cette hypothèse implique alors que la taille du pangénome est infinie. Cette hypothèse sera questionnée par Hogg \textit{et al.} dans leur étude du pangénome de \textit{Haemophilus influenzae} \cite{hogg_characterization_2007}. Ils vont alors proposer une modélisation basée sur l'hypothèse que le pangénome est fini. Dans leur modèle, chaque gène est associé à une variable aléatoire de Bernoulli, dont la probabilité correspond à la fréquence du gène dans les génomes. Un génome est ainsi généré en observant ces variables : un gène est présent si l’essai est un succès, sinon il est absent. Bien que certains gènes ne soient pas indépendants en raison de structures comme les îlots génomiques, cette hypothèse est conservée pour simplifier le modèle. Les fréquences réelles des gènes étant inconnues, elles sont modélisées de manière probabiliste en répartissant les gènes en $K$ classes distinctes, chacune ayant une fréquence de présence spécifique. À partir de ce modèle, sur le pangénome de \textit{H. influenzae} avec $K=7$, la taille du pangénome est estimée à 5 000 gènes (contre 2 800 gènes dans les 13 génomes de base). Ce modèle sera ensuite amélioré par Snipen \textit{et al.} \cite{snipen_microbial_2009}, qui proposeront une détermination automatique du nombre de classes $K$ et de la fréquence théorique des gènes pour chaque classe. Les modèles binomiaux proposent une perspective dans laquelle la diversité en gènes est finie et qu'il existe un nombre de génomes suffisamment grand pour que tout le répertoire génique soit connu. Cette vision semble de prime abord logique, car le nombre de combinaisons possibles de nucléotides ou d'acides aminés est fini. Pourtant, on peut y opposer que ce nombre, sans le calculer, semble démesuré et qu'il peut être considéré comme infini. De plus, les génomes évoluent continuellement et de nouveaux gènes apparaissent sans cesse. L'utilisation des modèles binomiaux semble alors plus appropriée à des espèces de niche, isolées et présentant un faible taux de transferts horizontaux.

En 2008, Tettelin \textit{et al.} vont proposer une nouvelle modélisation basée sur la loi de Heaps\footnote{Définit de manière empirique en linguistique, cette loi permet de décrire le nombre de mots d'une langue à partir d'un ensemble de documents.} \cite{tettelin_comparative_2008}. On estime le nombre $n$ de gènes distincts, en fonction du nombre $N$ de génomes étudiés, selon la relation :
\begin{equation}
    n=kN^\gamma, 0<\gamma<1,k\geq1
\end{equation}

Le paramètre $k$ est une constante de proportionnalité tandis que $\gamma$ reflète la tendance de la fonction. Ainsi, plus $\gamma$ est proche de 0 plus la croissance en gènes distincts est lente, et plus $\gamma$ est proche de 1 plus la croissance est rapide (\autoref{fig:HeaplawGamma}).

\begin{figure}[htbp]
    \centering
    \subfloat[Courbe de croissance selon la loi de Heap]{\includegraphics[width=0.48\linewidth]{images/HeapsLawgamma.png}
    \label{fig:HeaplawGamma}}
    \hfill
    \subfloat[Courbe de raréfaction selon la loi de Heap]{\includegraphics[width=0.48\linewidth]{images/HeapsLawAlpha.png}
    \label{fig:HeaplawAlpha}}
    \caption[Évolution du pangénome : visualisation de la croissance et de la raréfaction du contenue génique selon la loi de Heap]{\textbf{Évolution du pangénome : visualisation de la croissance et de la raréfaction du contenue génique selon la loi de Heap.}}
    \label{fig:Heaplaw}
\end{figure}

Selon la loi de Heap, le nombre de nouveaux gènes découverts diminue à mesure que l'on ajoute des génomes. On peut formuler ceci selon l'équation : 

\begin{equation}
    p(n)=kN^{(\gamma-1)}=kN^{-\alpha}, \alpha=1-\gamma
\end{equation}

Ainsi, sur la \autoref{fig:HeaplawAlpha}, lorsque $0<\alpha<1$, le taux de nouveaux gènes décroît en ajoutant des génomes, sans jamais être nul. Dans ce cas, le nombre de gènes distincts est croissant. Ce qui implique que si $0<\alpha<1$, le pangénome est ouvert. À partir d'un ensemble de génomes, il est possible d'estimer k et $\alpha$ (ou $\gamma$) et donc de caractériser si le pangénome est ouvert. Si $\alpha\geq1$, alors le taux de nouveaux gènes atteint 0, ce qui correspond à un pangénome fermé. 

\newpage

\subsection{Les différents types de pangénomes}

Les pangénomes peuvent être divisés en 2 catégories en fonction de l'unité choisie pour les construire. Le premier type, celui présenté par Tettelin \textit{et al.} \cite{tettelin_genome_2005}, utilise les gènes comme unité de base du pangénome (\autoref{fig:panType}.B). En regroupant les gènes par homologie (appelé famille de gènes, cf. \autoref{sec:clustering}), il est possible d'obtenir la présence/absence de gènes similaires dans les génomes. Ces pangénomes ont l'avantage d'être moins coûteux en ressources de calcul pour être construits. De plus, ils sont faciles à interpréter, car les gènes sont des unités déjà bien définies et parfois, ils sont même annotés fonctionnellement. Néanmoins, en utilisant les gènes, la méthode d'annotation a un impact important sur le pangénome et il est sensible aux erreurs d'annotation. De plus, les régions non codantes ne sont pas prises en compte dans cette approche. Enfin, les SNPs peuvent passer inaperçus après le regroupement, ainsi que les variants structuraux (SV).

L'autre type de pangénome est basé sur les séquences brutes des génomes. Bien que le terme pangénome n’ait pas encore été employé à l’époque, Chiapello \textit{et al.} \cite{chiapello_systematic_2005} ont proposé une méthode de segmentation des génomes en deux composantes : la "colonne vertébrale", représentant les régions conservées, et les "boucles", qui correspondent aux parties variables. Plus tard, l’outil GenomeMapper \cite{schneeberger_simultaneous_2009}, a explicitement introduit la notion de pangénome de séquence. Son approche repose sur un alignement global des séquences, analysées à travers des k-mers pour différencier les segments conservés des segments variables (\autoref{fig:panType}.C,D). Cette approche a l'intérêt de prendre en compte toute la diversité des génomes (codant, non codant, SNPs et SV). Toutefois, la construction de ces pangénomes est plus coûteuse en ressources. De plus, l'interprétation est plus complexe, car le pangénome n'est pas annoté au départ. Pour terminer, certaines méthodes de construction, utilisent un génome de référence comme séquence de base (\autoref{fig:panType}.C), ce qui peut aussi introduire un biais.

\begin{figure}[htbp]
    \centering
    \includegraphics[width=0.8\linewidth]{images/pangenomeTypes.jpeg}
    \caption[Différents types de pangénomes]{\textbf{Différents types de pangénomes.} Extrait de \cite{matthews_gentle_2024}}
    \label{fig:panType}
\end{figure}

\newpage
\section{Pangénome de séquences}

\subsection{Pangénome basé sur une séquence représentative}

Un pangénome basé sur les séquences correspond à un ensemble de génomes dont l'alignement minimise le nombre de régions homologues tout en rendant compte de toute la diversité. L'objectif derrière ces pangénomes est d'obtenir une séquence pangénomique de référence. De façon contre-intuitive (par rapport à la définition "sans-référence" des pangénomes), pour construire ces pangénomes, on utilise une séquence représentative comme base. Toutes les séquences seront alignées à partir de cette base, et les segments non redondants détectés dans au moins un génome seront intégrés à la référence non redondante (NRR, Non-Redundant Reference en anglais). L'ensemble, séquence représentante et NRRs, forme alors la séquence pangénomique de référence.

\subsubsection{Méthode de construction}

Pour construire ces pangénomes, il faut d'abord identifier une séquence représentative. Les autres séquences, en général des séquences non assemblées (lectures ou \textit{reads} en anglais), sont alignées contre la représentante et les séquences non alignables sont considérées comme des NRRs potentiels. Les NRRs de taille inférieure à 500 pb sont exclues, ainsi que celles dont la similarité avec la représentante est supérieure à un seuil (90 \% en général). Les NRRs restantes sont comparées à des bases de données pour retirer tous les contaminants potentiels. De ce schéma général, on peut identifier 4 méthodes différentes pour l'identification des NRRs potentiels :
\begin{itemize}
    \item \textbf{Assemblage de type métagénomique} : les lectures non alignées sur la référence sont regroupées et assemblées \textit{de novo}. Les contigs obtenus sont ajoutés à la séquence représentante. Cette méthode est efficace même avec une faible couverture des lectures.
    \item \textbf{Assemblage itératif} : Dans un premier temps, les lectures non alignées du premier échantillon sont assemblées et ajoutées au génome de référence. Ce génome mis à jour sert ensuite de base pour l’assemblage des échantillons suivants. Ce processus est répété pour tous les échantillons.
\end{itemize}
\newpage
\begin{itemize}
    \item \textbf{Assemblage indépendant des \textit{reads} non alignés} : Toutes les lectures non alignées sont séparées par échantillon\footnote{Ensemble de lectures obtenues simultanément} et assemblées \textit{de novo} indépendamment. Les contigs obtenus sont regroupés selon leur similarité. Dans chaque groupe, un contig référent est identifié et est intégré à la séquence référente. Cette méthode nécessite une couverture d’au moins 10×, pour obtenir des contigs de taille suffisante.
    \item \textbf{Assemblage génomique indépendant} : chaque échantillon est assemblé indépendamment, et les contigs obtenus sont alignés à la référence. Les contigs non alignés sont regroupés par similarité et un contig référent est ajouté à la séquence référente.
\end{itemize}

Le choix de la méthode dépend du type et de la quantité des données disponibles. Avec une faible couverture (<10×) et un grand nombre d’échantillons, l’approche métagénomique est recommandée, bien qu’elle puisse produire des contigs chimériques. Avec une couverture plus élevée (>10×), l’assemblage indépendant ou l’approche itérative sont préférables. Cette dernière est plus lente, mais facilite l’ajout de nouveaux échantillons. Enfin, si plusieurs assemblages de haute qualité existent déjà, l’assemblage génomique indépendant est la meilleure option. Ces méthodes peuvent être combinées pour optimiser l’utilisation des données disponibles.
 
\subsubsection{Domaines d'application des pangénomes basés sur une séquence représentative}

Ces pangénomes sont particulièrement utiles lorsque les données de départ sont des \textit{lectures}. En utilisant ces modèles, il est possible de revenir à une séquence linéaire qui peut être utilisée dans les outils classiques de génomique. De plus, il peut également être utilisé comme étape préliminaire à la construction d'autres types de pangénomes, en réduisant rapidement la redondance dans un sous-ensemble proche de génomes. 

L'outil NGSPanPipe \cite{kulsum_ngspanpipe_2018} est un pipeline intégré conçu pour l'identification du pangénome à partir de lectures courtes (short reads) issues du séquençage de nouvelle génération (NGS). Contrairement à d'autres méthodes nécessitant des prétraitements des lectures, NGSPanPipe permet une analyse directe des reads bruts pour identifier le pangénome. Il ne génère pas de séquence pangénomique linéaire, mais il permet de reconstruire des contigs à partir des lectures en utilisant un génome de référence. Les contigs obtenus à partir des lectures alignées, permettent de calculer la couverture du génome par rapport au pangénome. Les lectures non alignées sont comparées à des bases de données de \textit{reads} pour identifier de nouveaux \textit{reads}, puis ils sont assemblés en contigs. L'ensemble des contigs (de lectures alignées et non alignées) sont annotés et utilisés pour construire une matrice binaire représentant la présence ou l'absence des gènes dans la séquence de référence.

\subsection{Pangénome graphe}

Les graphes de séquences sont un modèle de pangénome permettant de visualiser la diversité génomique, qu’elle soit basée sur une séquence de référence ou non. Dans tous les cas, des segments de séquences vont constituer les n\oe uds du pangénome et les arêtes seront étiquetées par des informations permettant de retrouver le lien entre les segments (comme l'organisation dans les génomes). Ce modèle pangénomique a l'intérêt de représenter toute la diversité, codant et non codant. 

\subsubsection{Méthodes et outils de construction}

\paragraph{Graphe de variant prédéterminé}

La première méthode de construction des graphes de pangénome se base sur l'utilisation d'une séquence référente et d'un fichier contenant les variations connues dans les autres séquences par rapport à cette référence. Cette méthode a l'intérêt de demander peu de ressources, car les variations sont prédéterminées et données en entrée. Toutefois, pour obtenir un graphe fiable et précis, un génome complet de bonne qualité est requis.

L'outil VG (\textit{Variation Graph toolkit}) \cite{garrison_variation_2018}, contient un ensemble d'outils permettant de générer un graphe de variants. À partir de ce graphe, qui peut être assimilé à un graphe de pangénome, il est possible de détecter les variants structuraux (SVs) et les SNPs rapidement. Le graphe est indexé, rendant les recherches et l'alignement plus efficaces, notamment dans l'alignement de lectures ou dans la recherche de variants génétiques (\textit{variant calling}). L'outil a d'abord été développé pour la génomique humaine, mais il est tout à fait possible de l'utiliser avec des génomes procaryotes.

L'outil Minigraph \cite{li_design_2020}, lui aussi développé pour le variant calling sur le génome humain, propose une méthode demandant moins de ressources que VG. Le graphe est plus léger, sans annotation, permettant de construire des graphes de pangénome de grande taille, en utilisant peu de mémoire de calcul et de stockage. Minigraph permet de capturer les grandes variations génomiques, mais est moins performant sur la détection des SNPs par rapport à VG.

\paragraph{Graphe d'alignement multiple}

Une méthode, proche de la précédente, est celle basée sur l'alignement multiple des séquences (MSA\footnote{cf. \autoref{paragraph:MSA}}) entre elles. Cette méthode n'est pas dépendante d'une séquence référente. Le MSA permet de déterminer les variations entre les séquences, ce qui augmente le coût en ressources par rapport au graphe de variants prédéterminé. Toutefois, cette méthode est plus adaptée dans le cas où plusieurs séquences de bonne qualité sont disponibles pour construire le pangénome. En effet, le MSA permet de se passer du biais de la séquence référente dans la construction du graphe et d'ainsi mieux représenter la diversité génomique.

L'outil Harvest \cite{treangen_harvest_2014}, permet de comparer des génomes étroitement apparentés. Pour optimiser l'étape d'alignement, il utilise l'outil progressiveMauve \cite{darling_progressivemauve_2010}, qui fait un alignement progressif des séquences. Après l'alignement, il identifie le \textit{core genome} dans le pangénome et génère une phylogénie basée sur une matrice des SNPs. Bien qu'étant rapide et efficace, il n'est pas adapté aux génomes très divergents et il ne permet pas d'analyser les éléments mobiles (MGE).

PGGB \cite{garrison_building_2024}, utilise des algorithmes de graphes de préfixes minimaux (MPHF\footnote{Minimal Perfect Hash Function (MPHF) est une fonction qui associe de manière unique chaque élément d’un ensemble sans collisions et avec un espace mémoire minimal}), pour compresser le graphe et optimiser l'alignement. Il est capable d'identifier et de représenter les SNPs, SV, et les MGEs de manière efficace. PGGB est conçu pour mener des études pangénomiques à grande échelle, prenant en compte de grandes quantités de séquences, ce qui demande des ressources disponibles importantes. De plus, c'est un outil assez complet pour les analyses, ce qui peut le rendre difficile d'accès.

\paragraph{Graphe de De Bruijn}

Les graphes de De Bruijn (De Bruijn Graph : DBG) sont des graphes orientés dont les n\oe uds représentent des k-mers et les arêtes le chevauchement entre le suffixe et le préfixe (de taille k-1) des k-mers (\autoref{fig:deBruijn}). Ainsi, en suivant un chemin, il est possible de reconstituer une séquence. C'est pourquoi les DBG sont utilisés dans de nombreuses applications en bioinformatique (assemblage, correction des erreurs de séquençage, métagénomique\dots) et notamment en pangénomique.

\begin{figure}[htbp]
    \centering
    \includegraphics[width=.9\linewidth]{images/DBG.png}
    \caption[Exemple d'un graphe de De Bruijn]{\textbf{Exemple d'un graphe de De Bruijn.} Ici $k=3$, ce graphe permet de représenter et de reconstruire 3 séquences.}
    \label{fig:deBruijn}
\end{figure}

Les DBG, permettent d'avoir une structure compacte des séquences du pangénome. Les n\oe uds et les arêtes sont colorées en fonction des génomes dans lesquels ils sont retrouvés. Les DBG peuvent être compactés en cDBG, en fusionnant chaque région \textit{core}, \textit{i.e.} chaque suite de n\oe uds avec une seule arête entre chaque n\oe ud. Ces nouveaux n\oe uds fusionnés sont appelés "\textit{unitig"} et seront étiquetés avec la séquence combinée des k-mers.

L'une des premières méthodes développées utilisant des DBG est la méthode Cortex \cite{iqbal_novo_2012}, qui construit un DBG "coloré" (les arêtes et les n\oe uds sont étiquetés par les échantillons dans lesquels ils sont trouvés). À partir de ce DBG coloré, il est possible d'identifier les variants et de les associer à un génotype. Des outils plus récents, comme Bifrost \cite{holley_bifrost_2020}, améliorent les méthodes de coloration de DBG, permettant d'augmenter le volume de données pris en compte et supportant la mise à jour du graphe. Les auteurs de Bifrost ont notamment appliqué leur méthode sur une collection de plus de 100 000 génomes de \textit{Salmonella} \cite{luhmann_blastfrost_2021}, leur permettant d'identifier des gènes reliés à des îlots de pathogénicité et à une résistance aux fluoroquinolones\footnote{Classe d'antibiotique utilisée pour traiter les infections bactériennes graves.}.

\newpage
SplitMEM \cite{marcus_splitmem_2014}, permet de construire rapidement et efficacement des cDBG en intégrant une méthode appelée "saut de suffixe"\footnote{Le cDBG est relié à des arbres de suffixes, un saut de suffixe permet depuis un suffixe à l'extrémité d'une branche de l'arbre de sauté vers un même suffixe plus proche de la racine. Les sauts se poursuivent jusqu'à atteindre le suffixe le plus proche de la racine. Le chemin restant correspond au chemin le plus court sans ramification, entre la racine et le suffixe.}, qui permet de construire le cDBG sans passer par un DBG. L'outil permet ensuite de détecter dans l'ensemble des génomes ou dans un sous-ensemble de génomes les régions compressées (appelées \textit{Maximum Exact Matches} : MEMs), correspondant au \textit{core genome}. Cet outil est linéaire en temps et en espace pour identifier le \textit{core genome}, mais ne permet pas de mener d'autres analyses. De plus, la méthode a été testée sur un jeu de 62 génomes de \textit{E. coli}, le caractère linéaire est donc à vérifier sur de plus grands jeux de données.

PanTools \cite{sheikhizadeh_pantools_2016}, est un outil complet qui a largement évolué depuis sa publication. Il permet la construction de pangenomes basés sur des cDBG généralisés. PanTools est robuste à l'utilisation de grands volumes de données, que ce soit en temps, en mémoire ou en stockage. Il intègre également des méthodes d'annotation structurale et fonctionnelle, de partitionnement, d'alignement, de phylogénie, d'identification du \textit{core genome} et de visualisation. 

DBGWAS \cite{jaillard_fast_2018}, construit également les pangénomes avec des cDBG. Son originalité réside dans l'association de phénotypes (\textit{Genome Wild Association Study} : GWAS). L'intérêt d'utiliser le graphe de pangénome est qu'il n'est pas nécessaire d'utiliser une séquence de référence, contrairement aux approches classiques de GWAS. De plus, les phénotypes ne sont pas associés à des SNPs mais à des sous-graphes, permettant d'extraire des variants plus longs ou plus complexes. DBGWAS intègre une partie visualisation, permettant d'explorer les variants associés au phénotype dans leur contexte génomique pour identifier des régions variables plus larges comme les îlots génomiques. 

Le DBG (et ses dérivés) est une structure de données puissante, permettant de calculer, analyser et stocker, rapidement et efficacement, de grandes quantités de données. Néanmoins, ce qui fait la force de cette approche (l'utilisation de k-mer) est aussi sa faiblesse. Le choix de la taille du k-mer va influencer le graphe et donc la découverte des variations. De plus, cette structure est limitée dans l'identification et l'étude des régions répétées. C'est pourquoi des auteurs proposent des méthodes pour lier des informations au DBG \cite{turner_integrating_2018}.

\subsubsection{Application des graphes de pangénome.}

L'utilisation de pangénomes de séquence est très utile à partir de lectures courtes pour améliorer le génotypage. En utilisant le pangénome, contenant des variants connus, on améliore la couverture des lectures et donc on améliore le génotypage de ces lectures. Par rapport aux méthodes utilisant un génome de référence, le pangénome réduit le biais en faveur de la séquence de référence, particulièrement pour les grandes insertions/délétions et les SV. Le pangénome améliore aussi le \textit{variant calling} (VC) en augmentant sa précision, et à partir des DBG de faire du VC sans référence.

Les graphes de séquences sont également utilisés en métagénomique. L'outil MetaKallisto \cite{schaeffer_pseudoalignment_2017} utilise notamment une base de données de séquences représentantes qu'il représente sous forme de DBG coloré afin de faire de l'assignation taxonomique et de la quantification de séquences métagénomiques.

L'utilisation des graphes de séquences pour les GWAS permet de détecter finement des variations dans les populations associées à un phénotype. Chaguza \textit{et al.} \cite{chaguza_bacterial_2020} ont mené une étude sur 909 échantillons de souche hyper virulente de \textit{Streptococcus pneumoniae} (serotype 1). Ils ont pu identifier, grâce à l'outil DBGWAS, des mutations de certaines protéines associées à des phénotypes spécifiques (âge de l'hôte, géographie, structure des populations\dots). L'utilisation de graphes de pangénome a permis de mener une étude à large échelle, tout en prenant en compte toute la diversité sans nécessiter de référence.

\begin{table}[htbp]
    \centering
    \begin{tabular}{|p{.25\textwidth}|p{.3\textwidth}|p{.35\textwidth}|}
        \hline
        Nom & Méthode & Référence \\
        \hline
        NGSPanPipe & Séquence représentative & \cite{kulsum_ngspanpipe_2018} \\
        \hline
        Spine & Séquence représentative & \cite{ozer_characterization_2014}\\
        \hline
        VG toolkit & Variant prédéterminé & \cite{garrison_variation_2018} \\
        \hline
        Minigraph & Alignement sur graphe & \cite{li_design_2020} \\
        \hline
        PanVC & Variant prédéterminé & \cite{norri_founder_2021} \\
        \hline
        Minigraph-Cactus & MSA & \cite{hickey_pangenome_2024}\\
        \hline
        Harvest & MSA & \cite{treangen_harvest_2014} \\
        \hline
        PGGB & MSA & \cite{garrison_building_2024}\\
        \hline
        Cortex & graphe de De Bruijn & \cite{iqbal_novo_2012} \\
        \hline
        Bifrost & graphe de De Bruijn & \cite{holley_bifrost_2020} \\
        \hline
        SplitMEM & graphe de De Bruijn & \cite{marcus_splitmem_2014} \\
        \hline
        PanTools & graphe de De Bruijn & \cite{sheikhizadeh_pantools_2016} \\
        \hline
        twoPaCo & graphe de De Bruijn & \cite{minkin_twopaco_2017}\\
        \hline
        DBGWas & graphe de De Bruijn & \cite{jaillard_fast_2018}\\
        \hline
        PanVA & Visualisation & \cite{van_den_brandt_panva_2024} \\
        \hline
    \end{tabular}
    \caption[Outils de pangénomique basés sur les séquences]{\textbf{Liste non exhaustive d'outils de pangénomique basés sur les séquences.}}
    \label{tab:pangenomicToolsSeq}
\end{table}
\section{Pangenome de gènes}

\subsection{Famille de gènes homologues}
\label{sec:fam}

\newpage
\section{Conclusion sur les pangénomes et éléments de réflexions}

Nous avons présenté une grande variété de méthodes et d'outils de pangénomique (\textit{cf.} tableaux \ref{tab:pangenomicToolsSeq} et \ref{tab:pangenomicToolsFam}) et exploré plusieurs champs d'application des pangénomes. Cependant, plusieurs défis majeurs restent à relever en pangénomique.

Un premier défi concerne la représentation et la visualisation des pangénomes. En effet, ceux-ci doivent être visualisables par l'\oe il humain tout en intégrant l’ensemble de l’information génomique. Différentes approches ont été développées pour répondre à cette problématique, notamment des outils de visualisation interactifs. À titre d’exemple, Pan-Tetris\cite{hennig_pan-tetris_2015} permet de visualiser et de modifier la composition des groupes de gènes grâce à une technique inspirée du jeu Tetris. PanViz\cite{pedersen_panviz_2017} facilite la comparaison des génomes individuels aux pangénomes avec une navigation basée sur l'ontologie des gènes. PanVa\cite{van_den_brandt_panva_2024}, utilisant les pangénomes de PanTools\cite{sheikhizadeh_pantools_2016}, propose une approche centrée sur la variabilité structurale, permettant une analyse flexible des pangénomes en tenant compte des variations génomiques complexes. Enfin, PANACHE\cite{durant_panache_2021} propose une visualisation linéarisée des pangénomes, affichant la présence ou l’absence des blocs de séquences sous forme de navigateur génomique.

Un second défi crucial est celui du stockage et de la gestion des données pangénomiques. Contrairement aux génomes individuels, généralement stockés sous forme de texte et reliés à des bases de données, les pangénomes sont des structures plus complexes qui ne peuvent être stockées sous un format linéaire. Les BD doivent donc permettre un accès rapide et efficace aux informations, tout en étant mises à jour régulièrement pour suivre l’augmentation du volume des données génomiques. Des BD comme panKB \cite{sun_pankb_2025}, permettent d'avoir accès à des informations sur des pangénomes précalculés, reliés à des métadonnées comme : les publications de référence, les informations sur l'origine des génomes\dots 
Cependant, le pangénome en lui-même n'est pas disponible et seuls les résultats d'analyse sont disponibles.

Enfin, un défi essentiel concerne la construction du pangénome. Le choix des méthodes influence fortement le résultat final, mais les données d’entrée jouent également un rôle fondamental. L’objectif étant de représenter au mieux la diversité génomique, il est important de maximiser cette diversité tout en évitant les biais, tels que la surreprésentation de génomes pathogènes dans les bases de données. Un équilibre est nécessaire : trop de variations dans les données d’entrée peuvent complexifier l’analyse, en rendant par exemple les graphes de pangénome difficilement exploitables. La qualité des génomes, des annotations et des bases de données, sont également des facteurs déterminants pour garantir la robustesse des analyses pangénomiques.

Pour répondre à ces défis, les méthodes et les outils de pangénomique sont en constante évolution. Avec ces progrès, divers domaines de la microbiologie voient progressivement s'intégrer des analyses pangénomiques en routine. Cette intégration repose en partie sur des plateformes qui rendent ces outils accessibles et exploitables par la communauté scientifique. Par exemple, MicroScope\cite{vallenet_microscope_2020} permet de construire des pangénomes procaryotes à l’aide de PPanGGOLiN \cite{gautreau_ppanggolin_2020} et d’identifier les régions de plasticité génomique (RGP) grâce à la méthode panRGP\cite{bazin_panrgp_2020}. MicroScope constitue alors un point d'intersection entre les données génomiques, la production de pangénomes et leur utilisation effective, contribuant ainsi à relever progressivement les défis liés à la visualisation, au stockage et à la construction des pangénomes.